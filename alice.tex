%%% Tested & Worked using:
%%%
%%%  - XeTeX 3.1415926-2.5-0.9999.3-2014012222 (TeX Live 2013/Debian)
%%%  - kpathsea version 6.1.1
%%%  - Ubuntu Gnome 14.10
%%%
%%% Typed and compiled by Oto Bolyós in 2015-2016
%%%
%%% This compilation is licenced under GNU GPLv2.
%%% The text of the book is from Project Gutenberg (gutenberg.org/ebooks/11)
%%% and is licensed under the Project Gutenberg License which can be find
%%% at gutenberg.org/wiki/Gutenberg:The_Project_Gutenberg_License#START:_FULL_LICENSE.
%%%
%%%
%%%%%%%%%%%%%%%%%%%%%%%%%%%%%%%%%%%%%%%%%%%%%%%
%%%%                                       %%%%
%%%%   Alice’s Adventures in Wonderland    %%%%
%%%%                  by Lewis Carroll     %%%%
%%%%                                       %%%%
%%%%%%%%%%%%%%%%%%%%%%%%%%%%%%%%%%%%%%%%%%%%%%%

\documentclass[12pt]{book}
%\usepackage[utf8]{inputenc}
%\usepackage[slovak, english]{babel}
%\usepackage[T1, T2A]{fontenc}
%\usepackage{lipsum} % For dummy text
\usepackage[a4paper]{geometry}
 \geometry{
 a4paper,
% total={210mm,297mm},
 left=0.32in,
 right=0.32in,
 top=0.32in,
 bottom=0.32in,
 portrait
 }

 \usepackage{parallel}

%\usepackage{amsmath}

\usepackage{xunicode}% for xetex!
\usepackage{fontspec}% for xetex!
\usepackage{xltxtra}% for xetex!
\usepackage{url} % for xetex to break long urls at line ending
%\usepackage{ngerman} % choose your language here
\usepackage{multicol} % erlaubt es mit \begin un \end teilbereiche mehrspaltig zu setzen

\usepackage{letltxmacro} % needed for closed multiple sqrt
% closed mutiple sqrt settings
\LetLtxMacro{\oldsqrt}{\sqrt} % makes all sqrts closed
\renewcommand{\sqrt}[1][\ ]{%
  \def\DHLindex{#1}\mathpalette\DHLhksqrt}
\def\DHLhksqrt#1#2{%
  \setbox0=\hbox{$#1\oldsqrt[\DHLindex]{#2\,}$}\dimen0=\ht0
  \advance\dimen0-0.2\ht0
  \setbox2=\hbox{\vrule height\ht0 depth -\dimen0}%
  {\box0\lower0.71pt\box2}}


%\setromanfont[Mapping=tex-text, Numbers=OldStyle]{Linux Libertine O}

% german settings (decimal comma, etc)
\usepackage{ziffer}

% multicolumns
\usepackage{multicol}
\setlength{\columnsep}{.1in}

% set paragraph indentation
\setlength{\parindent}{0em}

% par spacing
\setlength{\parskip}{0.08in}

% line spacing
\renewcommand{\baselinestretch}{1.0}

% add hyperref plugin

\IfFileExists{libertine-type1.sty}{ % Check for libertine-type1 package
    \newcommand\libertine{LinuxLibertineO-OsF}
    \usepackage[oldstyle]{libertine}
    \fontfamily{LinuxLibertineO-OsF}\selectfont
}{
    \newcommand\libertine{fxlj}
    \usepackage[osf]{libertine}
    \usepackage[T1]{fontenc}
%    \usepackage[libertine]{newtxmath}

%    \setmathfont{texgyrepagellamath-regular.otf}
    \usepackage[libertine]{newtxmath}
}

\usepackage{fontspec,unicode-math}
%\usepackage{}
\usepackage{xltxtra}

\usepackage{titlesec} % to customise chapters and sections

\titlespacing*{\section}{0pt}{.17in}{.08in}
\titlespacing*{\subsection}{0pt}{.17in}{.08in}

\usepackage{sectsty}
\sectionfont{\addfontfeature{Letters=SmallCaps}\centering}

\removenolimits{\int}

\usepackage{bookmark}
\bookmarksetup{
  numbered,
  open
}

\title{Alice’s Adventures in Wonderland\protect\\Alica v krajine zázrakov}
\author{\textit{Lewis Carroll}}
\date{}

%% TODO
% - restart numbering for sk part
% - align en and sk text (at least the sections)
% - page numbering
% - greater margins (?)
% - format (a5?)
% - minimise the packages (and/or fonts) to generate it more quickly
% - add header (book title, section title, auther, line)
% - %2d section numbering? ('cos of the pdf bookmarks)
% - check the format (esp poems)
% - proofread sk translation
% - insert blank page before toc
% - change toc title to en&sk instead of de
% - insert dots in toc before page number

%%%%%%%%%%%%%%%%%%%%%%%%%%%%%%%%%%%%%%%%%%%%%%%%%%%%%%%%%%%%%%%%%%%%%%%%%%%%%%%%%%%%%%%%%%%%%%%%%%%%%%
%%%%%%%%%%%%%%%%%%%%%%%%%%%%%%%%%%%%%%%%%%%%%%%%%%%%%%%%%%%%%%%%%%%%%%%%%%%%%%%%%%%%%%%%%%%%%%%%%%%%%%

\begin{document}
\maketitle
\begin{center}Written in \XeTeX. | Naspísané v \XeTeX-u.

English text is from gutenberg.org. | Slovenský preklad je prepis tlačenej knihy z vydavateľstva Mladé letá z r. 1984.
\end{center}

\tableofcontents{}

\pagenumbering{arabic}

\begin{Parallel}[p]{}{}
\ParallelLText{  % closing bracket is after the en text

\selectlanguage{english}

% first chapter
\chaper{}
\section{Down the Rabbit-Hole}
Alice was beginning to get very tired of sitting by her sister on the bank, and of having nothing to do: once or twice she had peeped into the book her sister was reading, but it had no pictures or conversations in it, ‘and what is the use of a book,’ thought Alice ‘without pictures or conversations?’

So she was considering in her own mind (as well as she could, for the hot day made her feel very sleepy and stupid), whether the pleasure of making a daisy-chain would be worth the trouble of getting up and picking the daisies, when suddenly a White Rabbit with pink eyes ran close by her.

There was nothing so VERY remarkable in that; nor did Alice think it so VERY much out of the way to hear the Rabbit say to itself, ‘Oh dear! Oh dear! I shall be late!’ (when she thought it over afterwards, it occurred to her that she ought to have wondered at this, but at the time it all seemed quite natural); but when the Rabbit actually TOOK A WATCH OUT OF ITS WAISTCOAT-POCKET, and looked at it, and then hurried on, Alice started to her feet, for it flashed across her mind that she had never before seen a rabbit with either a waistcoat-pocket, or a watch to take out of it, and burning with curiosity, she ran across the field after it, and fortunately was just in time to see it pop down a large rabbit-hole under the hedge.

In another moment down went Alice after it, never once considering how in the world she was to get out again.

The rabbit-hole went straight on like a tunnel for some way, and then dipped suddenly down, so suddenly that Alice had not a moment to think about stopping herself before she found herself falling down a very deep well.

Either the well was very deep, or she fell very slowly, for she had plenty of time as she went down to look about her and to wonder what was going to happen next. First, she tried to look down and make out what she was coming to, but it was too dark to see anything; then she looked at the sides of the well, and noticed that they were filled with cupboards and book-shelves; here and there she saw maps and pictures hung upon pegs. She took down a jar from one of the shelves as she passed; it was labelled ‘ORANGE MARMALADE’, but to her great disappointment it was empty: she did not like to drop the jar for fear of killing somebody, so managed to put it into one of the cupboards as she fell past it.

‘Well!’ thought Alice to herself, ‘after such a fall as this, I shall think nothing of tumbling down stairs! How brave they’ll all think me at home! Why, I wouldn’t say anything about it, even if I fell off the top of the house!’ (Which was very likely true.)

Down, down, down. Would the fall NEVER come to an end! ‘I wonder how many miles I’ve fallen by this time?’ she said aloud. ‘I must be getting somewhere near the centre of the earth. Let me see: that would be four thousand miles down, I think—’ (for, you see, Alice had learnt several things of this sort in her lessons in the schoolroom, and though this was not a VERY good opportunity for showing off her knowledge, as there was no one to listen to her, still it was good practice to say it over) ‘—yes, that’s about the right distance—but then I wonder what Latitude or Longitude I’ve got to?’ (Alice had no idea what Latitude was, or Longitude either, but thought they were nice grand words to say.)

Presently she began again. ‘I wonder if I shall fall right THROUGH the earth! How funny it’ll seem to come out among the people that walk with their heads downward! The Antipathies, I think—’ (she was rather glad there WAS no one listening, this time, as it didn’t sound at all the right word) ‘—but I shall have to ask them what the name of the country is, you know. Please, Ma’am, is this New Zealand or Australia?’ (and she tried to curtsey as she spoke—fancy CURTSEYING as you’re falling through the air! Do you think you could manage it?) ‘And what an ignorant little girl she’ll think me for asking! No, it’ll never do to ask: perhaps I shall see it written up somewhere.’

Down, down, down. There was nothing else to do, so Alice soon began talking again. ‘Dinah’ll miss me very much to-night, I should think!’ (Dinah was the cat.) ‘I hope they’ll remember her saucer of milk at tea-time. Dinah my dear! I wish you were down here with me! There are no mice in the air, I’m afraid, but you might catch a bat, and that’s very like a mouse, you know. But do cats eat bats, I wonder?’ And here Alice began to get rather sleepy, and went on saying to herself, in a dreamy sort of way, ‘Do cats eat bats? Do cats eat bats?’ and sometimes, ‘Do bats eat cats?’ for, you see, as she couldn’t answer either question, it didn’t much matter which way she put it. She felt that she was dozing off, and had just begun to dream that she was walking hand in hand with Dinah, and saying to her very earnestly, ‘Now, Dinah, tell me the truth: did you ever eat a bat?’ when suddenly, thump! thump! down she came upon a heap of sticks and dry leaves, and the fall was over.

Alice was not a bit hurt, and she jumped up on to her feet in a moment: she looked up, but it was all dark overhead; before her was another long passage, and the White Rabbit was still in sight, hurrying down it. There was not a moment to be lost: away went Alice like the wind, and was just in time to hear it say, as it turned a corner, ‘Oh my ears and whiskers, how late it’s getting!’ She was close behind it when she turned the corner, but the Rabbit was no longer to be seen: she found herself in a long, low hall, which was lit up by a row of lamps hanging from the roof.

There were doors all round the hall, but they were all locked; and when Alice had been all the way down one side and up the other, trying every door, she walked sadly down the middle, wondering how she was ever to get out again.

Suddenly she came upon a little three-legged table, all made of solid glass; there was nothing on it except a tiny golden key, and Alice’s first thought was that it might belong to one of the doors of the hall; but, alas! either the locks were too large, or the key was too small, but at any rate it would not open any of them. However, on the second time round, she came upon a low curtain she had not noticed before, and behind it was a little door about fifteen inches high: she tried the little golden key in the lock, and to her great delight it fitted!

Alice opened the door and found that it led into a small passage, not much larger than a rat-hole: she knelt down and looked along the passage into the loveliest garden you ever saw. How she longed to get out of that dark hall, and wander about among those beds of bright flowers and those cool fountains, but she could not even get her head through the doorway; ‘and even if my head would go through,’ thought poor Alice, ‘it would be of very little use without my shoulders. Oh, how I wish I could shut up like a telescope! I think I could, if I only knew how to begin.’ For, you see, so many out-of-the-way things had happened lately, that Alice had begun to think that very few things indeed were really impossible.

There seemed to be no use in waiting by the little door, so she went back to the table, half hoping she might find another key on it, or at any rate a book of rules for shutting people up like telescopes: this time she found a little bottle on it, (‘which certainly was not here before,’ said Alice,) and round the neck of the bottle was a paper label, with the words ‘DRINK ME’ beautifully printed on it in large letters.

It was all very well to say ‘Drink me,’ but the wise little Alice was not going to do THAT in a hurry. ‘No, I’ll look first,’ she said, ‘and see whether it’s marked “poison” or not’; for she had read several nice little histories about children who had got burnt, and eaten up by wild beasts and other unpleasant things, all because they WOULD not remember the simple rules their friends had taught them: such as, that a red-hot poker will burn you if you hold it too long; and that if you cut your finger VERY deeply with a knife, it usually bleeds; and she had never forgotten that, if you drink much from a bottle marked ‘poison,’ it is almost certain to disagree with you, sooner or later.

However, this bottle was NOT marked ‘poison,’ so Alice ventured to taste it, and finding it very nice, (it had, in fact, a sort of mixed flavour of cherry-tart, custard, pine-apple, roast turkey, toffee, and hot buttered toast,) she very soon finished it off.

\begin{center}
\quad*\quad*\quad*\quad*\quad*\quad*\quad*
\par
\quad*\quad*\quad*\quad*\quad*\quad*
\par
\quad*\quad*\quad*\quad*\quad*\quad*\quad*
\end{center}

‘What a curious feeling!’ said Alice; ‘I must be shutting up like a telescope.’

And so it was indeed: she was now only ten inches high, and her face brightened up at the thought that she was now the right size for going through the little door into that lovely garden. First, however, she waited for a few minutes to see if she was going to shrink any further: she felt a little nervous about this; ‘for it might end, you know,’ said Alice to herself, ‘in my going out altogether, like a candle. I wonder what I should be like then?’ And she tried to fancy what the flame of a candle is like after the candle is blown out, for she could not remember ever having seen such a thing.

After a while, finding that nothing more happened, she decided on going into the garden at once; but, alas for poor Alice! when she got to the door, she found she had forgotten the little golden key, and when she went back to the table for it, she found she could not possibly reach it: she could see it quite plainly through the glass, and she tried her best to climb up one of the legs of the table, but it was too slippery; and when she had tired herself out with trying, the poor little thing sat down and cried.

‘Come, there’s no use in crying like that!’ said Alice to herself, rather sharply; ‘I advise you to leave off this minute!’ She generally gave herself very good advice, (though she very seldom followed it), and sometimes she scolded herself so severely as to bring tears into her eyes; and once she remembered trying to box her own ears for having cheated herself in a game of croquet she was playing against herself, for this curious child was very fond of pretending to be two people. ‘But it’s no use now,’ thought poor Alice, ‘to pretend to be two people! Why, there’s hardly enough of me left to make ONE respectable person!’

Soon her eye fell on a little glass box that was lying under the table: she opened it, and found in it a very small cake, on which the words ‘EAT ME’ were beautifully marked in currants. ‘Well, I’ll eat it,’ said Alice, ‘and if it makes me grow larger, I can reach the key; and if it makes me grow smaller, I can creep under the door; so either way I’ll get into the garden, and I don’t care which happens!’

She ate a little bit, and said anxiously to herself, ‘Which way? Which way?’, holding her hand on the top of her head to feel which way it was growing, and she was quite surprised to find that she remained the same size: to be sure, this generally happens when one eats cake, but Alice had got so much into the way of expecting nothing but out-of-the-way things to happen, that it seemed quite dull and stupid for life to go on in the common way.

So she set to work, and very soon finished off the cake.

\begin{center}
\quad*\quad*\quad*\quad*\quad*\quad*\quad*
\par
\quad*\quad*\quad*\quad*\quad*\quad*
\par
\quad*\quad*\quad*\quad*\quad*\quad*\quad*
\end{center}

% second chapter
\section{The Pool of Tears}

‘Curiouser and curiouser!’ cried Alice (she was so much surprised, that for the moment she quite forgot how to speak good English); ‘now I’m opening out like the largest telescope that ever was! Good-bye, feet!’ (for when she looked down at her feet, they seemed to be almost out of sight, they were getting so far off). ‘Oh, my poor little feet, I wonder who will put on your shoes and stockings for you now, dears? I’m sure \textit{I} shan’t be able! I shall be a great deal too far off to trouble myself about you: you must manage the best way you can;—but I must be kind to them,’ thought Alice, ‘or perhaps they won’t walk the way I want to go! Let me see: I’ll give them a new pair of boots every Christmas.’

And she went on planning to herself how she would manage it. ‘They must go by the carrier,’ she thought; ‘and how funny it’ll seem, sending presents to one’s own feet! And how odd the directions will look!

{\setlength{\parskip}{0em}
\par\quad ALICE’S RIGHT FOOT, ESQ.
\par\quad\quad HEARTHRUG,
\par\quad\quad\quad NEAR THE FENDER,
\par\quad\quad\quad\quad (WITH ALICE’S LOVE).
}\\
\par

Oh dear, what nonsense I’m talking!’

Just then her head struck against the roof of the hall: in fact she was now more than nine feet high, and she at once took up the little golden key and hurried off to the garden door.

Poor Alice! It was as much as she could do, lying down on one side, to look through into the garden with one eye; but to get through was more hopeless than ever: she sat down and began to cry again.

‘You ought to be ashamed of yourself,’ said Alice, ‘a great girl like you,’ (she might well say this), ‘to go on crying in this way! Stop this moment, I tell you!’ But she went on all the same, shedding gallons of tears, until there was a large pool all round her, about four inches deep and reaching half down the hall.

After a time she heard a little pattering of feet in the distance, and she hastily dried her eyes to see what was coming. It was the White Rabbit returning, splendidly dressed, with a pair of white kid gloves in one hand and a large fan in the other: he came trotting along in a great hurry, muttering to himself as he came, ‘Oh! the Duchess, the Duchess! Oh! won’t she be savage if I’ve kept her waiting!’ Alice felt so desperate that she was ready to ask help of any one; so, when the Rabbit came near her, she began, in a low, timid voice, ‘If you please, sir—’ The Rabbit started violently, dropped the white kid gloves and the fan, and skurried away into the darkness as hard as he could go.

Alice took up the fan and gloves, and, as the hall was very hot, she kept fanning herself all the time she went on talking: ‘Dear, dear! How queer everything is to-day! And yesterday things went on just as usual. I wonder if I’ve been changed in the night? Let me think: was I the same when I got up this morning? I almost think I can remember feeling a little different. But if I’m not the same, the next question is, Who in the world am I? Ah, THAT’S the great puzzle!’ And she began thinking over all the children she knew that were of the same age as herself, to see if she could have been changed for any of them.

‘I’m sure I’m not Ada,’ she said, ‘for her hair goes in such long ringlets, and mine doesn’t go in ringlets at all; and I’m sure I can’t be Mabel, for I know all sorts of things, and she, oh! she knows such a very little! Besides, SHE’S she, and I’m I, and—oh dear, how puzzling it all is! I’ll try if I know all the things I used to know. Let me see: four times five is twelve, and four times six is thirteen, and four times seven is—oh dear! I shall never get to twenty at that rate! However, the Multiplication Table doesn’t signify: let’s try Geography. London is the capital of Paris, and Paris is the capital of Rome, and Rome—no, THAT’S all wrong, I’m certain! I must have been changed for Mabel! I’ll try and say “How doth the little—“’ and she crossed her hands on her lap as if she were saying lessons, and began to repeat it, but her voice sounded hoarse and strange, and the words did not come the same as they used to do:—

     ‘How doth the little crocodile
      Improve his shining tail,
     And pour the waters of the Nile
      On every golden scale!

     ‘How cheerfully he seems to grin,
      How neatly spread his claws,
     And welcome little fishes in
      With gently smiling jaws!’

‘I’m sure those are not the right words,’ said poor Alice, and her eyes filled with tears again as she went on, ‘I must be Mabel after all, and I shall have to go and live in that poky little house, and have next to no toys to play with, and oh! ever so many lessons to learn! No, I’ve made up my mind about it; if I’m Mabel, I’ll stay down here! It’ll be no use their putting their heads down and saying “Come up again, dear!” I shall only look up and say “Who am I then? Tell me that first, and then, if I like being that person, I’ll come up: if not, I’ll stay down here till I’m somebody else”—but, oh dear!’ cried Alice, with a sudden burst of tears, ‘I do wish they WOULD put their heads down! I am so VERY tired of being all alone here!’

As she said this she looked down at her hands, and was surprised to see that she had put on one of the Rabbit’s little white kid gloves while she was talking. ‘How CAN I have done that?’ she thought. ‘I must be growing small again.’ She got up and went to the table to measure herself by it, and found that, as nearly as she could guess, she was now about two feet high, and was going on shrinking rapidly: she soon found out that the cause of this was the fan she was holding, and she dropped it hastily, just in time to avoid shrinking away altogether.

‘That WAS a narrow escape!’ said Alice, a good deal frightened at the sudden change, but very glad to find herself still in existence; ‘and now for the garden!’ and she ran with all speed back to the little door: but, alas! the little door was shut again, and the little golden key was lying on the glass table as before, ‘and things are worse than ever,’ thought the poor child, ‘for I never was so small as this before, never! And I declare it’s too bad, that it is!’

As she said these words her foot slipped, and in another moment, splash! she was up to her chin in salt water. Her first idea was that she had somehow fallen into the sea, ‘and in that case I can go back by railway,’ she said to herself. (Alice had been to the seaside once in her life, and had come to the general conclusion, that wherever you go to on the English coast you find a number of bathing machines in the sea, some children digging in the sand with wooden spades, then a row of lodging houses, and behind them a railway station.) However, she soon made out that she was in the pool of tears which she had wept when she was nine feet high.

‘I wish I hadn’t cried so much!’ said Alice, as she swam about, trying to find her way out. ‘I shall be punished for it now, I suppose, by being drowned in my own tears! That WILL be a queer thing, to be sure! However, everything is queer to-day.’

Just then she heard something splashing about in the pool a little way off, and she swam nearer to make out what it was: at first she thought it must be a walrus or hippopotamus, but then she remembered how small she was now, and she soon made out that it was only a mouse that had slipped in like herself.

‘Would it be of any use, now,’ thought Alice, ‘to speak to this mouse? Everything is so out-of-the-way down here, that I should think very likely it can talk: at any rate, there’s no harm in trying.’ So she began: ‘O Mouse, do you know the way out of this pool? I am very tired of swimming about here, O Mouse!’ (Alice thought this must be the right way of speaking to a mouse: she had never done such a thing before, but she remembered having seen in her brother’s Latin Grammar, ‘A mouse—of a mouse—to a mouse—a mouse—O mouse!’) The Mouse looked at her rather inquisitively, and seemed to her to wink with one of its little eyes, but it said nothing.

‘Perhaps it doesn’t understand English,’ thought Alice; ‘I daresay it’s a French mouse, come over with William the Conqueror.’ (For, with all her knowledge of history, Alice had no very clear notion how long ago anything had happened.) So she began again: ‘Ou est ma chatte?’ which was the first sentence in her French lesson-book. The Mouse gave a sudden leap out of the water, and seemed to quiver all over with fright. ‘Oh, I beg your pardon!’ cried Alice hastily, afraid that she had hurt the poor animal’s feelings. ‘I quite forgot you didn’t like cats.’

‘Not like cats!’ cried the Mouse, in a shrill, passionate voice. ‘Would YOU like cats if you were me?’

‘Well, perhaps not,’ said Alice in a soothing tone: ‘don’t be angry about it. And yet I wish I could show you our cat Dinah: I think you’d take a fancy to cats if you could only see her. She is such a dear quiet thing,’ Alice went on, half to herself, as she swam lazily about in the pool, ‘and she sits purring so nicely by the fire, licking her paws and washing her face—and she is such a nice soft thing to nurse—and she’s such a capital one for catching mice—oh, I beg your pardon!’ cried Alice again, for this time the Mouse was bristling all over, and she felt certain it must be really offended. ‘We won’t talk about her any more if you’d rather not.’

‘We indeed!’ cried the Mouse, who was trembling down to the end of his tail. ‘As if I would talk on such a subject! Our family always HATED cats: nasty, low, vulgar things! Don’t let me hear the name again!’

‘I won’t indeed!’ said Alice, in a great hurry to change the subject of conversation. ‘Are you—are you fond—of—of dogs?’ The Mouse did not answer, so Alice went on eagerly: ‘There is such a nice little dog near our house I should like to show you! A little bright-eyed terrier, you know, with oh, such long curly brown hair! And it’ll fetch things when you throw them, and it’ll sit up and beg for its dinner, and all sorts of things—I can’t remember half of them—and it belongs to a farmer, you know, and he says it’s so useful, it’s worth a hundred pounds! He says it kills all the rats and—oh dear!’ cried Alice in a sorrowful tone, ‘I’m afraid I’ve offended it again!’ For the Mouse was swimming away from her as hard as it could go, and making quite a commotion in the pool as it went.

So she called softly after it, ‘Mouse dear! Do come back again, and we won’t talk about cats or dogs either, if you don’t like them!’ When the Mouse heard this, it turned round and swam slowly back to her: its face was quite pale (with passion, Alice thought), and it said in a low trembling voice, ‘Let us get to the shore, and then I’ll tell you my history, and you’ll understand why it is I hate cats and dogs.’

It was high time to go, for the pool was getting quite crowded with the birds and animals that had fallen into it: there were a Duck and a Dodo, a Lory and an Eaglet, and several other curious creatures. Alice led the way, and the whole party swam to the shore.

\section{A Caucus-Race and a Long Tale}

They were indeed a queer-looking party that assembled on the bank—the birds with draggled feathers, the animals with their fur clinging close to them, and all dripping wet, cross, and uncomfortable.

The first question of course was, how to get dry again: they had a consultation about this, and after a few minutes it seemed quite natural to Alice to find herself talking familiarly with them, as if she had known them all her life. Indeed, she had quite a long argument with the Lory, who at last turned sulky, and would only say, ‘I am older than you, and must know better’; and this Alice would not allow without knowing how old it was, and, as the Lory positively refused to tell its age, there was no more to be said.

At last the Mouse, who seemed to be a person of authority among them, called out, ‘Sit down, all of you, and listen to me! I’LL soon make you dry enough!’ They all sat down at once, in a large ring, with the Mouse in the middle. Alice kept her eyes anxiously fixed on it, for she felt sure she would catch a bad cold if she did not get dry very soon.

‘Ahem!’ said the Mouse with an important air, ‘are you all ready? This is the driest thing I know. Silence all round, if you please! “William the Conqueror, whose cause was favoured by the pope, was soon submitted to by the English, who wanted leaders, and had been of late much accustomed to usurpation and conquest. Edwin and Morcar, the earls of Mercia and Northumbria—“’

‘Ugh!’ said the Lory, with a shiver.

‘I beg your pardon!’ said the Mouse, frowning, but very politely: ‘Did you speak?’

‘Not I!’ said the Lory hastily.

‘I thought you did,’ said the Mouse. ‘—I proceed. “Edwin and Morcar, the earls of Mercia and Northumbria, declared for him: and even Stigand, the patriotic archbishop of Canterbury, found it advisable—“’

‘Found WHAT?’ said the Duck.

‘Found IT,’ the Mouse replied rather crossly: ‘of course you know what “it” means.’

‘I know what “it” means well enough, when I find a thing,’ said the Duck: ‘it’s generally a frog or a worm. The question is, what did the archbishop find?’

The Mouse did not notice this question, but hurriedly went on, ‘”—found it advisable to go with Edgar Atheling to meet William and offer him the crown. William’s conduct at first was moderate. But the insolence of his Normans—“ How are you getting on now, my dear?’ it continued, turning to Alice as it spoke.

‘As wet as ever,’ said Alice in a melancholy tone: ‘it doesn’t seem to dry me at all.’

‘In that case,’ said the Dodo solemnly, rising to its feet, ‘I move that the meeting adjourn, for the immediate adoption of more energetic remedies—’

‘Speak English!’ said the Eaglet. ‘I don’t know the meaning of half those long words, and, what’s more, I don’t believe you do either!’ And the Eaglet bent down its head to hide a smile: some of the other birds tittered audibly.

‘What I was going to say,’ said the Dodo in an offended tone, ‘was, that the best thing to get us dry would be a Caucus-race.’

‘What IS a Caucus-race?’ said Alice; not that she wanted much to know, but the Dodo had paused as if it thought that SOMEBODY ought to speak, and no one else seemed inclined to say anything.

‘Why,’ said the Dodo, ‘the best way to explain it is to do it.’ (And, as you might like to try the thing yourself, some winter day, I will tell you how the Dodo managed it.)

First it marked out a race-course, in a sort of circle, (‘the exact shape doesn’t matter,’ it said,) and then all the party were placed along the course, here and there. There was no ‘One, two, three, and away,’ but they began running when they liked, and left off when they liked, so that it was not easy to know when the race was over. However, when they had been running half an hour or so, and were quite dry again, the Dodo suddenly called out ‘The race is over!’ and they all crowded round it, panting, and asking, ‘But who has won?’

This question the Dodo could not answer without a great deal of thought, and it sat for a long time with one finger pressed upon its forehead (the position in which you usually see Shakespeare, in the pictures of him), while the rest waited in silence. At last the Dodo said, ‘EVERYBODY has won, and all must have prizes.’

‘But who is to give the prizes?’ quite a chorus of voices asked.

‘Why, SHE, of course,’ said the Dodo, pointing to Alice with one finger; and the whole party at once crowded round her, calling out in a confused way, ‘Prizes! Prizes!’

Alice had no idea what to do, and in despair she put her hand in her pocket, and pulled out a box of comfits, (luckily the salt water had not got into it), and handed them round as prizes. There was exactly one a-piece all round.

‘But she must have a prize herself, you know,’ said the Mouse.

‘Of course,’ the Dodo replied very gravely. ‘What else have you got in your pocket?’ he went on, turning to Alice.

‘Only a thimble,’ said Alice sadly.

‘Hand it over here,’ said the Dodo.

Then they all crowded round her once more, while the Dodo solemnly presented the thimble, saying ‘We beg your acceptance of this elegant thimble’; and, when it had finished this short speech, they all cheered.

Alice thought the whole thing very absurd, but they all looked so grave that she did not dare to laugh; and, as she could not think of anything to say, she simply bowed, and took the thimble, looking as solemn as she could.

The next thing was to eat the comfits: this caused some noise and confusion, as the large birds complained that they could not taste theirs, and the small ones choked and had to be patted on the back. However, it was over at last, and they sat down again in a ring, and begged the Mouse to tell them something more.

‘You promised to tell me your history, you know,’ said Alice, ‘and why it is you hate—C and D,’ she added in a whisper, half afraid that it would be offended again.

‘Mine is a long and a sad tale!’ said the Mouse, turning to Alice, and sighing.

‘It IS a long tail, certainly,’ said Alice, looking down with wonder at the Mouse’s tail; ‘but why do you call it sad?’ And she kept on puzzling about it while the Mouse was speaking, so that her idea of the tale was something like this:—

         ‘Fury said to a
         mouse, That he
        met in the
       house,
     “Let us
      both go to
       law: I will
        prosecute
         YOU.—Come,
           I’ll take no
           denial; We
          must have a
        trial: For
      really this
     morning I’ve
    nothing
    to do.”
     Said the
      mouse to the
       cur, “Such
        a trial,
         dear Sir,
            With
          no jury
        or judge,
       would be
      wasting
      our
      breath.”
       “I’ll be
        judge, I’ll
         be jury,”
            Said
         cunning
          old Fury:
          “I’ll
          try the
            whole
            cause,
              and
           condemn
           you
          to
           death.”’


‘You are not attending!’ said the Mouse to Alice severely. ‘What are you thinking of?’

‘I beg your pardon,’ said Alice very humbly: ‘you had got to the fifth bend, I think?’

‘I had NOT!’ cried the Mouse, sharply and very angrily.

‘A knot!’ said Alice, always ready to make herself useful, and looking anxiously about her. ‘Oh, do let me help to undo it!’

‘I shall do nothing of the sort,’ said the Mouse, getting up and walking away. ‘You insult me by talking such nonsense!’

‘I didn’t mean it!’ pleaded poor Alice. ‘But you’re so easily offended, you know!’

The Mouse only growled in reply.

‘Please come back and finish your story!’ Alice called after it; and the others all joined in chorus, ‘Yes, please do!’ but the Mouse only shook its head impatiently, and walked a little quicker.

‘What a pity it wouldn’t stay!’ sighed the Lory, as soon as it was quite out of sight; and an old Crab took the opportunity of saying to her daughter ‘Ah, my dear! Let this be a lesson to you never to lose YOUR temper!’ ‘Hold your tongue, Ma!’ said the young Crab, a little snappishly. ‘You’re enough to try the patience of an oyster!’

‘I wish I had our Dinah here, I know I do!’ said Alice aloud, addressing nobody in particular. ‘She’d soon fetch it back!’

‘And who is Dinah, if I might venture to ask the question?’ said the Lory.

Alice replied eagerly, for she was always ready to talk about her pet: ‘Dinah’s our cat. And she’s such a capital one for catching mice you can’t think! And oh, I wish you could see her after the birds! Why, she’ll eat a little bird as soon as look at it!’

This speech caused a remarkable sensation among the party. Some of the birds hurried off at once: one old Magpie began wrapping itself up very carefully, remarking, ‘I really must be getting home; the night-air doesn’t suit my throat!’ and a Canary called out in a trembling voice to its children, ‘Come away, my dears! It’s high time you were all in bed!’ On various pretexts they all moved off, and Alice was soon left alone.

‘I wish I hadn’t mentioned Dinah!’ she said to herself in a melancholy tone. ‘Nobody seems to like her, down here, and I’m sure she’s the best cat in the world! Oh, my dear Dinah! I wonder if I shall ever see you any more!’ And here poor Alice began to cry again, for she felt very lonely and low-spirited. In a little while, however, she again heard a little pattering of footsteps in the distance, and she looked up eagerly, half hoping that the Mouse had changed his mind, and was coming back to finish his story.




\section{The Rabbit Sends in a Little Bill}

It was the White Rabbit, trotting slowly back again, and looking anxiously about as it went, as if it had lost something; and she heard it muttering to itself ‘The Duchess! The Duchess! Oh my dear paws! Oh my fur and whiskers! She’ll get me executed, as sure as ferrets are ferrets! Where CAN I have dropped them, I wonder?’ Alice guessed in a moment that it was looking for the fan and the pair of white kid gloves, and she very good-naturedly began hunting about for them, but they were nowhere to be seen—everything seemed to have changed since her swim in the pool, and the great hall, with the glass table and the little door, had vanished completely.

Very soon the Rabbit noticed Alice, as she went hunting about, and called out to her in an angry tone, ‘Why, Mary Ann, what ARE you doing out here? Run home this moment, and fetch me a pair of gloves and a fan! Quick, now!’ And Alice was so much frightened that she ran off at once in the direction it pointed to, without trying to explain the mistake it had made.

‘He took me for his housemaid,’ she said to herself as she ran. ‘How surprised he’ll be when he finds out who I am! But I’d better take him his fan and gloves—that is, if I can find them.’ As she said this, she came upon a neat little house, on the door of which was a bright brass plate with the name ‘W. RABBIT’ engraved upon it. She went in without knocking, and hurried upstairs, in great fear lest she should meet the real Mary Ann, and be turned out of the house before she had found the fan and gloves.

‘How queer it seems,’ Alice said to herself, ‘to be going messages for a rabbit! I suppose Dinah’ll be sending me on messages next!’ And she began fancying the sort of thing that would happen: ‘“Miss Alice! Come here directly, and get ready for your walk!” “Coming in a minute, nurse! But I’ve got to see that the mouse doesn’t get out.” Only I don’t think,’ Alice went on, ‘that they’d let Dinah stop in the house if it began ordering people about like that!’

By this time she had found her way into a tidy little room with a table in the window, and on it (as she had hoped) a fan and two or three pairs of tiny white kid gloves: she took up the fan and a pair of the gloves, and was just going to leave the room, when her eye fell upon a little bottle that stood near the looking-glass. There was no label this time with the words ‘DRINK ME,’ but nevertheless she uncorked it and put it to her lips. ‘I know SOMETHING interesting is sure to happen,’ she said to herself, ‘whenever I eat or drink anything; so I’ll just see what this bottle does. I do hope it’ll make me grow large again, for really I’m quite tired of being such a tiny little thing!’

It did so indeed, and much sooner than she had expected: before she had drunk half the bottle, she found her head pressing against the ceiling, and had to stoop to save her neck from being broken. She hastily put down the bottle, saying to herself ‘That’s quite enough—I hope I shan’t grow any more—As it is, I can’t get out at the door—I do wish I hadn’t drunk quite so much!’

Alas! it was too late to wish that! She went on growing, and growing, and very soon had to kneel down on the floor: in another minute there was not even room for this, and she tried the effect of lying down with one elbow against the door, and the other arm curled round her head. Still she went on growing, and, as a last resource, she put one arm out of the window, and one foot up the chimney, and said to herself ‘Now I can do no more, whatever happens. What WILL become of me?’

Luckily for Alice, the little magic bottle had now had its full effect, and she grew no larger: still it was very uncomfortable, and, as there seemed to be no sort of chance of her ever getting out of the room again, no wonder she felt unhappy.

‘It was much pleasanter at home,’ thought poor Alice, ‘when one wasn’t always growing larger and smaller, and being ordered about by mice and rabbits. I almost wish I hadn’t gone down that rabbit-hole—and yet—and yet—it’s rather curious, you know, this sort of life! I do wonder what CAN have happened to me! When I used to read fairy-tales, I fancied that kind of thing never happened, and now here I am in the middle of one! There ought to be a book written about me, that there ought! And when I grow up, I’ll write one—but I’m grown up now,’ she added in a sorrowful tone; ‘at least there’s no room to grow up any more HERE.’

‘But then,’ thought Alice, ‘shall I NEVER get any older than I am now? That’ll be a comfort, one way—never to be an old woman—but then—always to have lessons to learn! Oh, I shouldn’t like THAT!’

‘Oh, you foolish Alice!’ she answered herself. ‘How can you learn lessons in here? Why, there’s hardly room for YOU, and no room at all for any lesson-books!’

And so she went on, taking first one side and then the other, and making quite a conversation of it altogether; but after a few minutes she heard a voice outside, and stopped to listen.

‘Mary Ann! Mary Ann!’ said the voice. ‘Fetch me my gloves this moment!’ Then came a little pattering of feet on the stairs. Alice knew it was the Rabbit coming to look for her, and she trembled till she shook the house, quite forgetting that she was now about a thousand times as large as the Rabbit, and had no reason to be afraid of it.

Presently the Rabbit came up to the door, and tried to open it; but, as the door opened inwards, and Alice’s elbow was pressed hard against it, that attempt proved a failure. Alice heard it say to itself ‘Then I’ll go round and get in at the window.’

‘THAT you won’t’ thought Alice, and, after waiting till she fancied she heard the Rabbit just under the window, she suddenly spread out her hand, and made a snatch in the air. She did not get hold of anything, but she heard a little shriek and a fall, and a crash of broken glass, from which she concluded that it was just possible it had fallen into a cucumber-frame, or something of the sort.

Next came an angry voice—the Rabbit’s—‘Pat! Pat! Where are you?’ And then a voice she had never heard before, ‘Sure then I’m here! Digging for apples, yer honour!’

‘Digging for apples, indeed!’ said the Rabbit angrily. ‘Here! Come and help me out of THIS!’ (Sounds of more broken glass.)

‘Now tell me, Pat, what’s that in the window?’

‘Sure, it’s an arm, yer honour!’ (He pronounced it ‘arrum.’)

‘An arm, you goose! Who ever saw one that size? Why, it fills the whole window!’

‘Sure, it does, yer honour: but it’s an arm for all that.’

‘Well, it’s got no business there, at any rate: go and take it away!’

There was a long silence after this, and Alice could only hear whispers now and then; such as, ‘Sure, I don’t like it, yer honour, at all, at all!’ ‘Do as I tell you, you coward!’ and at last she spread out her hand again, and made another snatch in the air. This time there were TWO little shrieks, and more sounds of broken glass. ‘What a number of cucumber-frames there must be!’ thought Alice. ‘I wonder what they’ll do next! As for pulling me out of the window, I only wish they COULD! I’m sure I don’t want to stay in here any longer!’

She waited for some time without hearing anything more: at last came a rumbling of little cartwheels, and the sound of a good many voices all talking together: she made out the words: ‘Where’s the other ladder?—Why, I hadn’t to bring but one; Bill’s got the other—Bill! fetch it here, lad!—Here, put ‘em up at this corner—No, tie ‘em together first—they don’t reach half high enough yet—Oh! they’ll do well enough; don’t be particular—Here, Bill! catch hold of this rope—Will the roof bear?—Mind that loose slate—Oh, it’s coming down! Heads below!’ (a loud crash)—‘Now, who did that?—It was Bill, I fancy—Who’s to go down the chimney?—Nay, I shan’t! YOU do it!—That I won’t, then!—Bill’s to go down—Here, Bill! the master says you’re to go down the chimney!’

‘Oh! So Bill’s got to come down the chimney, has he?’ said Alice to herself. ‘Shy, they seem to put everything upon Bill! I wouldn’t be in Bill’s place for a good deal: this fireplace is narrow, to be sure; but I THINK I can kick a little!’

She drew her foot as far down the chimney as she could, and waited till she heard a little animal (she couldn’t guess of what sort it was) scratching and scrambling about in the chimney close above her: then, saying to herself ‘This is Bill,’ she gave one sharp kick, and waited to see what would happen next.

The first thing she heard was a general chorus of ‘There goes Bill!’ then the Rabbit’s voice along—‘Catch him, you by the hedge!’ then silence, and then another confusion of voices—‘Hold up his head—Brandy now—Don’t choke him—How was it, old fellow? What happened to you? Tell us all about it!’

Last came a little feeble, squeaking voice, (‘That’s Bill,’ thought Alice,) ‘Well, I hardly know—No more, thank ye; I’m better now—but I’m a deal too flustered to tell you—all I know is, something comes at me like a Jack-in-the-box, and up I goes like a sky-rocket!’

‘So you did, old fellow!’ said the others.

‘We must burn the house down!’ said the Rabbit’s voice; and Alice called out as loud as she could, ‘If you do. I’ll set Dinah at you!’

There was a dead silence instantly, and Alice thought to herself, ‘I wonder what they WILL do next! If they had any sense, they’d take the roof off.’ After a minute or two, they began moving about again, and Alice heard the Rabbit say, ‘A barrowful will do, to begin with.’

‘A barrowful of WHAT?’ thought Alice; but she had not long to doubt, for the next moment a shower of little pebbles came rattling in at the window, and some of them hit her in the face. ‘I’ll put a stop to this,’ she said to herself, and shouted out, ‘You’d better not do that again!’ which produced another dead silence.

Alice noticed with some surprise that the pebbles were all turning into little cakes as they lay on the floor, and a bright idea came into her head. ‘If I eat one of these cakes,’ she thought, ‘it’s sure to make SOME change in my size; and as it can’t possibly make me larger, it must make me smaller, I suppose.’

So she swallowed one of the cakes, and was delighted to find that she began shrinking directly. As soon as she was small enough to get through the door, she ran out of the house, and found quite a crowd of little animals and birds waiting outside. The poor little Lizard, Bill, was in the middle, being held up by two guinea-pigs, who were giving it something out of a bottle. They all made a rush at Alice the moment she appeared; but she ran off as hard as she could, and soon found herself safe in a thick wood.

‘The first thing I’ve got to do,’ said Alice to herself, as she wandered about in the wood, ‘is to grow to my right size again; and the second thing is to find my way into that lovely garden. I think that will be the best plan.’

It sounded an excellent plan, no doubt, and very neatly and simply arranged; the only difficulty was, that she had not the smallest idea how to set about it; and while she was peering about anxiously among the trees, a little sharp bark just over her head made her look up in a great hurry.

An enormous puppy was looking down at her with large round eyes, and feebly stretching out one paw, trying to touch her. ‘Poor little thing!’ said Alice, in a coaxing tone, and she tried hard to whistle to it; but she was terribly frightened all the time at the thought that it might be hungry, in which case it would be very likely to eat her up in spite of all her coaxing.

Hardly knowing what she did, she picked up a little bit of stick, and held it out to the puppy; whereupon the puppy jumped into the air off all its feet at once, with a yelp of delight, and rushed at the stick, and made believe to worry it; then Alice dodged behind a great thistle, to keep herself from being run over; and the moment she appeared on the other side, the puppy made another rush at the stick, and tumbled head over heels in its hurry to get hold of it; then Alice, thinking it was very like having a game of play with a cart-horse, and expecting every moment to be trampled under its feet, ran round the thistle again; then the puppy began a series of short charges at the stick, running a very little way forwards each time and a long way back, and barking hoarsely all the while, till at last it sat down a good way off, panting, with its tongue hanging out of its mouth, and its great eyes half shut.

This seemed to Alice a good opportunity for making her escape; so she set off at once, and ran till she was quite tired and out of breath, and till the puppy’s bark sounded quite faint in the distance.

‘And yet what a dear little puppy it was!’ said Alice, as she leant against a buttercup to rest herself, and fanned herself with one of the leaves: ‘I should have liked teaching it tricks very much, if—if I’d only been the right size to do it! Oh dear! I’d nearly forgotten that I’ve got to grow up again! Let me see—how IS it to be managed? I suppose I ought to eat or drink something or other; but the great question is, what?’

The great question certainly was, what? Alice looked all round her at the flowers and the blades of grass, but she did not see anything that looked like the right thing to eat or drink under the circumstances. There was a large mushroom growing near her, about the same height as herself; and when she had looked under it, and on both sides of it, and behind it, it occurred to her that she might as well look and see what was on the top of it.

She stretched herself up on tiptoe, and peeped over the edge of the mushroom, and her eyes immediately met those of a large caterpillar, that was sitting on the top with its arms folded, quietly smoking a long hookah, and taking not the smallest notice of her or of anything else.




\section{Advice from a Caterpillar}

The Caterpillar and Alice looked at each other for some time in silence: at last the Caterpillar took the hookah out of its mouth, and addressed her in a languid, sleepy voice.

‘Who are YOU?’ said the Caterpillar.

This was not an encouraging opening for a conversation. Alice replied, rather shyly, ‘I—I hardly know, sir, just at present—at least I know who I WAS when I got up this morning, but I think I must have been changed several times since then.’

‘What do you mean by that?’ said the Caterpillar sternly. ‘Explain yourself!’

‘I can’t explain MYSELF, I’m afraid, sir’ said Alice, ‘because I’m not myself, you see.’

‘I don’t see,’ said the Caterpillar.

‘I’m afraid I can’t put it more clearly,’ Alice replied very politely, ‘for I can’t understand it myself to begin with; and being so many different sizes in a day is very confusing.’

‘It isn’t,’ said the Caterpillar.

‘Well, perhaps you haven’t found it so yet,’ said Alice; ‘but when you have to turn into a chrysalis—you will some day, you know—and then after that into a butterfly, I should think you’ll feel it a little queer, won’t you?’

‘Not a bit,’ said the Caterpillar.

‘Well, perhaps your feelings may be different,’ said Alice; ‘all I know is, it would feel very queer to ME.’

‘You!’ said the Caterpillar contemptuously. ‘Who are YOU?’

Which brought them back again to the beginning of the conversation. Alice felt a little irritated at the Caterpillar’s making such VERY short remarks, and she drew herself up and said, very gravely, ‘I think, you ought to tell me who YOU are, first.’

‘Why?’ said the Caterpillar.

Here was another puzzling question; and as Alice could not think of any good reason, and as the Caterpillar seemed to be in a VERY unpleasant state of mind, she turned away.

‘Come back!’ the Caterpillar called after her. ‘I’ve something important to say!’

This sounded promising, certainly: Alice turned and came back again.

‘Keep your temper,’ said the Caterpillar.

‘Is that all?’ said Alice, swallowing down her anger as well as she could.

‘No,’ said the Caterpillar.

Alice thought she might as well wait, as she had nothing else to do, and perhaps after all it might tell her something worth hearing. For some minutes it puffed away without speaking, but at last it unfolded its arms, took the hookah out of its mouth again, and said, ‘So you think you’re changed, do you?’

‘I’m afraid I am, sir,’ said Alice; ‘I can’t remember things as I used—and I don’t keep the same size for ten minutes together!’

‘Can’t remember WHAT things?’ said the Caterpillar.

‘Well, I’ve tried to say “HOW DOTH THE LITTLE BUSY BEE,” but it all came different!’ Alice replied in a very melancholy voice.

‘Repeat, “YOU ARE OLD, FATHER WILLIAM,”’ said the Caterpillar.

Alice folded her hands, and began:—

   ‘You are old, Father William,’ the young man said,
    ‘And your hair has become very white;
   And yet you incessantly stand on your head—
    Do you think, at your age, it is right?’

   ‘In my youth,’ Father William replied to his son,
    ‘I feared it might injure the brain;
   But, now that I’m perfectly sure I have none,
    Why, I do it again and again.’

   ‘You are old,’ said the youth, ‘as I mentioned before,
    And have grown most uncommonly fat;
   Yet you turned a back-somersault in at the door—
    Pray, what is the reason of that?’

   ‘In my youth,’ said the sage, as he shook his grey locks,
    ‘I kept all my limbs very supple
   By the use of this ointment—one shilling the box—
    Allow me to sell you a couple?’

   ‘You are old,’ said the youth, ‘and your jaws are too weak
    For anything tougher than suet;
   Yet you finished the goose, with the bones and the beak—
    Pray how did you manage to do it?’

   ‘In my youth,’ said his father, ‘I took to the law,
    And argued each case with my wife;
   And the muscular strength, which it gave to my jaw,
    Has lasted the rest of my life.’

   ‘You are old,’ said the youth, ‘one would hardly suppose
    That your eye was as steady as ever;
   Yet you balanced an eel on the end of your nose—
    What made you so awfully clever?’

   ‘I have answered three questions, and that is enough,’
    Said his father; ‘don’t give yourself airs!
   Do you think I can listen all day to such stuff?
    Be off, or I’ll kick you down stairs!’


‘That is not said right,’ said the Caterpillar.

‘Not QUITE right, I’m afraid,’ said Alice, timidly; ‘some of the words have got altered.’

‘It is wrong from beginning to end,’ said the Caterpillar decidedly, and there was silence for some minutes.

The Caterpillar was the first to speak.

‘What size do you want to be?’ it asked.

‘Oh, I’m not particular as to size,’ Alice hastily replied; ‘only one doesn’t like changing so often, you know.’

‘I DON’T know,’ said the Caterpillar.

Alice said nothing: she had never been so much contradicted in her life before, and she felt that she was losing her temper.

‘Are you content now?’ said the Caterpillar.

‘Well, I should like to be a LITTLE larger, sir, if you wouldn’t mind,’ said Alice: ‘three inches is such a wretched height to be.’

‘It is a very good height indeed!’ said the Caterpillar angrily, rearing itself upright as it spoke (it was exactly three inches high).

‘But I’m not used to it!’ pleaded poor Alice in a piteous tone. And she thought of herself, ‘I wish the creatures wouldn’t be so easily offended!’

‘You’ll get used to it in time,’ said the Caterpillar; and it put the hookah into its mouth and began smoking again.

This time Alice waited patiently until it chose to speak again. In a minute or two the Caterpillar took the hookah out of its mouth and yawned once or twice, and shook itself. Then it got down off the mushroom, and crawled away in the grass, merely remarking as it went, ‘One side will make you grow taller, and the other side will make you grow shorter.’

‘One side of WHAT? The other side of WHAT?’ thought Alice to herself.

‘Of the mushroom,’ said the Caterpillar, just as if she had asked it aloud; and in another moment it was out of sight.

Alice remained looking thoughtfully at the mushroom for a minute, trying to make out which were the two sides of it; and as it was perfectly round, she found this a very difficult question. However, at last she stretched her arms round it as far as they would go, and broke off a bit of the edge with each hand.

‘And now which is which?’ she said to herself, and nibbled a little of the right-hand bit to try the effect: the next moment she felt a violent blow underneath her chin: it had struck her foot!

She was a good deal frightened by this very sudden change, but she felt that there was no time to be lost, as she was shrinking rapidly; so she set to work at once to eat some of the other bit. Her chin was pressed so closely against her foot, that there was hardly room to open her mouth; but she did it at last, and managed to swallow a morsel of the lefthand bit.


\begin{center}
\quad*\quad*\quad*\quad*\quad*\quad*\quad*
\par
\quad*\quad*\quad*\quad*\quad*\quad*
\par
\quad*\quad*\quad*\quad*\quad*\quad*\quad*
\end{center}

‘Come, my head’s free at last!’ said Alice in a tone of delight, which changed into alarm in another moment, when she found that her shoulders were nowhere to be found: all she could see, when she looked down, was an immense length of neck, which seemed to rise like a stalk out of a sea of green leaves that lay far below her.

‘What CAN all that green stuff be?’ said Alice. ‘And where HAVE my shoulders got to? And oh, my poor hands, how is it I can’t see you?’ She was moving them about as she spoke, but no result seemed to follow, except a little shaking among the distant green leaves.

As there seemed to be no chance of getting her hands up to her head, she tried to get her head down to them, and was delighted to find that her neck would bend about easily in any direction, like a serpent. She had just succeeded in curving it down into a graceful zigzag, and was going to dive in among the leaves, which she found to be nothing but the tops of the trees under which she had been wandering, when a sharp hiss made her draw back in a hurry: a large pigeon had flown into her face, and was beating her violently with its wings.

‘Serpent!’ screamed the Pigeon.

‘I’m NOT a serpent!’ said Alice indignantly. ‘Let me alone!’

‘Serpent, I say again!’ repeated the Pigeon, but in a more subdued tone, and added with a kind of sob, ‘I’ve tried every way, and nothing seems to suit them!’

‘I haven’t the least idea what you’re talking about,’ said Alice.

‘I’ve tried the roots of trees, and I’ve tried banks, and I’ve tried hedges,’ the Pigeon went on, without attending to her; ‘but those serpents! There’s no pleasing them!’

Alice was more and more puzzled, but she thought there was no use in saying anything more till the Pigeon had finished.

‘As if it wasn’t trouble enough hatching the eggs,’ said the Pigeon; ‘but I must be on the look-out for serpents night and day! Why, I haven’t had a wink of sleep these three weeks!’

‘I’m very sorry you’ve been annoyed,’ said Alice, who was beginning to see its meaning.

‘And just as I’d taken the highest tree in the wood,’ continued the Pigeon, raising its voice to a shriek, ‘and just as I was thinking I should be free of them at last, they must needs come wriggling down from the sky! Ugh, Serpent!’

‘But I’m NOT a serpent, I tell you!’ said Alice. ‘I’m a—I’m a—’

‘Well! WHAT are you?’ said the Pigeon. ‘I can see you’re trying to invent something!’

‘I—I’m a little girl,’ said Alice, rather doubtfully, as she remembered the number of changes she had gone through that day.

‘A likely story indeed!’ said the Pigeon in a tone of the deepest contempt. ‘I’ve seen a good many little girls in my time, but never ONE with such a neck as that! No, no! You’re a serpent; and there’s no use denying it. I suppose you’ll be telling me next that you never tasted an egg!’

‘I HAVE tasted eggs, certainly,’ said Alice, who was a very truthful child; ‘but little girls eat eggs quite as much as serpents do, you know.’

‘I don’t believe it,’ said the Pigeon; ‘but if they do, why then they’re a kind of serpent, that’s all I can say.’

This was such a new idea to Alice, that she was quite silent for a minute or two, which gave the Pigeon the opportunity of adding, ‘You’re looking for eggs, I know THAT well enough; and what does it matter to me whether you’re a little girl or a serpent?’

‘It matters a good deal to ME,’ said Alice hastily; ‘but I’m not looking for eggs, as it happens; and if I was, I shouldn’t want YOURS: I don’t like them raw.’

‘Well, be off, then!’ said the Pigeon in a sulky tone, as it settled down again into its nest. Alice crouched down among the trees as well as she could, for her neck kept getting entangled among the branches, and every now and then she had to stop and untwist it. After a while she remembered that she still held the pieces of mushroom in her hands, and she set to work very carefully, nibbling first at one and then at the other, and growing sometimes taller and sometimes shorter, until she had succeeded in bringing herself down to her usual height.

It was so long since she had been anything near the right size, that it felt quite strange at first; but she got used to it in a few minutes, and began talking to herself, as usual. ‘Come, there’s half my plan done now! How puzzling all these changes are! I’m never sure what I’m going to be, from one minute to another! However, I’ve got back to my right size: the next thing is, to get into that beautiful garden—how IS that to be done, I wonder?’ As she said this, she came suddenly upon an open place, with a little house in it about four feet high. ‘Whoever lives there,’ thought Alice, ‘it’ll never do to come upon them THIS size: why, I should frighten them out of their wits!’ So she began nibbling at the righthand bit again, and did not venture to go near the house till she had brought herself down to nine inches high.




\section{Pig and Pepper}

For a minute or two she stood looking at the house, and wondering what to do next, when suddenly a footman in livery came running out of the wood—(she considered him to be a footman because he was in livery: otherwise, judging by his face only, she would have called him a fish)—and rapped loudly at the door with his knuckles. It was opened by another footman in livery, with a round face, and large eyes like a frog; and both footmen, Alice noticed, had powdered hair that curled all over their heads. She felt very curious to know what it was all about, and crept a little way out of the wood to listen.

The Fish-Footman began by producing from under his arm a great letter, nearly as large as himself, and this he handed over to the other, saying, in a solemn tone, ‘For the Duchess. An invitation from the Queen to play croquet.’ The Frog-Footman repeated, in the same solemn tone, only changing the order of the words a little, ‘From the Queen. An invitation for the Duchess to play croquet.’

Then they both bowed low, and their curls got entangled together.

Alice laughed so much at this, that she had to run back into the wood for fear of their hearing her; and when she next peeped out the Fish-Footman was gone, and the other was sitting on the ground near the door, staring stupidly up into the sky.

Alice went timidly up to the door, and knocked.

‘There’s no sort of use in knocking,’ said the Footman, ‘and that for two reasons. First, because I’m on the same side of the door as you are; secondly, because they’re making such a noise inside, no one could possibly hear you.’ And certainly there was a most extraordinary noise going on within—a constant howling and sneezing, and every now and then a great crash, as if a dish or kettle had been broken to pieces.

‘Please, then,’ said Alice, ‘how am I to get in?’

‘There might be some sense in your knocking,’ the Footman went on without attending to her, ‘if we had the door between us. For instance, if you were INSIDE, you might knock, and I could let you out, you know.’ He was looking up into the sky all the time he was speaking, and this Alice thought decidedly uncivil. ‘But perhaps he can’t help it,’ she said to herself; ‘his eyes are so VERY nearly at the top of his head. But at any rate he might answer questions.—How am I to get in?’ she repeated, aloud.

‘I shall sit here,’ the Footman remarked, ‘till tomorrow—’

At this moment the door of the house opened, and a large plate came skimming out, straight at the Footman’s head: it just grazed his nose, and broke to pieces against one of the trees behind him.

‘—or next day, maybe,’ the Footman continued in the same tone, exactly as if nothing had happened.

‘How am I to get in?’ asked Alice again, in a louder tone.

‘ARE you to get in at all?’ said the Footman. ‘That’s the first question, you know.’

It was, no doubt: only Alice did not like to be told so. ‘It’s really dreadful,’ she muttered to herself, ‘the way all the creatures argue. It’s enough to drive one crazy!’

The Footman seemed to think this a good opportunity for repeating his remark, with variations. ‘I shall sit here,’ he said, ‘on and off, for days and days.’

‘But what am I to do?’ said Alice.

‘Anything you like,’ said the Footman, and began whistling.

‘Oh, there’s no use in talking to him,’ said Alice desperately: ‘he’s perfectly idiotic!’ And she opened the door and went in.

The door led right into a large kitchen, which was full of smoke from one end to the other: the Duchess was sitting on a three-legged stool in the middle, nursing a baby; the cook was leaning over the fire, stirring a large cauldron which seemed to be full of soup.

‘There’s certainly too much pepper in that soup!’ Alice said to herself, as well as she could for sneezing.

There was certainly too much of it in the air. Even the Duchess sneezed occasionally; and as for the baby, it was sneezing and howling alternately without a moment’s pause. The only things in the kitchen that did not sneeze, were the cook, and a large cat which was sitting on the hearth and grinning from ear to ear.

‘Please would you tell me,’ said Alice, a little timidly, for she was not quite sure whether it was good manners for her to speak first, ‘why your cat grins like that?’

‘It’s a Cheshire cat,’ said the Duchess, ‘and that’s why. Pig!’

She said the last word with such sudden violence that Alice quite jumped; but she saw in another moment that it was addressed to the baby, and not to her, so she took courage, and went on again:—

‘I didn’t know that Cheshire cats always grinned; in fact, I didn’t know that cats COULD grin.’

‘They all can,’ said the Duchess; ‘and most of ‘em do.’

‘I don’t know of any that do,’ Alice said very politely, feeling quite pleased to have got into a conversation.

‘You don’t know much,’ said the Duchess; ‘and that’s a fact.’

Alice did not at all like the tone of this remark, and thought it would be as well to introduce some other subject of conversation. While she was trying to fix on one, the cook took the cauldron of soup off the fire, and at once set to work throwing everything within her reach at the Duchess and the baby—the fire-irons came first; then followed a shower of saucepans, plates, and dishes. The Duchess took no notice of them even when they hit her; and the baby was howling so much already, that it was quite impossible to say whether the blows hurt it or not.

‘Oh, PLEASE mind what you’re doing!’ cried Alice, jumping up and down in an agony of terror. ‘Oh, there goes his PRECIOUS nose’; as an unusually large saucepan flew close by it, and very nearly carried it off.

‘If everybody minded their own business,’ the Duchess said in a hoarse growl, ‘the world would go round a deal faster than it does.’

‘Which would NOT be an advantage,’ said Alice, who felt very glad to get an opportunity of showing off a little of her knowledge. ‘Just think of what work it would make with the day and night! You see the earth takes twenty-four hours to turn round on its axis—’

‘Talking of axes,’ said the Duchess, ‘chop off her head!’

Alice glanced rather anxiously at the cook, to see if she meant to take the hint; but the cook was busily stirring the soup, and seemed not to be listening, so she went on again: ‘Twenty-four hours, I THINK; or is it twelve? I—’

‘Oh, don’t bother ME,’ said the Duchess; ‘I never could abide figures!’ And with that she began nursing her child again, singing a sort of lullaby to it as she did so, and giving it a violent shake at the end of every line:

   ‘Speak roughly to your little boy,
    And beat him when he sneezes:
   He only does it to annoy,
    Because he knows it teases.’

         CHORUS.

 (In which the cook and the baby joined):—

       ‘Wow! wow! wow!’

While the Duchess sang the second verse of the song, she kept tossing the baby violently up and down, and the poor little thing howled so, that Alice could hardly hear the words:—

   ‘I speak severely to my boy,
    I beat him when he sneezes;
   For he can thoroughly enjoy
    The pepper when he pleases!’

         CHORUS.

       ‘Wow! wow! wow!’

‘Here! you may nurse it a bit, if you like!’ the Duchess said to Alice, flinging the baby at her as she spoke. ‘I must go and get ready to play croquet with the Queen,’ and she hurried out of the room. The cook threw a frying-pan after her as she went out, but it just missed her.

Alice caught the baby with some difficulty, as it was a queer-shaped little creature, and held out its arms and legs in all directions, ‘just like a star-fish,’ thought Alice. The poor little thing was snorting like a steam-engine when she caught it, and kept doubling itself up and straightening itself out again, so that altogether, for the first minute or two, it was as much as she could do to hold it.

As soon as she had made out the proper way of nursing it, (which was to twist it up into a sort of knot, and then keep tight hold of its right ear and left foot, so as to prevent its undoing itself,) she carried it out into the open air. ‘IF I don’t take this child away with me,’ thought Alice, ‘they’re sure to kill it in a day or two: wouldn’t it be murder to leave it behind?’ She said the last words out loud, and the little thing grunted in reply (it had left off sneezing by this time). ‘Don’t grunt,’ said Alice; ‘that’s not at all a proper way of expressing yourself.’

The baby grunted again, and Alice looked very anxiously into its face to see what was the matter with it. There could be no doubt that it had a VERY turn-up nose, much more like a snout than a real nose; also its eyes were getting extremely small for a baby: altogether Alice did not like the look of the thing at all. ‘But perhaps it was only sobbing,’ she thought, and looked into its eyes again, to see if there were any tears.

No, there were no tears. ‘If you’re going to turn into a pig, my dear,’ said Alice, seriously, ‘I’ll have nothing more to do with you. Mind now!’ The poor little thing sobbed again (or grunted, it was impossible to say which), and they went on for some while in silence.

Alice was just beginning to think to herself, ‘Now, what am I to do with this creature when I get it home?’ when it grunted again, so violently, that she looked down into its face in some alarm. This time there could be NO mistake about it: it was neither more nor less than a pig, and she felt that it would be quite absurd for her to carry it further.

So she set the little creature down, and felt quite relieved to see it trot away quietly into the wood. ‘If it had grown up,’ she said to herself, ‘it would have made a dreadfully ugly child: but it makes rather a handsome pig, I think.’ And she began thinking over other children she knew, who might do very well as pigs, and was just saying to herself, ‘if one only knew the right way to change them—’ when she was a little startled by seeing the Cheshire Cat sitting on a bough of a tree a few yards off.

The Cat only grinned when it saw Alice. It looked good-natured, she thought: still it had VERY long claws and a great many teeth, so she felt that it ought to be treated with respect.

‘Cheshire Puss,’ she began, rather timidly, as she did not at all know whether it would like the name: however, it only grinned a little wider. ‘Come, it’s pleased so far,’ thought Alice, and she went on. ‘Would you tell me, please, which way I ought to go from here?’

‘That depends a good deal on where you want to get to,’ said the Cat.

‘I don’t much care where—’ said Alice.

‘Then it doesn’t matter which way you go,’ said the Cat.

‘—so long as I get SOMEWHERE,’ Alice added as an explanation.

‘Oh, you’re sure to do that,’ said the Cat, ‘if you only walk long enough.’

Alice felt that this could not be denied, so she tried another question. ‘What sort of people live about here?’

‘In THAT direction,’ the Cat said, waving its right paw round, ‘lives a Hatter: and in THAT direction,’ waving the other paw, ‘lives a March Hare. Visit either you like: they’re both mad.’

‘But I don’t want to go among mad people,’ Alice remarked.

‘Oh, you can’t help that,’ said the Cat: ‘we’re all mad here. I’m mad. You’re mad.’

‘How do you know I’m mad?’ said Alice.

‘You must be,’ said the Cat, ‘or you wouldn’t have come here.’

Alice didn’t think that proved it at all; however, she went on ‘And how do you know that you’re mad?’

‘To begin with,’ said the Cat, ‘a dog’s not mad. You grant that?’

‘I suppose so,’ said Alice.

‘Well, then,’ the Cat went on, ‘you see, a dog growls when it’s angry, and wags its tail when it’s pleased. Now I growl when I’m pleased, and wag my tail when I’m angry. Therefore I’m mad.’

‘I call it purring, not growling,’ said Alice.

‘Call it what you like,’ said the Cat. ‘Do you play croquet with the Queen to-day?’

‘I should like it very much,’ said Alice, ‘but I haven’t been invited yet.’

‘You’ll see me there,’ said the Cat, and vanished.

Alice was not much surprised at this, she was getting so used to queer things happening. While she was looking at the place where it had been, it suddenly appeared again.

‘By-the-bye, what became of the baby?’ said the Cat. ‘I’d nearly forgotten to ask.’

‘It turned into a pig,’ Alice quietly said, just as if it had come back in a natural way.

‘I thought it would,’ said the Cat, and vanished again.

Alice waited a little, half expecting to see it again, but it did not appear, and after a minute or two she walked on in the direction in which the March Hare was said to live. ‘I’ve seen hatters before,’ she said to herself; ‘the March Hare will be much the most interesting, and perhaps as this is May it won’t be raving mad—at least not so mad as it was in March.’ As she said this, she looked up, and there was the Cat again, sitting on a branch of a tree.

‘Did you say pig, or fig?’ said the Cat.

‘I said pig,’ replied Alice; ‘and I wish you wouldn’t keep appearing and vanishing so suddenly: you make one quite giddy.’

‘All right,’ said the Cat; and this time it vanished quite slowly, beginning with the end of the tail, and ending with the grin, which remained some time after the rest of it had gone.

‘Well! I’ve often seen a cat without a grin,’ thought Alice; ‘but a grin without a cat! It’s the most curious thing I ever saw in my life!’

She had not gone much farther before she came in sight of the house of the March Hare: she thought it must be the right house, because the chimneys were shaped like ears and the roof was thatched with fur. It was so large a house, that she did not like to go nearer till she had nibbled some more of the lefthand bit of mushroom, and raised herself to about two feet high: even then she walked up towards it rather timidly, saying to herself ‘Suppose it should be raving mad after all! I almost wish I’d gone to see the Hatter instead!’




\section{A Mad Tea-Party}

There was a table set out under a tree in front of the house, and the March Hare and the Hatter were having tea at it: a Dormouse was sitting between them, fast asleep, and the other two were using it as a cushion, resting their elbows on it, and talking over its head. ‘Very uncomfortable for the Dormouse,’ thought Alice; ‘only, as it’s asleep, I suppose it doesn’t mind.’

The table was a large one, but the three were all crowded together at one corner of it: ‘No room! No room!’ they cried out when they saw Alice coming. ‘There’s PLENTY of room!’ said Alice indignantly, and she sat down in a large arm-chair at one end of the table.

‘Have some wine,’ the March Hare said in an encouraging tone.

Alice looked all round the table, but there was nothing on it but tea. ‘I don’t see any wine,’ she remarked.

‘There isn’t any,’ said the March Hare.

‘Then it wasn’t very civil of you to offer it,’ said Alice angrily.

‘It wasn’t very civil of you to sit down without being invited,’ said the March Hare.

‘I didn’t know it was YOUR table,’ said Alice; ‘it’s laid for a great many more than three.’

‘Your hair wants cutting,’ said the Hatter. He had been looking at Alice for some time with great curiosity, and this was his first speech.

‘You should learn not to make personal remarks,’ Alice said with some severity; ‘it’s very rude.’

The Hatter opened his eyes very wide on hearing this; but all he SAID was, ‘Why is a raven like a writing-desk?’

‘Come, we shall have some fun now!’ thought Alice. ‘I’m glad they’ve begun asking riddles.—I believe I can guess that,’ she added aloud.

‘Do you mean that you think you can find out the answer to it?’ said the March Hare.

‘Exactly so,’ said Alice.

‘Then you should say what you mean,’ the March Hare went on.

‘I do,’ Alice hastily replied; ‘at least—at least I mean what I say—that’s the same thing, you know.’

‘Not the same thing a bit!’ said the Hatter. ‘You might just as well say that “I see what I eat” is the same thing as “I eat what I see”!’

‘You might just as well say,’ added the March Hare, ‘that “I like what I get” is the same thing as “I get what I like”!’

‘You might just as well say,’ added the Dormouse, who seemed to be talking in his sleep, ‘that “I breathe when I sleep” is the same thing as “I sleep when I breathe”!’

‘It IS the same thing with you,’ said the Hatter, and here the conversation dropped, and the party sat silent for a minute, while Alice thought over all she could remember about ravens and writing-desks, which wasn’t much.

The Hatter was the first to break the silence. ‘What day of the month is it?’ he said, turning to Alice: he had taken his watch out of his pocket, and was looking at it uneasily, shaking it every now and then, and holding it to his ear.

Alice considered a little, and then said ‘The fourth.’

‘Two days wrong!’ sighed the Hatter. ‘I told you butter wouldn’t suit the works!’ he added looking angrily at the March Hare.

‘It was the BEST butter,’ the March Hare meekly replied.

‘Yes, but some crumbs must have got in as well,’ the Hatter grumbled: ‘you shouldn’t have put it in with the bread-knife.’

The March Hare took the watch and looked at it gloomily: then he dipped it into his cup of tea, and looked at it again: but he could think of nothing better to say than his first remark, ‘It was the BEST butter, you know.’

Alice had been looking over his shoulder with some curiosity. ‘What a funny watch!’ she remarked. ‘It tells the day of the month, and doesn’t tell what o’clock it is!’

‘Why should it?’ muttered the Hatter. ‘Does YOUR watch tell you what year it is?’

‘Of course not,’ Alice replied very readily: ‘but that’s because it stays the same year for such a long time together.’

‘Which is just the case with MINE,’ said the Hatter.

Alice felt dreadfully puzzled. The Hatter’s remark seemed to have no sort of meaning in it, and yet it was certainly English. ‘I don’t quite understand you,’ she said, as politely as she could.

‘The Dormouse is asleep again,’ said the Hatter, and he poured a little hot tea upon its nose.

The Dormouse shook its head impatiently, and said, without opening its eyes, ‘Of course, of course; just what I was going to remark myself.’

‘Have you guessed the riddle yet?’ the Hatter said, turning to Alice again.

‘No, I give it up,’ Alice replied: ‘what’s the answer?’

‘I haven’t the slightest idea,’ said the Hatter.

‘Nor I,’ said the March Hare.

Alice sighed wearily. ‘I think you might do something better with the time,’ she said, ‘than waste it in asking riddles that have no answers.’

‘If you knew Time as well as I do,’ said the Hatter, ‘you wouldn’t talk about wasting IT. It’s HIM.’

‘I don’t know what you mean,’ said Alice.

‘Of course you don’t!’ the Hatter said, tossing his head contemptuously. ‘I dare say you never even spoke to Time!’

‘Perhaps not,’ Alice cautiously replied: ‘but I know I have to beat time when I learn music.’

‘Ah! that accounts for it,’ said the Hatter. ‘He won’t stand beating. Now, if you only kept on good terms with him, he’d do almost anything you liked with the clock. For instance, suppose it were nine o’clock in the morning, just time to begin lessons: you’d only have to whisper a hint to Time, and round goes the clock in a twinkling! Half-past one, time for dinner!’

(‘I only wish it was,’ the March Hare said to itself in a whisper.)

‘That would be grand, certainly,’ said Alice thoughtfully: ‘but then—I shouldn’t be hungry for it, you know.’

‘Not at first, perhaps,’ said the Hatter: ‘but you could keep it to half-past one as long as you liked.’

‘Is that the way YOU manage?’ Alice asked.

The Hatter shook his head mournfully. ‘Not I!’ he replied. ‘We quarrelled last March—just before HE went mad, you know—’ (pointing with his tea spoon at the March Hare,) ‘—it was at the great concert given by the Queen of Hearts, and I had to sing

     “Twinkle, twinkle, little bat!
     How I wonder what you’re at!”

You know the song, perhaps?’

‘I’ve heard something like it,’ said Alice.

‘It goes on, you know,’ the Hatter continued, ‘in this way:—

     “Up above the world you fly,
     Like a tea-tray in the sky.
         Twinkle, twinkle—“’

Here the Dormouse shook itself, and began singing in its sleep ‘Twinkle, twinkle, twinkle, twinkle—’ and went on so long that they had to pinch it to make it stop.

‘Well, I’d hardly finished the first verse,’ said the Hatter, ‘when the Queen jumped up and bawled out, “He’s murdering the time! Off with his head!”’

‘How dreadfully savage!’ exclaimed Alice.

‘And ever since that,’ the Hatter went on in a mournful tone, ‘he won’t do a thing I ask! It’s always six o’clock now.’

A bright idea came into Alice’s head. ‘Is that the reason so many tea-things are put out here?’ she asked.

‘Yes, that’s it,’ said the Hatter with a sigh: ‘it’s always tea-time, and we’ve no time to wash the things between whiles.’

‘Then you keep moving round, I suppose?’ said Alice.

‘Exactly so,’ said the Hatter: ‘as the things get used up.’

‘But what happens when you come to the beginning again?’ Alice ventured to ask.

‘Suppose we change the subject,’ the March Hare interrupted, yawning. ‘I’m getting tired of this. I vote the young lady tells us a story.’

‘I’m afraid I don’t know one,’ said Alice, rather alarmed at the proposal.

‘Then the Dormouse shall!’ they both cried. ‘Wake up, Dormouse!’ And they pinched it on both sides at once.

The Dormouse slowly opened his eyes. ‘I wasn’t asleep,’ he said in a hoarse, feeble voice: ‘I heard every word you fellows were saying.’

‘Tell us a story!’ said the March Hare.

‘Yes, please do!’ pleaded Alice.

‘And be quick about it,’ added the Hatter, ‘or you’ll be asleep again before it’s done.’

‘Once upon a time there were three little sisters,’ the Dormouse began in a great hurry; ‘and their names were Elsie, Lacie, and Tillie; and they lived at the bottom of a well—’

‘What did they live on?’ said Alice, who always took a great interest in questions of eating and drinking.

‘They lived on treacle,’ said the Dormouse, after thinking a minute or two.

‘They couldn’t have done that, you know,’ Alice gently remarked; ‘they’d have been ill.’

‘So they were,’ said the Dormouse; ‘VERY ill.’

Alice tried to fancy to herself what such an extraordinary ways of living would be like, but it puzzled her too much, so she went on: ‘But why did they live at the bottom of a well?’

‘Take some more tea,’ the March Hare said to Alice, very earnestly.

‘I’ve had nothing yet,’ Alice replied in an offended tone, ‘so I can’t take more.’

‘You mean you can’t take LESS,’ said the Hatter: ‘it’s very easy to take MORE than nothing.’

‘Nobody asked YOUR opinion,’ said Alice.

‘Who’s making personal remarks now?’ the Hatter asked triumphantly.

Alice did not quite know what to say to this: so she helped herself to some tea and bread-and-butter, and then turned to the Dormouse, and repeated her question. ‘Why did they live at the bottom of a well?’

The Dormouse again took a minute or two to think about it, and then said, ‘It was a treacle-well.’

‘There’s no such thing!’ Alice was beginning very angrily, but the Hatter and the March Hare went ‘Sh! sh!’ and the Dormouse sulkily remarked, ‘If you can’t be civil, you’d better finish the story for yourself.’

‘No, please go on!’ Alice said very humbly; ‘I won’t interrupt again. I dare say there may be ONE.’

‘One, indeed!’ said the Dormouse indignantly. However, he consented to go on. ‘And so these three little sisters—they were learning to draw, you know—’

‘What did they draw?’ said Alice, quite forgetting her promise.

‘Treacle,’ said the Dormouse, without considering at all this time.

‘I want a clean cup,’ interrupted the Hatter: ‘let’s all move one place on.’

He moved on as he spoke, and the Dormouse followed him: the March Hare moved into the Dormouse’s place, and Alice rather unwillingly took the place of the March Hare. The Hatter was the only one who got any advantage from the change: and Alice was a good deal worse off than before, as the March Hare had just upset the milk-jug into his plate.

Alice did not wish to offend the Dormouse again, so she began very cautiously: ‘But I don’t understand. Where did they draw the treacle from?’

‘You can draw water out of a water-well,’ said the Hatter; ‘so I should think you could draw treacle out of a treacle-well—eh, stupid?’

‘But they were IN the well,’ Alice said to the Dormouse, not choosing to notice this last remark.

‘Of course they were’, said the Dormouse; ‘—well in.’

This answer so confused poor Alice, that she let the Dormouse go on for some time without interrupting it.

‘They were learning to draw,’ the Dormouse went on, yawning and rubbing its eyes, for it was getting very sleepy; ‘and they drew all manner of things—everything that begins with an M—’

‘Why with an M?’ said Alice.

‘Why not?’ said the March Hare.

Alice was silent.

The Dormouse had closed its eyes by this time, and was going off into a doze; but, on being pinched by the Hatter, it woke up again with a little shriek, and went on: ‘—that begins with an M, such as mouse-traps, and the moon, and memory, and muchness—you know you say things are “much of a muchness”—did you ever see such a thing as a drawing of a muchness?’

‘Really, now you ask me,’ said Alice, very much confused, ‘I don’t think—’

‘Then you shouldn’t talk,’ said the Hatter.

This piece of rudeness was more than Alice could bear: she got up in great disgust, and walked off; the Dormouse fell asleep instantly, and neither of the others took the least notice of her going, though she looked back once or twice, half hoping that they would call after her: the last time she saw them, they were trying to put the Dormouse into the teapot.

‘At any rate I’ll never go THERE again!’ said Alice as she picked her way through the wood. ‘It’s the stupidest tea-party I ever was at in all my life!’

Just as she said this, she noticed that one of the trees had a door leading right into it. ‘That’s very curious!’ she thought. ‘But everything’s curious today. I think I may as well go in at once.’ And in she went.

Once more she found herself in the long hall, and close to the little glass table. ‘Now, I’ll manage better this time,’ she said to herself, and began by taking the little golden key, and unlocking the door that led into the garden. Then she went to work nibbling at the mushroom (she had kept a piece of it in her pocket) till she was about a foot high: then she walked down the little passage: and THEN—she found herself at last in the beautiful garden, among the bright flower-beds and the cool fountains.




\section{The Queen’s Croquet-Ground}

A large rose-tree stood near the entrance of the garden: the roses growing on it were white, but there were three gardeners at it, busily painting them red. Alice thought this a very curious thing, and she went nearer to watch them, and just as she came up to them she heard one of them say, ‘Look out now, Five! Don’t go splashing paint over me like that!’

‘I couldn’t help it,’ said Five, in a sulky tone; ‘Seven jogged my elbow.’

On which Seven looked up and said, ‘That’s right, Five! Always lay the blame on others!’

‘YOU’D better not talk!’ said Five. ‘I heard the Queen say only yesterday you deserved to be beheaded!’

‘What for?’ said the one who had spoken first.

‘That’s none of YOUR business, Two!’ said Seven.

‘Yes, it IS his business!’ said Five, ‘and I’ll tell him—it was for bringing the cook tulip-roots instead of onions.’

Seven flung down his brush, and had just begun ‘Well, of all the unjust things—’ when his eye chanced to fall upon Alice, as she stood watching them, and he checked himself suddenly: the others looked round also, and all of them bowed low.

‘Would you tell me,’ said Alice, a little timidly, ‘why you are painting those roses?’

Five and Seven said nothing, but looked at Two. Two began in a low voice, ‘Why the fact is, you see, Miss, this here ought to have been a RED rose-tree, and we put a white one in by mistake; and if the Queen was to find it out, we should all have our heads cut off, you know. So you see, Miss, we’re doing our best, afore she comes, to—’ At this moment Five, who had been anxiously looking across the garden, called out ‘The Queen! The Queen!’ and the three gardeners instantly threw themselves flat upon their faces. There was a sound of many footsteps, and Alice looked round, eager to see the Queen.

First came ten soldiers carrying clubs; these were all shaped like the three gardeners, oblong and flat, with their hands and feet at the corners: next the ten courtiers; these were ornamented all over with diamonds, and walked two and two, as the soldiers did. After these came the royal children; there were ten of them, and the little dears came jumping merrily along hand in hand, in couples: they were all ornamented with hearts. Next came the guests, mostly Kings and Queens, and among them Alice recognised the White Rabbit: it was talking in a hurried nervous manner, smiling at everything that was said, and went by without noticing her. Then followed the Knave of Hearts, carrying the King’s crown on a crimson velvet cushion; and, last of all this grand procession, came THE KING AND QUEEN OF HEARTS.

Alice was rather doubtful whether she ought not to lie down on her face like the three gardeners, but she could not remember ever having heard of such a rule at processions; ‘and besides, what would be the use of a procession,’ thought she, ‘if people had all to lie down upon their faces, so that they couldn’t see it?’ So she stood still where she was, and waited.

When the procession came opposite to Alice, they all stopped and looked at her, and the Queen said severely ‘Who is this?’ She said it to the Knave of Hearts, who only bowed and smiled in reply.

‘Idiot!’ said the Queen, tossing her head impatiently; and, turning to Alice, she went on, ‘What’s your name, child?’

‘My name is Alice, so please your Majesty,’ said Alice very politely; but she added, to herself, ‘Why, they’re only a pack of cards, after all. I needn’t be afraid of them!’

‘And who are THESE?’ said the Queen, pointing to the three gardeners who were lying round the rosetree; for, you see, as they were lying on their faces, and the pattern on their backs was the same as the rest of the pack, she could not tell whether they were gardeners, or soldiers, or courtiers, or three of her own children.

‘How should I know?’ said Alice, surprised at her own courage. ‘It’s no business of MINE.’

The Queen turned crimson with fury, and, after glaring at her for a moment like a wild beast, screamed ‘Off with her head! Off—’

‘Nonsense!’ said Alice, very loudly and decidedly, and the Queen was silent.

The King laid his hand upon her arm, and timidly said ‘Consider, my dear: she is only a child!’

The Queen turned angrily away from him, and said to the Knave ‘Turn them over!’

The Knave did so, very carefully, with one foot.

‘Get up!’ said the Queen, in a shrill, loud voice, and the three gardeners instantly jumped up, and began bowing to the King, the Queen, the royal children, and everybody else.

‘Leave off that!’ screamed the Queen. ‘You make me giddy.’ And then, turning to the rose-tree, she went on, ‘What HAVE you been doing here?’

‘May it please your Majesty,’ said Two, in a very humble tone, going down on one knee as he spoke, ‘we were trying—’

‘I see!’ said the Queen, who had meanwhile been examining the roses. ‘Off with their heads!’ and the procession moved on, three of the soldiers remaining behind to execute the unfortunate gardeners, who ran to Alice for protection.

‘You shan’t be beheaded!’ said Alice, and she put them into a large flower-pot that stood near. The three soldiers wandered about for a minute or two, looking for them, and then quietly marched off after the others.

‘Are their heads off?’ shouted the Queen.

‘Their heads are gone, if it please your Majesty!’ the soldiers shouted in reply.

‘That’s right!’ shouted the Queen. ‘Can you play croquet?’

The soldiers were silent, and looked at Alice, as the question was evidently meant for her.

‘Yes!’ shouted Alice.

‘Come on, then!’ roared the Queen, and Alice joined the procession, wondering very much what would happen next.

‘It’s—it’s a very fine day!’ said a timid voice at her side. She was walking by the White Rabbit, who was peeping anxiously into her face.

‘Very,’ said Alice: ‘—where’s the Duchess?’

‘Hush! Hush!’ said the Rabbit in a low, hurried tone. He looked anxiously over his shoulder as he spoke, and then raised himself upon tiptoe, put his mouth close to her ear, and whispered ‘She’s under sentence of execution.’

‘What for?’ said Alice.

‘Did you say “What a pity!”?’ the Rabbit asked.

‘No, I didn’t,’ said Alice: ‘I don’t think it’s at all a pity. I said “What for?”’

‘She boxed the Queen’s ears—’ the Rabbit began. Alice gave a little scream of laughter. ‘Oh, hush!’ the Rabbit whispered in a frightened tone. ‘The Queen will hear you! You see, she came rather late, and the Queen said—’

‘Get to your places!’ shouted the Queen in a voice of thunder, and people began running about in all directions, tumbling up against each other; however, they got settled down in a minute or two, and the game began. Alice thought she had never seen such a curious croquet-ground in her life; it was all ridges and furrows; the balls were live hedgehogs, the mallets live flamingoes, and the soldiers had to double themselves up and to stand on their hands and feet, to make the arches.

The chief difficulty Alice found at first was in managing her flamingo: she succeeded in getting its body tucked away, comfortably enough, under her arm, with its legs hanging down, but generally, just as she had got its neck nicely straightened out, and was going to give the hedgehog a blow with its head, it WOULD twist itself round and look up in her face, with such a puzzled expression that she could not help bursting out laughing: and when she had got its head down, and was going to begin again, it was very provoking to find that the hedgehog had unrolled itself, and was in the act of crawling away: besides all this, there was generally a ridge or furrow in the way wherever she wanted to send the hedgehog to, and, as the doubled-up soldiers were always getting up and walking off to other parts of the ground, Alice soon came to the conclusion that it was a very difficult game indeed.

The players all played at once without waiting for turns, quarrelling all the while, and fighting for the hedgehogs; and in a very short time the Queen was in a furious passion, and went stamping about, and shouting ‘Off with his head!’ or ‘Off with her head!’ about once in a minute.

Alice began to feel very uneasy: to be sure, she had not as yet had any dispute with the Queen, but she knew that it might happen any minute, ‘and then,’ thought she, ‘what would become of me? They’re dreadfully fond of beheading people here; the great wonder is, that there’s any one left alive!’

She was looking about for some way of escape, and wondering whether she could get away without being seen, when she noticed a curious appearance in the air: it puzzled her very much at first, but, after watching it a minute or two, she made it out to be a grin, and she said to herself ‘It’s the Cheshire Cat: now I shall have somebody to talk to.’

‘How are you getting on?’ said the Cat, as soon as there was mouth enough for it to speak with.

Alice waited till the eyes appeared, and then nodded. ‘It’s no use speaking to it,’ she thought, ‘till its ears have come, or at least one of them.’ In another minute the whole head appeared, and then Alice put down her flamingo, and began an account of the game, feeling very glad she had someone to listen to her. The Cat seemed to think that there was enough of it now in sight, and no more of it appeared.

‘I don’t think they play at all fairly,’ Alice began, in rather a complaining tone, ‘and they all quarrel so dreadfully one can’t hear oneself speak—and they don’t seem to have any rules in particular; at least, if there are, nobody attends to them—and you’ve no idea how confusing it is all the things being alive; for instance, there’s the arch I’ve got to go through next walking about at the other end of the ground—and I should have croqueted the Queen’s hedgehog just now, only it ran away when it saw mine coming!’

‘How do you like the Queen?’ said the Cat in a low voice.

‘Not at all,’ said Alice: ‘she’s so extremely—’ Just then she noticed that the Queen was close behind her, listening: so she went on, ‘—likely to win, that it’s hardly worth while finishing the game.’

The Queen smiled and passed on.

‘Who ARE you talking to?’ said the King, going up to Alice, and looking at the Cat’s head with great curiosity.

‘It’s a friend of mine—a Cheshire Cat,’ said Alice: ‘allow me to introduce it.’

‘I don’t like the look of it at all,’ said the King: ‘however, it may kiss my hand if it likes.’

‘I’d rather not,’ the Cat remarked.

‘Don’t be impertinent,’ said the King, ‘and don’t look at me like that!’ He got behind Alice as he spoke.

‘A cat may look at a king,’ said Alice. ‘I’ve read that in some book, but I don’t remember where.’

‘Well, it must be removed,’ said the King very decidedly, and he called the Queen, who was passing at the moment, ‘My dear! I wish you would have this cat removed!’

The Queen had only one way of settling all difficulties, great or small. ‘Off with his head!’ she said, without even looking round.

‘I’ll fetch the executioner myself,’ said the King eagerly, and he hurried off.

Alice thought she might as well go back, and see how the game was going on, as she heard the Queen’s voice in the distance, screaming with passion. She had already heard her sentence three of the players to be executed for having missed their turns, and she did not like the look of things at all, as the game was in such confusion that she never knew whether it was her turn or not. So she went in search of her hedgehog.

The hedgehog was engaged in a fight with another hedgehog, which seemed to Alice an excellent opportunity for croqueting one of them with the other: the only difficulty was, that her flamingo was gone across to the other side of the garden, where Alice could see it trying in a helpless sort of way to fly up into a tree.

By the time she had caught the flamingo and brought it back, the fight was over, and both the hedgehogs were out of sight: ‘but it doesn’t matter much,’ thought Alice, ‘as all the arches are gone from this side of the ground.’ So she tucked it away under her arm, that it might not escape again, and went back for a little more conversation with her friend.

When she got back to the Cheshire Cat, she was surprised to find quite a large crowd collected round it: there was a dispute going on between the executioner, the King, and the Queen, who were all talking at once, while all the rest were quite silent, and looked very uncomfortable.

The moment Alice appeared, she was appealed to by all three to settle the question, and they repeated their arguments to her, though, as they all spoke at once, she found it very hard indeed to make out exactly what they said.

The executioner’s argument was, that you couldn’t cut off a head unless there was a body to cut it off from: that he had never had to do such a thing before, and he wasn’t going to begin at HIS time of life.

The King’s argument was, that anything that had a head could be beheaded, and that you weren’t to talk nonsense.

The Queen’s argument was, that if something wasn’t done about it in less than no time she’d have everybody executed, all round. (It was this last remark that had made the whole party look so grave and anxious.)

Alice could think of nothing else to say but ‘It belongs to the Duchess: you’d better ask HER about it.’

‘She’s in prison,’ the Queen said to the executioner: ‘fetch her here.’ And the executioner went off like an arrow.

 The Cat’s head began fading away the moment he was gone, and, by the time he had come back with the Duchess, it had entirely disappeared; so the King and the executioner ran wildly up and down looking for it, while the rest of the party went back to the game.




\section{The Mock Turtle’s Story}

‘You can’t think how glad I am to see you again, you dear old thing!’ said the Duchess, as she tucked her arm affectionately into Alice’s, and they walked off together.

Alice was very glad to find her in such a pleasant temper, and thought to herself that perhaps it was only the pepper that had made her so savage when they met in the kitchen.

‘When I’M a Duchess,’ she said to herself, (not in a very hopeful tone though), ‘I won’t have any pepper in my kitchen AT ALL. Soup does very well without—Maybe it’s always pepper that makes people hot-tempered,’ she went on, very much pleased at having found out a new kind of rule, ‘and vinegar that makes them sour—and camomile that makes them bitter—and—and barley-sugar and such things that make children sweet-tempered. I only wish people knew that: then they wouldn’t be so stingy about it, you know—’

She had quite forgotten the Duchess by this time, and was a little startled when she heard her voice close to her ear. ‘You’re thinking about something, my dear, and that makes you forget to talk. I can’t tell you just now what the moral of that is, but I shall remember it in a bit.’

‘Perhaps it hasn’t one,’ Alice ventured to remark.

‘Tut, tut, child!’ said the Duchess. ‘Everything’s got a moral, if only you can find it.’ And she squeezed herself up closer to Alice’s side as she spoke.

Alice did not much like keeping so close to her: first, because the Duchess was VERY ugly; and secondly, because she was exactly the right height to rest her chin upon Alice’s shoulder, and it was an uncomfortably sharp chin. However, she did not like to be rude, so she bore it as well as she could.

‘The game’s going on rather better now,’ she said, by way of keeping up the conversation a little.

‘’Tis so,’ said the Duchess: ‘and the moral of that is—“Oh, ’tis love, ’tis love, that makes the world go round!”’

‘Somebody said,’ Alice whispered, ‘that it’s done by everybody minding their own business!’

‘Ah, well! It means much the same thing,’ said the Duchess, digging her sharp little chin into Alice’s shoulder as she added, ‘and the moral of THAT is—“Take care of the sense, and the sounds will take care of themselves.”’

‘How fond she is of finding morals in things!’ Alice thought to herself.

‘I dare say you’re wondering why I don’t put my arm round your waist,’ the Duchess said after a pause: ‘the reason is, that I’m doubtful about the temper of your flamingo. Shall I try the experiment?’

‘HE might bite,’ Alice cautiously replied, not feeling at all anxious to have the experiment tried.

‘Very true,’ said the Duchess: ‘flamingoes and mustard both bite. And the moral of that is—“Birds of a feather flock together.”’

‘Only mustard isn’t a bird,’ Alice remarked.

‘Right, as usual,’ said the Duchess: ‘what a clear way you have of putting things!’

‘It’s a mineral, I THINK,’ said Alice.

‘Of course it is,’ said the Duchess, who seemed ready to agree to everything that Alice said; ‘there’s a large mustard-mine near here. And the moral of that is—“The more there is of mine, the less there is of yours.”’

‘Oh, I know!’ exclaimed Alice, who had not attended to this last remark, ‘it’s a vegetable. It doesn’t look like one, but it is.’

‘I quite agree with you,’ said the Duchess; ‘and the moral of that is—“Be what you would seem to be”—or if you’d like it put more simply—“Never imagine yourself not to be otherwise than what it might appear to others that what you were or might have been was not otherwise than what you had been would have appeared to them to be otherwise.”’

‘I think I should understand that better,’ Alice said very politely, ‘if I had it written down: but I can’t quite follow it as you say it.’

‘That’s nothing to what I could say if I chose,’ the Duchess replied, in a pleased tone.

‘Pray don’t trouble yourself to say it any longer than that,’ said Alice.

‘Oh, don’t talk about trouble!’ said the Duchess. ‘I make you a present of everything I’ve said as yet.’

‘A cheap sort of present!’ thought Alice. ‘I’m glad they don’t give birthday presents like that!’ But she did not venture to say it out loud.

‘Thinking again?’ the Duchess asked, with another dig of her sharp little chin.

‘I’ve a right to think,’ said Alice sharply, for she was beginning to feel a little worried.

‘Just about as much right,’ said the Duchess, ‘as pigs have to fly; and the m—’

But here, to Alice’s great surprise, the Duchess’s voice died away, even in the middle of her favourite word ‘moral,’ and the arm that was linked into hers began to tremble. Alice looked up, and there stood the Queen in front of them, with her arms folded, frowning like a thunderstorm.

‘A fine day, your Majesty!’ the Duchess began in a low, weak voice.

‘Now, I give you fair warning,’ shouted the Queen, stamping on the ground as she spoke; ‘either you or your head must be off, and that in about half no time! Take your choice!’

The Duchess took her choice, and was gone in a moment.

‘Let’s go on with the game,’ the Queen said to Alice; and Alice was too much frightened to say a word, but slowly followed her back to the croquet-ground.

The other guests had taken advantage of the Queen’s absence, and were resting in the shade: however, the moment they saw her, they hurried back to the game, the Queen merely remarking that a moment’s delay would cost them their lives.

All the time they were playing the Queen never left off quarrelling with the other players, and shouting ‘Off with his head!’ or ‘Off with her head!’ Those whom she sentenced were taken into custody by the soldiers, who of course had to leave off being arches to do this, so that by the end of half an hour or so there were no arches left, and all the players, except the King, the Queen, and Alice, were in custody and under sentence of execution.

Then the Queen left off, quite out of breath, and said to Alice, ‘Have you seen the Mock Turtle yet?’

‘No,’ said Alice. ‘I don’t even know what a Mock Turtle is.’

‘It’s the thing Mock Turtle Soup is made from,’ said the Queen.

‘I never saw one, or heard of one,’ said Alice.

‘Come on, then,’ said the Queen, ‘and he shall tell you his history,’

As they walked off together, Alice heard the King say in a low voice, to the company generally, ‘You are all pardoned.’ ‘Come, THAT’S a good thing!’ she said to herself, for she had felt quite unhappy at the number of executions the Queen had ordered.

They very soon came upon a Gryphon, lying fast asleep in the sun. (IF you don’t know what a Gryphon is, look at the picture.) ‘Up, lazy thing!’ said the Queen, ‘and take this young lady to see the Mock Turtle, and to hear his history. I must go back and see after some executions I have ordered’; and she walked off, leaving Alice alone with the Gryphon. Alice did not quite like the look of the creature, but on the whole she thought it would be quite as safe to stay with it as to go after that savage Queen: so she waited.

The Gryphon sat up and rubbed its eyes: then it watched the Queen till she was out of sight: then it chuckled. ‘What fun!’ said the Gryphon, half to itself, half to Alice.

‘What IS the fun?’ said Alice.

‘Why, SHE,’ said the Gryphon. ‘It’s all her fancy, that: they never executes nobody, you know. Come on!’

‘Everybody says “come on!” here,’ thought Alice, as she went slowly after it: ‘I never was so ordered about in all my life, never!’

They had not gone far before they saw the Mock Turtle in the distance, sitting sad and lonely on a little ledge of rock, and, as they came nearer, Alice could hear him sighing as if his heart would break. She pitied him deeply. ‘What is his sorrow?’ she asked the Gryphon, and the Gryphon answered, very nearly in the same words as before, ‘It’s all his fancy, that: he hasn’t got no sorrow, you know. Come on!’

So they went up to the Mock Turtle, who looked at them with large eyes full of tears, but said nothing.

‘This here young lady,’ said the Gryphon, ‘she wants for to know your history, she do.’

‘I’ll tell it her,’ said the Mock Turtle in a deep, hollow tone: ‘sit down, both of you, and don’t speak a word till I’ve finished.’

So they sat down, and nobody spoke for some minutes. Alice thought to herself, ‘I don’t see how he can EVEN finish, if he doesn’t begin.’ But she waited patiently.

‘Once,’ said the Mock Turtle at last, with a deep sigh, ‘I was a real Turtle.’

These words were followed by a very long silence, broken only by an occasional exclamation of ‘Hjckrrh!’ from the Gryphon, and the constant heavy sobbing of the Mock Turtle. Alice was very nearly getting up and saying, ‘Thank you, sir, for your interesting story,’ but she could not help thinking there MUST be more to come, so she sat still and said nothing.

‘When we were little,’ the Mock Turtle went on at last, more calmly, though still sobbing a little now and then, ‘we went to school in the sea. The master was an old Turtle—we used to call him Tortoise—’

‘Why did you call him Tortoise, if he wasn’t one?’ Alice asked.

‘We called him Tortoise because he taught us,’ said the Mock Turtle angrily: ‘really you are very dull!’

‘You ought to be ashamed of yourself for asking such a simple question,’ added the Gryphon; and then they both sat silent and looked at poor Alice, who felt ready to sink into the earth. At last the Gryphon said to the Mock Turtle, ‘Drive on, old fellow! Don’t be all day about it!’ and he went on in these words:

‘Yes, we went to school in the sea, though you mayn’t believe it—’

‘I never said I didn’t!’ interrupted Alice.

‘You did,’ said the Mock Turtle.

‘Hold your tongue!’ added the Gryphon, before Alice could speak again. The Mock Turtle went on.

‘We had the best of educations—in fact, we went to school every day—’

‘I’VE been to a day-school, too,’ said Alice; ‘you needn’t be so proud as all that.’

‘With extras?’ asked the Mock Turtle a little anxiously.

‘Yes,’ said Alice, ‘we learned French and music.’

‘And washing?’ said the Mock Turtle.

‘Certainly not!’ said Alice indignantly.

‘Ah! then yours wasn’t a really good school,’ said the Mock Turtle in a tone of great relief. ‘Now at OURS they had at the end of the bill, “French, music, AND WASHING—extra.”’

‘You couldn’t have wanted it much,’ said Alice; ‘living at the bottom of the sea.’

‘I couldn’t afford to learn it.’ said the Mock Turtle with a sigh. ‘I only took the regular course.’

‘What was that?’ inquired Alice.

‘Reeling and Writhing, of course, to begin with,’ the Mock Turtle replied; ‘and then the different branches of Arithmetic—Ambition, Distraction, Uglification, and Derision.’

‘I never heard of “Uglification,”’ Alice ventured to say. ‘What is it?’

The Gryphon lifted up both its paws in surprise. ‘What! Never heard of uglifying!’ it exclaimed. ‘You know what to beautify is, I suppose?’

‘Yes,’ said Alice doubtfully: ‘it means—to—make—anything—prettier.’

‘Well, then,’ the Gryphon went on, ‘if you don’t know what to uglify is, you ARE a simpleton.’

Alice did not feel encouraged to ask any more questions about it, so she turned to the Mock Turtle, and said ‘What else had you to learn?’

‘Well, there was Mystery,’ the Mock Turtle replied, counting off the subjects on his flappers, ‘—Mystery, ancient and modern, with Seaography: then Drawling—the Drawling-master was an old conger-eel, that used to come once a week: HE taught us Drawling, Stretching, and Fainting in Coils.’

‘What was THAT like?’ said Alice.

‘Well, I can’t show it you myself,’ the Mock Turtle said: ‘I’m too stiff. And the Gryphon never learnt it.’

‘Hadn’t time,’ said the Gryphon: ‘I went to the Classics master, though. He was an old crab, HE was.’

‘I never went to him,’ the Mock Turtle said with a sigh: ‘he taught Laughing and Grief, they used to say.’

‘So he did, so he did,’ said the Gryphon, sighing in his turn; and both creatures hid their faces in their paws.

‘And how many hours a day did you do lessons?’ said Alice, in a hurry to change the subject.

‘Ten hours the first day,’ said the Mock Turtle: ‘nine the next, and so on.’

‘What a curious plan!’ exclaimed Alice.

‘That’s the reason they’re called lessons,’ the Gryphon remarked: ‘because they lessen from day to day.’

This was quite a new idea to Alice, and she thought it over a little before she made her next remark. ‘Then the eleventh day must have been a holiday?’

‘Of course it was,’ said the Mock Turtle.

‘And how did you manage on the twelfth?’ Alice went on eagerly.

‘That’s enough about lessons,’ the Gryphon interrupted in a very decided tone: ‘tell her something about the games now.’




\section{The Lobster Quadrille}

The Mock Turtle sighed deeply, and drew the back of one flapper across his eyes. He looked at Alice, and tried to speak, but for a minute or two sobs choked his voice. ‘Same as if he had a bone in his throat,’ said the Gryphon: and it set to work shaking him and punching him in the back. At last the Mock Turtle recovered his voice, and, with tears running down his cheeks, he went on again:—

‘You may not have lived much under the sea—’ (‘I haven’t,’ said Alice)—‘and perhaps you were never even introduced to a lobster—’ (Alice began to say ‘I once tasted—’ but checked herself hastily, and said ‘No, never’) ‘—so you can have no idea what a delightful thing a Lobster Quadrille is!’

‘No, indeed,’ said Alice. ‘What sort of a dance is it?’

‘Why,’ said the Gryphon, ‘you first form into a line along the sea-shore—’

‘Two lines!’ cried the Mock Turtle. ‘Seals, turtles, salmon, and so on; then, when you’ve cleared all the jelly-fish out of the way—’

‘THAT generally takes some time,’ interrupted the Gryphon.

‘—you advance twice—’

‘Each with a lobster as a partner!’ cried the Gryphon.

‘Of course,’ the Mock Turtle said: ‘advance twice, set to partners—’

‘—change lobsters, and retire in same order,’ continued the Gryphon.

‘Then, you know,’ the Mock Turtle went on, ‘you throw the—’

‘The lobsters!’ shouted the Gryphon, with a bound into the air.

‘—as far out to sea as you can—’

‘Swim after them!’ screamed the Gryphon.

‘Turn a somersault in the sea!’ cried the Mock Turtle, capering wildly about.

‘Change lobsters again!’ yelled the Gryphon at the top of its voice.

‘Back to land again, and that’s all the first figure,’ said the Mock Turtle, suddenly dropping his voice; and the two creatures, who had been jumping about like mad things all this time, sat down again very sadly and quietly, and looked at Alice.

‘It must be a very pretty dance,’ said Alice timidly.

‘Would you like to see a little of it?’ said the Mock Turtle.

‘Very much indeed,’ said Alice.

‘Come, let’s try the first figure!’ said the Mock Turtle to the Gryphon. ‘We can do without lobsters, you know. Which shall sing?’

‘Oh, YOU sing,’ said the Gryphon. ‘I’ve forgotten the words.’

So they began solemnly dancing round and round Alice, every now and then treading on her toes when they passed too close, and waving their forepaws to mark the time, while the Mock Turtle sang this, very slowly and sadly:—

 ‘“Will you walk a little faster?” said a whiting to a snail.
 “There’s a porpoise close behind us, and he’s treading on my tail.

 See how eagerly the lobsters and the turtles all advance!
 They are waiting on the shingle—will you come and join the dance?

 Will you, won’t you, will you, won’t you, will you join the dance?
 Will you, won’t you, will you, won’t you, won’t you join the dance?

 “You can really have no notion how delightful it will be
 When they take us up and throw us, with the lobsters, out to sea!”
 But the snail replied “Too far, too far!” and gave a look askance—
 Said he thanked the whiting kindly, but he would not join the dance.

 Would not, could not, would not, could not, would not join the dance.
 Would not, could not, would not, could not, could not join the dance.

 ‘“What matters it how far we go?” his scaly friend replied.
 “There is another shore, you know, upon the other side.
 The further off from England the nearer is to France—
 Then turn not pale, beloved snail, but come and join the dance.

 Will you, won’t you, will you, won’t you, will you join the dance?
 Will you, won’t you, will you, won’t you, won’t you join the dance?”’

‘Thank you, it’s a very interesting dance to watch,’ said Alice, feeling very glad that it was over at last: ‘and I do so like that curious song about the whiting!’

‘Oh, as to the whiting,’ said the Mock Turtle, ‘they—you’ve seen them, of course?’

‘Yes,’ said Alice, ‘I’ve often seen them at dinn—’ she checked herself hastily.

‘I don’t know where Dinn may be,’ said the Mock Turtle, ‘but if you’ve seen them so often, of course you know what they’re like.’

‘I believe so,’ Alice replied thoughtfully. ‘They have their tails in their mouths—and they’re all over crumbs.’

‘You’re wrong about the crumbs,’ said the Mock Turtle: ‘crumbs would all wash off in the sea. But they HAVE their tails in their mouths; and the reason is—’ here the Mock Turtle yawned and shut his eyes.—‘Tell her about the reason and all that,’ he said to the Gryphon.

‘The reason is,’ said the Gryphon, ‘that they WOULD go with the lobsters to the dance. So they got thrown out to sea. So they had to fall a long way. So they got their tails fast in their mouths. So they couldn’t get them out again. That’s all.’

‘Thank you,’ said Alice, ‘it’s very interesting. I never knew so much about a whiting before.’

‘I can tell you more than that, if you like,’ said the Gryphon. ‘Do you know why it’s called a whiting?’

‘I never thought about it,’ said Alice. ‘Why?’

‘IT DOES THE BOOTS AND SHOES.’ the Gryphon replied very solemnly.

Alice was thoroughly puzzled. ‘Does the boots and shoes!’ she repeated in a wondering tone.

‘Why, what are YOUR shoes done with?’ said the Gryphon. ‘I mean, what makes them so shiny?’

Alice looked down at them, and considered a little before she gave her answer. ‘They’re done with blacking, I believe.’

‘Boots and shoes under the sea,’ the Gryphon went on in a deep voice, ‘are done with a whiting. Now you know.’

‘And what are they made of?’ Alice asked in a tone of great curiosity.

‘Soles and eels, of course,’ the Gryphon replied rather impatiently: ‘any shrimp could have told you that.’

‘If I’d been the whiting,’ said Alice, whose thoughts were still running on the song, ‘I’d have said to the porpoise, “Keep back, please: we don’t want YOU with us!”’

‘They were obliged to have him with them,’ the Mock Turtle said: ‘no wise fish would go anywhere without a porpoise.’

‘Wouldn’t it really?’ said Alice in a tone of great surprise.

‘Of course not,’ said the Mock Turtle: ‘why, if a fish came to ME, and told me he was going a journey, I should say “With what porpoise?”’

‘Don’t you mean “purpose”?’ said Alice.

‘I mean what I say,’ the Mock Turtle replied in an offended tone. And the Gryphon added ‘Come, let’s hear some of YOUR adventures.’

‘I could tell you my adventures—beginning from this morning,’ said Alice a little timidly: ‘but it’s no use going back to yesterday, because I was a different person then.’

‘Explain all that,’ said the Mock Turtle.

‘No, no! The adventures first,’ said the Gryphon in an impatient tone: ‘explanations take such a dreadful time.’

So Alice began telling them her adventures from the time when she first saw the White Rabbit. She was a little nervous about it just at first, the two creatures got so close to her, one on each side, and opened their eyes and mouths so VERY wide, but she gained courage as she went on. Her listeners were perfectly quiet till she got to the part about her repeating ‘YOU ARE OLD, FATHER WILLIAM,’ to the Caterpillar, and the words all coming different, and then the Mock Turtle drew a long breath, and said ‘That’s very curious.’

‘It’s all about as curious as it can be,’ said the Gryphon.

‘It all came different!’ the Mock Turtle repeated thoughtfully. ‘I should like to hear her try and repeat something now. Tell her to begin.’ He looked at the Gryphon as if he thought it had some kind of authority over Alice.

‘Stand up and repeat ”’TIS THE VOICE OF THE SLUGGARD,”’ said the Gryphon.

‘How the creatures order one about, and make one repeat lessons!’ thought Alice; ‘I might as well be at school at once.’ However, she got up, and began to repeat it, but her head was so full of the Lobster Quadrille, that she hardly knew what she was saying, and the words came very queer indeed:—

  ‘’Tis the voice of the Lobster; I heard him declare,
  “You have baked me too brown, I must sugar my hair.”
  As a duck with its eyelids, so he with his nose
  Trims his belt and his buttons, and turns out his toes.’

       [later editions continued as follows
  When the sands are all dry, he is gay as a lark,
  And will talk in contemptuous tones of the Shark,
  But, when the tide rises and sharks are around,
  His voice has a timid and tremulous sound.]

‘That’s different from what I used to say when I was a child,’ said the Gryphon.

‘Well, I never heard it before,’ said the Mock Turtle; ‘but it sounds uncommon nonsense.’

Alice said nothing; she had sat down with her face in her hands, wondering if anything would EVER happen in a natural way again.

‘I should like to have it explained,’ said the Mock Turtle.

‘She can’t explain it,’ said the Gryphon hastily. ‘Go on with the next verse.’

‘But about his toes?’ the Mock Turtle persisted. ‘How COULD he turn them out with his nose, you know?’

‘It’s the first position in dancing.’ Alice said; but was dreadfully puzzled by the whole thing, and longed to change the subject.

‘Go on with the next verse,’ the Gryphon repeated impatiently: ‘it begins “I passed by his garden.”’

Alice did not dare to disobey, though she felt sure it would all come wrong, and she went on in a trembling voice:—

  ‘I passed by his garden, and marked, with one eye,
  How the Owl and the Panther were sharing a pie—’

    [later editions continued as follows
  The Panther took pie-crust, and gravy, and meat,
  While the Owl had the dish as its share of the treat.
  When the pie was all finished, the Owl, as a boon,
  Was kindly permitted to pocket the spoon:
  While the Panther received knife and fork with a growl,
  And concluded the banquet—]

‘What IS the use of repeating all that stuff,’ the Mock Turtle interrupted, ‘if you don’t explain it as you go on? It’s by far the most confusing thing I ever heard!’

‘Yes, I think you’d better leave off,’ said the Gryphon: and Alice was only too glad to do so.

‘Shall we try another figure of the Lobster Quadrille?’ the Gryphon went on. ‘Or would you like the Mock Turtle to sing you a song?’

‘Oh, a song, please, if the Mock Turtle would be so kind,’ Alice replied, so eagerly that the Gryphon said, in a rather offended tone, ‘Hm! No accounting for tastes! Sing her “Turtle Soup,” will you, old fellow?’

The Mock Turtle sighed deeply, and began, in a voice sometimes choked with sobs, to sing this:—

   ‘Beautiful Soup, so rich and green,
   Waiting in a hot tureen!
   Who for such dainties would not stoop?
   Soup of the evening, beautiful Soup!
   Soup of the evening, beautiful Soup!
     Beau—ootiful Soo—oop!
     Beau—ootiful Soo—oop!
   Soo—oop of the e—e—evening,
     Beautiful, beautiful Soup!

   ‘Beautiful Soup! Who cares for fish,
   Game, or any other dish?
   Who would not give all else for two
   Pennyworth only of beautiful Soup?
   Pennyworth only of beautiful Soup?
     Beau—ootiful Soo—oop!
     Beau—ootiful Soo—oop!
   Soo—oop of the e—e—evening,
     Beautiful, beauti—FUL SOUP!’

‘Chorus again!’ cried the Gryphon, and the Mock Turtle had just begun to repeat it, when a cry of ‘The trial’s beginning!’ was heard in the distance.

‘Come on!’ cried the Gryphon, and, taking Alice by the hand, it hurried off, without waiting for the end of the song.

‘What trial is it?’ Alice panted as she ran; but the Gryphon only answered ‘Come on!’ and ran the faster, while more and more faintly came, carried on the breeze that followed them, the melancholy words:—

   ‘Soo—oop of the e—e—evening,
     Beautiful, beautiful Soup!’




\section{Who Stole the Tarts?}

The King and Queen of Hearts were seated on their throne when they arrived, with a great crowd assembled about them—all sorts of little birds and beasts, as well as the whole pack of cards: the Knave was standing before them, in chains, with a soldier on each side to guard him; and near the King was the White Rabbit, with a trumpet in one hand, and a scroll of parchment in the other. In the very middle of the court was a table, with a large dish of tarts upon it: they looked so good, that it made Alice quite hungry to look at them—‘I wish they’d get the trial done,’ she thought, ‘and hand round the refreshments!’ But there seemed to be no chance of this, so she began looking at everything about her, to pass away the time.

Alice had never been in a court of justice before, but she had read about them in books, and she was quite pleased to find that she knew the name of nearly everything there. ‘That’s the judge,’ she said to herself, ‘because of his great wig.’

The judge, by the way, was the King; and as he wore his crown over the wig, (look at the frontispiece if you want to see how he did it,) he did not look at all comfortable, and it was certainly not becoming.

‘And that’s the jury-box,’ thought Alice, ‘and those twelve creatures,’ (she was obliged to say ‘creatures,’ you see, because some of them were animals, and some were birds,) ‘I suppose they are the jurors.’ She said this last word two or three times over to herself, being rather proud of it: for she thought, and rightly too, that very few little girls of her age knew the meaning of it at all. However, ‘jury-men’ would have done just as well.

The twelve jurors were all writing very busily on slates. ‘What are they doing?’ Alice whispered to the Gryphon. ‘They can’t have anything to put down yet, before the trial’s begun.’

‘They’re putting down their names,’ the Gryphon whispered in reply, ‘for fear they should forget them before the end of the trial.’

‘Stupid things!’ Alice began in a loud, indignant voice, but she stopped hastily, for the White Rabbit cried out, ‘Silence in the court!’ and the King put on his spectacles and looked anxiously round, to make out who was talking.

Alice could see, as well as if she were looking over their shoulders, that all the jurors were writing down ‘stupid things!’ on their slates, and she could even make out that one of them didn’t know how to spell ‘stupid,’ and that he had to ask his neighbour to tell him. ‘A nice muddle their slates’ll be in before the trial’s over!’ thought Alice.

One of the jurors had a pencil that squeaked. This of course, Alice could not stand, and she went round the court and got behind him, and very soon found an opportunity of taking it away. She did it so quickly that the poor little juror (it was Bill, the Lizard) could not make out at all what had become of it; so, after hunting all about for it, he was obliged to write with one finger for the rest of the day; and this was of very little use, as it left no mark on the slate.

‘Herald, read the accusation!’ said the King.

On this the White Rabbit blew three blasts on the trumpet, and then unrolled the parchment scroll, and read as follows:—

   ‘The Queen of Hearts, she made some tarts,
      All on a summer day:
    The Knave of Hearts, he stole those tarts,
      And took them quite away!’

‘Consider your verdict,’ the King said to the jury.

‘Not yet, not yet!’ the Rabbit hastily interrupted. ‘There’s a great deal to come before that!’

‘Call the first witness,’ said the King; and the White Rabbit blew three blasts on the trumpet, and called out, ‘First witness!’

The first witness was the Hatter. He came in with a teacup in one hand and a piece of bread-and-butter in the other. ‘I beg pardon, your Majesty,’ he began, ‘for bringing these in: but I hadn’t quite finished my tea when I was sent for.’

‘You ought to have finished,’ said the King. ‘When did you begin?’

The Hatter looked at the March Hare, who had followed him into the court, arm-in-arm with the Dormouse. ‘Fourteenth of March, I think it was,’ he said.

‘Fifteenth,’ said the March Hare.

‘Sixteenth,’ added the Dormouse.

‘Write that down,’ the King said to the jury, and the jury eagerly wrote down all three dates on their slates, and then added them up, and reduced the answer to shillings and pence.

‘Take off your hat,’ the King said to the Hatter.

‘It isn’t mine,’ said the Hatter.

‘Stolen!’ the King exclaimed, turning to the jury, who instantly made a memorandum of the fact.

‘I keep them to sell,’ the Hatter added as an explanation; ‘I’ve none of my own. I’m a hatter.’

Here the Queen put on her spectacles, and began staring at the Hatter, who turned pale and fidgeted.

‘Give your evidence,’ said the King; ‘and don’t be nervous, or I’ll have you executed on the spot.’

This did not seem to encourage the witness at all: he kept shifting from one foot to the other, looking uneasily at the Queen, and in his confusion he bit a large piece out of his teacup instead of the bread-and-butter.

Just at this moment Alice felt a very curious sensation, which puzzled her a good deal until she made out what it was: she was beginning to grow larger again, and she thought at first she would get up and leave the court; but on second thoughts she decided to remain where she was as long as there was room for her.

‘I wish you wouldn’t squeeze so.’ said the Dormouse, who was sitting next to her. ‘I can hardly breathe.’

‘I can’t help it,’ said Alice very meekly: ‘I’m growing.’

‘You’ve no right to grow here,’ said the Dormouse.

‘Don’t talk nonsense,’ said Alice more boldly: ‘you know you’re growing too.’

‘Yes, but I grow at a reasonable pace,’ said the Dormouse: ‘not in that ridiculous fashion.’ And he got up very sulkily and crossed over to the other side of the court.

All this time the Queen had never left off staring at the Hatter, and, just as the Dormouse crossed the court, she said to one of the officers of the court, ‘Bring me the list of the singers in the last concert!’ on which the wretched Hatter trembled so, that he shook both his shoes off.

‘Give your evidence,’ the King repeated angrily, ‘or I’ll have you executed, whether you’re nervous or not.’

‘I’m a poor man, your Majesty,’ the Hatter began, in a trembling voice, ‘—and I hadn’t begun my tea—not above a week or so—and what with the bread-and-butter getting so thin—and the twinkling of the tea—’

‘The twinkling of the what?’ said the King.

‘It began with the tea,’ the Hatter replied.

‘Of course twinkling begins with a T!’ said the King sharply. ‘Do you take me for a dunce? Go on!’

‘I’m a poor man,’ the Hatter went on, ‘and most things twinkled after that—only the March Hare said—’

‘I didn’t!’ the March Hare interrupted in a great hurry.

‘You did!’ said the Hatter.

‘I deny it!’ said the March Hare.

‘He denies it,’ said the King: ‘leave out that part.’

‘Well, at any rate, the Dormouse said—’ the Hatter went on, looking anxiously round to see if he would deny it too: but the Dormouse denied nothing, being fast asleep.

‘After that,’ continued the Hatter, ‘I cut some more bread-and-butter—’

‘But what did the Dormouse say?’ one of the jury asked.

‘That I can’t remember,’ said the Hatter.

‘You MUST remember,’ remarked the King, ‘or I’ll have you executed.’

The miserable Hatter dropped his teacup and bread-and-butter, and went down on one knee. ‘I’m a poor man, your Majesty,’ he began.

‘You’re a very poor speaker,’ said the King.

Here one of the guinea-pigs cheered, and was immediately suppressed by the officers of the court. (As that is rather a hard word, I will just explain to you how it was done. They had a large canvas bag, which tied up at the mouth with strings: into this they slipped the guinea-pig, head first, and then sat upon it.)

‘I’m glad I’ve seen that done,’ thought Alice. ‘I’ve so often read in the newspapers, at the end of trials, “There was some attempts at applause, which was immediately suppressed by the officers of the court,” and I never understood what it meant till now.’

‘If that’s all you know about it, you may stand down,’ continued the King.

‘I can’t go no lower,’ said the Hatter: ‘I’m on the floor, as it is.’

‘Then you may SIT down,’ the King replied.

Here the other guinea-pig cheered, and was suppressed.

‘Come, that finished the guinea-pigs!’ thought Alice. ‘Now we shall get on better.’

‘I’d rather finish my tea,’ said the Hatter, with an anxious look at the Queen, who was reading the list of singers.

‘You may go,’ said the King, and the Hatter hurriedly left the court, without even waiting to put his shoes on.

‘—and just take his head off outside,’ the Queen added to one of the officers: but the Hatter was out of sight before the officer could get to the door.

‘Call the next witness!’ said the King.

The next witness was the Duchess’s cook. She carried the pepper-box in her hand, and Alice guessed who it was, even before she got into the court, by the way the people near the door began sneezing all at once.

‘Give your evidence,’ said the King.

‘Shan’t,’ said the cook.

The King looked anxiously at the White Rabbit, who said in a low voice, ‘Your Majesty must cross-examine THIS witness.’

‘Well, if I must, I must,’ the King said, with a melancholy air, and, after folding his arms and frowning at the cook till his eyes were nearly out of sight, he said in a deep voice, ‘What are tarts made of?’

‘Pepper, mostly,’ said the cook.

‘Treacle,’ said a sleepy voice behind her.

‘Collar that Dormouse,’ the Queen shrieked out. ‘Behead that Dormouse! Turn that Dormouse out of court! Suppress him! Pinch him! Off with his whiskers!’

For some minutes the whole court was in confusion, getting the Dormouse turned out, and, by the time they had settled down again, the cook had disappeared.

‘Never mind!’ said the King, with an air of great relief. ‘Call the next witness.’ And he added in an undertone to the Queen, ‘Really, my dear, YOU must cross-examine the next witness. It quite makes my forehead ache!’

Alice watched the White Rabbit as he fumbled over the list, feeling very curious to see what the next witness would be like, ‘—for they haven’t got much evidence YET,’ she said to herself. Imagine her surprise, when the White Rabbit read out, at the top of his shrill little voice, the name ‘Alice!’




\section{Alice’s Evidence}


‘Here!’ cried Alice, quite forgetting in the flurry of the moment how large she had grown in the last few minutes, and she jumped up in such a hurry that she tipped over the jury-box with the edge of her skirt, upsetting all the jurymen on to the heads of the crowd below, and there they lay sprawling about, reminding her very much of a globe of goldfish she had accidentally upset the week before.

‘Oh, I BEG your pardon!’ she exclaimed in a tone of great dismay, and began picking them up again as quickly as she could, for the accident of the goldfish kept running in her head, and she had a vague sort of idea that they must be collected at once and put back into the jury-box, or they would die.

‘The trial cannot proceed,’ said the King in a very grave voice, ‘until all the jurymen are back in their proper places—ALL,’ he repeated with great emphasis, looking hard at Alice as he said do.

Alice looked at the jury-box, and saw that, in her haste, she had put the Lizard in head downwards, and the poor little thing was waving its tail about in a melancholy way, being quite unable to move. She soon got it out again, and put it right; ‘not that it signifies much,’ she said to herself; ‘I should think it would be QUITE as much use in the trial one way up as the other.’

As soon as the jury had a little recovered from the shock of being upset, and their slates and pencils had been found and handed back to them, they set to work very diligently to write out a history of the accident, all except the Lizard, who seemed too much overcome to do anything but sit with its mouth open, gazing up into the roof of the court.

‘What do you know about this business?’ the King said to Alice.

‘Nothing,’ said Alice.

‘Nothing WHATEVER?’ persisted the King.

‘Nothing whatever,’ said Alice.

‘That’s very important,’ the King said, turning to the jury. They were just beginning to write this down on their slates, when the White Rabbit interrupted: ‘UNimportant, your Majesty means, of course,’ he said in a very respectful tone, but frowning and making faces at him as he spoke.

‘UNimportant, of course, I meant,’ the King hastily said, and went on to himself in an undertone,

‘important—unimportant—unimportant—important—’ as if he were trying which word sounded best.

Some of the jury wrote it down ‘important,’ and some ‘unimportant.’ Alice could see this, as she was near enough to look over their slates; ‘but it doesn’t matter a bit,’ she thought to herself.

At this moment the King, who had been for some time busily writing in his note-book, cackled out ‘Silence!’ and read out from his book, ‘Rule Forty-two. ALL PERSONS MORE THAN A MILE HIGH TO LEAVE THE COURT.’

Everybody looked at Alice.

‘I’M not a mile high,’ said Alice.

‘You are,’ said the King.

‘Nearly two miles high,’ added the Queen.

‘Well, I shan’t go, at any rate,’ said Alice: ‘besides, that’s not a regular rule: you invented it just now.’

‘It’s the oldest rule in the book,’ said the King.

‘Then it ought to be Number One,’ said Alice.

The King turned pale, and shut his note-book hastily. ‘Consider your verdict,’ he said to the jury, in a low, trembling voice.

‘There’s more evidence to come yet, please your Majesty,’ said the White Rabbit, jumping up in a great hurry; ‘this paper has just been picked up.’

‘What’s in it?’ said the Queen.

‘I haven’t opened it yet,’ said the White Rabbit, ‘but it seems to be a letter, written by the prisoner to—to somebody.’

‘It must have been that,’ said the King, ‘unless it was written to nobody, which isn’t usual, you know.’

‘Who is it directed to?’ said one of the jurymen.

‘It isn’t directed at all,’ said the White Rabbit; ‘in fact, there’s nothing written on the OUTSIDE.’ He unfolded the paper as he spoke, and added ‘It isn’t a letter, after all: it’s a set of verses.’

‘Are they in the prisoner’s handwriting?’ asked another of the jurymen.

‘No, they’re not,’ said the White Rabbit, ‘and that’s the queerest thing about it.’ (The jury all looked puzzled.)

‘He must have imitated somebody else’s hand,’ said the King. (The jury all brightened up again.)

‘Please your Majesty,’ said the Knave, ‘I didn’t write it, and they can’t prove I did: there’s no name signed at the end.’

‘If you didn’t sign it,’ said the King, ‘that only makes the matter worse. You MUST have meant some mischief, or else you’d have signed your name like an honest man.’

There was a general clapping of hands at this: it was the first really clever thing the King had said that day.

‘That PROVES his guilt,’ said the Queen.

‘It proves nothing of the sort!’ said Alice. ‘Why, you don’t even know what they’re about!’

‘Read them,’ said the King.

The White Rabbit put on his spectacles. ‘Where shall I begin, please your Majesty?’ he asked.

‘Begin at the beginning,’ the King said gravely, ‘and go on till you come to the end: then stop.’

These were the verses the White Rabbit read:—\\
\par

{\setlength{\parskip}{0em}
\par\quad ‘They told me you had been to her,
\par\quad\quad And mentioned me to him:
\par\quad She gave me a good character,
\par\quad\quad But said I could not swim.
}\\
\par

{\setlength{\parskip}{0em}
\par\quad He sent them word I had not gone
\par\quad\quad (We know it to be true):
\par\quad If she should push the matter on,
\par\quad\quad What would become of you?
}\\
\par

{\setlength{\parskip}{0em}
\par\quad I gave her one, they gave him two,
\par\quad\quad You gave us three or more;
\par\quad They all returned from him to you,
\par\quad\quad Though they were mine before.
}\\
\par

{\setlength{\parskip}{0em}
\par\quad If I or she should chance to be
\par\quad\quad Involved in this affair,
\par\quad He trusts to you to set them free,
\par\quad\quad Exactly as we were.
}\\
\par

{\setlength{\parskip}{0em}
\par\quad My notion was that you had been
\par\quad\quad (Before she had this fit)
\par\quad An obstacle that came between
\par\quad\quad Him, and ourselves, and it.
}\\
\par

{\setlength{\parskip}{0em}
\par\quad Don’t let him know she liked them best,
\par\quad\quad For this must ever be
\par\quad A secret, kept from all the rest,
\par\quad\quad Between yourself and me.’
}\\
\par

‘That’s the most important piece of evidence we’ve heard yet,’ said the King, rubbing his hands; ‘so now let the jury—’

‘If any one of them can explain it,’ said Alice, (she had grown so large in the last few minutes that she wasn’t a bit afraid of interrupting him,) ‘I’ll give him sixpence. \textit{I} don’t believe there’s an atom of meaning in it.’

The jury all wrote down on their slates, ‘SHE doesn’t believe there’s an atom of meaning in it,’ but none of them attempted to explain the paper.

‘If there’s no meaning in it,’ said the King, ‘that saves a world of trouble, you know, as we needn’t try to find any. And yet I don’t know,’ he went on, spreading out the verses on his knee, and looking at them with one eye; ‘I seem to see some meaning in them, after all. ”—SAID I COULD NOT SWIM—“ you can’t swim, can you?’ he added, turning to the Knave.

The Knave shook his head sadly. ‘Do I look like it?’ he said. (Which he certainly did NOT, being made entirely of cardboard.)

‘All right, so far,’ said the King, and he went on muttering over the verses to himself: ‘“WE KNOW IT TO BE TRUE—“ that’s the jury, of course—“I GAVE HER ONE, THEY GAVE HIM TWO—“ why, that must be what he did with the tarts, you know—’

‘But, it goes on “THEY ALL RETURNED FROM HIM TO YOU,”’ said Alice.

‘Why, there they are!’ said the King triumphantly, pointing to the tarts on the table. ‘Nothing can be clearer than THAT. Then again—“BEFORE SHE HAD THIS FIT—“ you never had fits, my dear, I think?’ he said to the Queen.

‘Never!’ said the Queen furiously, throwing an inkstand at the Lizard as she spoke. (The unfortunate little Bill had left off writing on his slate with one finger, as he found it made no mark; but he now hastily began again, using the ink, that was trickling down his face, as long as it lasted.)

‘Then the words don’t FIT you,’ said the King, looking round the court with a smile. There was a dead silence.

‘It’s a pun!’ the King added in an offended tone, and everybody laughed, ‘Let the jury consider their verdict,’ the King said, for about the twentieth time that day.

‘No, no!’ said the Queen. ‘Sentence first—verdict afterwards.’

‘Stuff and nonsense!’ said Alice loudly. ‘The idea of having the sentence first!’

‘Hold your tongue!’ said the Queen, turning purple.

‘I won’t!’ said Alice.

‘Off with her head!’ the Queen shouted at the top of her voice. Nobody moved.

‘Who cares for you?’ said Alice, (she had grown to her full size by this time.) ‘You’re nothing but a pack of cards!’

At this the whole pack rose up into the air, and came flying down upon her: she gave a little scream, half of fright and half of anger, and tried to beat them off, and found herself lying on the bank, with her head in the lap of her sister, who was gently brushing away some dead leaves that had fluttered down from the trees upon her face.

‘Wake up, Alice dear!’ said her sister; ‘Why, what a long sleep you’ve had!’

‘Oh, I’ve had such a curious dream!’ said Alice, and she told her sister, as well as she could remember them, all these strange Adventures of hers that you have just been reading about; and when she had finished, her sister kissed her, and said, ‘It WAS a curious dream, dear, certainly: but now run in to your tea; it’s getting late.’ So Alice got up and ran off, thinking while she ran, as well she might, what a wonderful dream it had been.

But her sister sat still just as she left her, leaning her head on her hand, watching the setting sun, and thinking of little Alice and all her wonderful Adventures, till she too began dreaming after a fashion, and this was her dream:—

First, she dreamed of little Alice herself, and once again the tiny hands were clasped upon her knee, and the bright eager eyes were looking up into hers—she could hear the very tones of her voice, and see that queer little toss of her head to keep back the wandering hair that WOULD always get into her eyes—and still as she listened, or seemed to listen, the whole place around her became alive with the strange creatures of her little sister’s dream.

The long grass rustled at her feet as the White Rabbit hurried by—the frightened Mouse splashed his way through the neighbouring pool—she could hear the rattle of the teacups as the March Hare and his friends shared their never-ending meal, and the shrill voice of the Queen ordering off her unfortunate guests to execution—once more the pig-baby was sneezing on the Duchess’s knee, while plates and dishes crashed around it—once more the shriek of the Gryphon, the squeaking of the Lizard’s slate-pencil, and the choking of the suppressed guinea-pigs, filled the air, mixed up with the distant sobs of the miserable Mock Turtle.

So she sat on, with closed eyes, and half believed herself in Wonderland, though she knew she had but to open them again, and all would change to dull reality—the grass would be only rustling in the wind, and the pool rippling to the waving of the reeds—the rattling teacups would change to tinkling sheep-bells, and the Queen’s shrill cries to the voice of the shepherd boy—and the sneeze of the baby, the shriek of the Gryphon, and all the other queer noises, would change (she knew) to the confused clamour of the busy farm-yard—while the lowing of the cattle in the distance would take the place of the Mock Turtle’s heavy sobs.

Lastly, she pictured to herself how this same little sister of hers would, in the after-time, be herself a grown woman; and how she would keep, through all her riper years, the simple and loving heart of her childhood: and how she would gather about her other little children, and make THEIR eyes bright and eager with many a strange tale, perhaps even with the dream of Wonderland of long ago: and how she would feel with all their simple sorrows, and find a pleasure in all their simple joys, remembering her own child-life, and the happy summer days.

\quad\quad\quad\quad THE END
}  % closing bracket for \ParallelLText

%%%%%%%%%%%%%%%%%%%%%%%%%%%%%%%%%%%%%%%%%%%%%%%%%%%%%%%%%%%%%%%%%%%%%%%%%%%%%%%%%%%%%%%%%%%%%%%%%%%%%%%%%%%%%%%%%
%%%%%%%%%%%%%%%%%%%%%%%%% Slovakian version %%%%%%%%%%%%%%%%%%%%%%%%%%%%%%%%%%%%%%%%%%%%%%%%%%%%%%%%%%%%%%%%%%%%%
%%%%%%%%%%%%%%%%%%%%%%%%%%%%%%%%%%%%%%%%%%%%%%%%%%%%%%%%%%%%%%%%%%%%%%%%%%%%%%%%%%%%%%%%%%%%%%%%%%%%%%%%%%%%%%%%%

\ParallelRText{  % closing bracket is at the end of the document
%\setcounter{section}{0}
\selectlanguage{slovak}

\tableofcontents{}

% Na medzinárodnej výstave knižného umenia v Lipsku roku 1982 odmenili túto knihu striebornou medailou.
% Ilustrácie Dušana Kállaya získali Grand Prix na Bienále ilustrácií Bratislava 1983

% Translation © Juraj Vojtek, Viera Vojtková 1981
% Illustrations © Dušan Kállay 1981

%Zakaždým, keď je pretrhnutá nitka
%a fantázia na dne vysychá,
%ako ten v púšti volám: „Ďalej zajtra!“
%a ako on už temer nedýcham,
%no „Dnes je zajtra!“ veselo mi zvonia
%okovy, ktoré rád si ponechám.
%
%Tak pučala a rástla, zvoľna zrela
%rozprávka, ktorá vám tu ponúkam;
%po kúsku, ako pavúk pavučinu
%snoval som s nimi tento nežný klam.
%Až keď si slnko sadlo na obzor,
%smel som sa vesiel chytiť v tichu sám.
%
%Alica! Tebe príbeh do rúk vkladám.
%Čo s ním, to Ty už iste dobre vieš:
%z kvetov, čo popri našom člne kvitli,
%do pamäti si veniec uviješ;
%suchý raz bude sladko voňať diaľkou
%a pamäť nad ním zaznie ako spiež…
%
%Zrkadlo vôd je nečujné a hladké
%a popoludnie ako zo zlata.
%Priútle paže s veslami sa trápia,
%nuž, neveľká je za to odplata.
%Všetky tri sa však celkom vážne tvária,
%že samy vedú naš čln do blata.
%
%Trojica mojich krutých spoločníčok!
%V objatí tejto snivej hodiny
%rozprávku chcieť o najjemnejšom vánku,
%čo neodfúkne pierko z periny…!
%No zmôže čosi proti vášmu triu
%žalostne slabý hlas môj jediný?
%
%Prima na čele velí: „Rozprávajte,“
%priam panovačne žiada začiatok;
%Secunda čaká (jemne, ticho prosí)
%čo najviac nezmyslov a mačiatok;
%Tertia každé moje odmlčanie
%vyplní prudkou paľbou otázok.
%
%Napokon zmĺknu. Mĺkvo sprevádzajú
%do bájnych krajín plných zázrakov
%to dieťa mojich predstáv, čo sa stretá
%so spoločnosťou zvierat storakou,
%s kvetmi i vtákmi… Dobre vedia: rojčí,
%a predsa vieru čítam zo zrakov.

\chapter{}
% first chapter
\section{Do nory za králikom}[
Alicu už omrzelo sedieť so sestrou len tak nečinne na brehu rieky. Niekoľko ráz nazrela do sestrinej knihy, ale neboli v nej ani obrázky, ani sa tam nič nerozprávalo; „a čo s knihou, keď tam nie sú obrázky, ani sa tam nič nerozpráva,“ pomyslela si Alica.

A tak uvažovala (pokiaľ sa to vôbec dalo, lebo od horúčavy bola ospanlivá a celkom otupená), či by stálo za to vstať, natrhať margaréty a uviť z nich veniec, keď tu zrazu prebehol popri nej biely králik s červenými očami.

Na tom ešte nebolo nič zvláštne a nezdalo sa jej nezvyčajné, ani keď začula, ako si králik vzdychol: „Ach, jaj! Ach, jaj! Už prídem neskoro!“ (Keď o tom neskoršie premýšľala, uvedomila si, že nad tým sa mala pozastaviť, no vtedy sa jej to videlo celkom prirodzené.) Ale keď si králik napokon vybral z vrecka na veste hodinky, pozrel na ne a hneď bežal ďalej, Alica vyskočila; až v tej chvíli jej blýslo umom, že vlastne ešte nikdy nevidela králika s vreckom na veste, navyše aby si z vrecka vyberal hodinky. Rozpalená od zvedavosti rozbehla sa cez pole za ním, no zazrela už len to, ako sa šuchol do veľkej králičej nory pod kríkom.

V mihu sa spustila za ním a ani len nepomyslela na to, ako sa stade dostane zasa von.

Králičia nora bola na kúsku rovná ako tunel, no odrazu sa začala prudko zvažovať, tak prudko, že sa Alica nestihla ani zastaviť, a už padala do akejsi hlbokej jamy.

Alebo bola tá jama veľmi hlboká, alebo padala veľmi pomaly, no Alica mala dosť času obzerať sa okolo seba a rozmýšľať, čo bude ďalej. Najsamprv sa pokúšala pozerať dolu a zistiť, kam to vlastne padá, ale nič nevidela, tak tam bolo tma. Nuž si obzerala steny jamy a videla na nich veľa kuchynských políc a priečinkov na knihy. Tu a tam viseli na klincoch mapy a obrazy. Ako padala, vzala z jednej police zaváraninovú fľašu s nálepkou POMARANČOVÁ MARMELÁDA; no sklamala sa, fľaša bola prázdna. Bála sa odhodiť ju, aby dolu niekoho nezabila, tak ju zasa šikovne vopchala do police, popri ktorej letela.

„Teda,“ pomyslela si Alica, „po takomto páde zletieť zo schodov bude už pre mňa úplná hračka! A zá akú hrdinku ma budú doma považovať! Čože, teraz keby som aj zo strechy spadla, ani by som nemukla!“ (To pravdepodobne naozaj nie.)

A stále dolu a dolu a dolu! Vari to padanie nemá konca? „Ktovie, koľko kilometrov som sa už prepadla?“ povedala nahlas. „Už budem hádam až dakde pri strede zemegule. Počkať — to by bolo, ak sa nemýlim, asi šesťtisíc päťsto kilometrov do hĺbky.“ (Ako vidíte, Alica sa už v škole voľačo naučila, a hoci teraz nebola práve najvhodnejšia príležitosť popýšiť sa vedomosťami, lebo ju nikto nepočul, opakovanie nikomu nezaškodí.) „Áno, asi toľko to bude – a zaujímalo by ma, na akej zemepisnej šírke a dĺžke sa nachádzam…“ (Alica nemala ani potuchy o zemepisnej šírke a zemepisnej dĺžke, ale čo, keď sú to také nádherné slová.)

A znova začala: „Čo ak sa prepadnem cez celú zemeguľu! To bude paráda, keď vyleziem medzi ľuďmi, ktorí chodia dolu hlavou! Ako sa len volajú – tuším protichodci.“ (Teraz bola celkom rada, že ju nik nepočuje, lebo to slovo sa ani jej nepozdávalo.) „No musím sa ich opýtať, ako sa tá ich krajina volá. Prosím vás, pani, je toto Nový Zéland, či Austrália?“ (Pri tých slovách sa pokúšala ukloniť — predstavte si, klaňať sa, keď človek padá! Myslíte, že by sa vám to podarilo?) „Nie, ešte by si tá pani pomyslela, čo som to za nevzdelané dievča! Nie, pýtať sa nebudem, možno to uvidím dakde napísané.“

A stále dolu a dolu a dolu. Nič iné sa nedalo robiť, tak Alica znova začala vykladať: „Tine budem dnes večer iste veľmi chýbať!“ (Tina bola mačka.) „Dúfam, že jej na olovrant nezabudnú naliať do misky mlieka. Tinuška moja! Keby si tu tak bola pri mne! Vo vzduchu asi nijaké myši nie sú, ale možno by si si tu chytila netopiera a ten vyzerá skoro ako myš. Len ktovie, či mačky žerú netopiere?“ A vtom padli na Alicu driemoty a ona si ospanlivo opakovala: „Žerú mačky netopiere? Žerú mačky netopiere?“ A potom zasa: „Žerú netopiere mačky?“ Pretože — chápete — ak si ani na jednu otázku nevedela odpovedať, bolo úplne jedno, či sa pýta tak, alebo onak. Spánok ju premáhal a práve sa jej začalo snívať, že sa s Tinou vedú za ruky a ona jej veľmi vážne hovorí: „Počuj, Tina, povedz mi pravdu: jedla si už netopiera?“ — keď tu zrazu bum-bác! dopadla na kopu suchého lístia a raždia a bolo po páde.

Alica sa ani trošku neudrela a vo chvíli bola už zasa na nohách. Pozrela hore, ale tam bola iba tma tmúca; pred sebou mala ďalšiu dlhú chodbu a v nej zazrela uháňajúceho Bieleho králika. Nesmela stratiť ani minútu, letela za ním ako vietor, a keď zahýbal za roh, počula, ako zahundral: „Sto uší a tristo fúzov, ale je už neskoro!“ Ešte keď zahýbala za roh, bola mu za pätami, no odrazu sa stratil. Ocitla sa v dlhej nízkej sieni, osvetlenej radom lámp visiacich zo stropu.

V sieni boli dookola samé dvere, no všetky zamknuté. Alica prešla sieňou po jednej strane, potom nazad po druhej a rad-radom všetky vyskúšala. Nakoniec smutne kráčala prostriedkom a rozmýšľala, ako sa dostať von.

A vtom sa ocitla pred malým trojnohým stolíkom; celý bol z hrubého skla. Bol na ňom iba zlatý kľúčik a Alici hneď blyslo umom, či nie je od niektorých dverí v sieni. Ale kdeže! Alebo boli zámky priveľké, alebo kľúčik primalý, ani jedny dvere sa nedali odomknúť. No keď sieň obchádzala po druhý raz, zbadala nízko nad zemou záves, ktorý si predtým nevšimla, a za ním boli dvierka vysoké ani nie pol metra. Skúsila zlatý kľúčik, a sláva — kľúčik vkĺzol do zámky!

Alica odomkla dvierka a zistila, že vedú do chodbičky len o niečo širšej než potkania diera. Kľakla si a na druhom konci chodbičky videla čarokrásnu záhradu. Zatúžila dostať sa z tmavej siene a prechádzať sa medzi záhonmi žiarivých kvetov a chladivými fontánami, ale cez dvierka by neprestrčila ani hlavu. „A keby som hlavou aj prešla,“ rozmýšľala chudera Alica, „čo z toho, keď plecami neprejdem! Ach, keby som sa tak mohla poskladať ako sklápací ďalekohľad! Myslím, že by to aj šlo, len keby som vedela, ako začať.“ Lebo, ako ste si všimli, Alici sa v poslednom čase prihodilo toľko nezvyčajného, že sa jej už skoro všetko zdalo možné.

Vyčkávať pri dvierkach očividne nemalo zmysel, nuž sa vrátila k stolíku, či tam azda nenájde nejaký iný kľúč alebo aspoň príručku s návodom, podľa ktorého by sa človek sklápal ako ďalekohľad. Tentoraz tam našla fľaštičku („Tá tu predtým zaručene nebola!“ povedala si Alica); na hrdle mala ceduľku a na ceduľke prekrásne vytlačené veľkými písmenami: VYPI MA!

To sa ľahko povie, „Vypi ma“, ale múdra Alica hneď na všetko nenaletí. „Nie,“ povedala si, „najprv sa pozriem, či tam nie je označenie Pozor — jed! Neraz čítala príbehy o deťoch, ktoré zhoreli alebo ich zožrala divá zver, a vôbec sa im prihodilo veľa nepríjemnosti len preto, že nedbali na základné ponaučenia svojich najbližších, tak napríklad, že človek si popáli prsty, ak pridlho drží v ruke žeravý kutáč, alebo že mu prst začne krvácať, ak si doň zareže dosť hlboko. Alica nezabúdala ani na to, že ak sa človek napije z fľašky označenej Pozor — jed!, je takmer isté, že skôr či neskôr mu to nebude na osoh.

Ale na tejto flaštičke nijakú výstrahu nenašla, tak sa odhodlala ochutnať, čo je v nej. Keď zistila, že je to veľmi dobré (malo to chuť ako čerešňový koláč, vaječný krém, ananás, pečený moriak, karamelky a maslové hrianky dokopy), raz-dva všetko vypila.

„Akosi mi je čudne!“ povedala Alica. „Asi sa už sklápam ako ten ďalekohľad!“

A naozaj: bola už iba dvadsaťpäť centimetrov vysoká a tvár sa jej rozžiarila, keď si pomyslela, že teraz má najlepšiu veľkosť na to, aby sa prepchala cez dvierka do tej utešenej záhrady. Pre istotu niekoľko minút počkala, či sa ešte väčšmi nescvrkne. Trochu ju to znepokojovalo, „lebo, chápete, čo ak pochodím ako sviečka, keď dohorí,“ pomyslela si Alica. „Ktovie, ako by som potom vyzerala?“ Skúšala si predstaviť, ako vyzerá plameň sviečky, keď sa sfúkne, ale nie a nie sa rozpamätať, či také niečo dakedy videla.

Nič sa s ňou už nedialo, nuž sa o chvíľu rozhodla, že pôjde rovno do záhrady. Ale beda-prebeda! Pri dvierkach si chuderka uvedomila, že zlatý kľúčik zabudla na stolíku, a keď sa poň vrátila, zistila, že teraz ho už nedočiahne. Cez sklo ho videla celkom zreteľne a zúfalo sa šplhala po jednej stolovej nohe, ale tá bola veľmi klzká. Nakoniec si, chúďa, celkom vyčerpaná sadla na zem a rozplakala sa.

„Tak už dosť, plačom si nepomôžeš,“ oborila sa Alica na seba. „Radím ti, aby si s tým hneď a zaraz prestala!“ Alica si zväčša dávala veľmi dobré rady (no iba zriedka sa podľa nich aj správala) a niekedy sa vyhrešila tak prísne, až jej slzy vyhŕkli. Raz sa dokonca skoro vyzauškovala za švindľovanie pri krokete, keď hrala sama so sebou. Toto zvláštne dieťa si totiž veľmi rado predstavovalo, že je v dvoch osobách. „V dvoch osobách…“ pomyslela si nešťastná Alica, „teraz to už nemá zmysel! Veď zo mňa neostalo ani na jednu poriadnu osobu!“

Vtom zazrela pod stolom sklenú škatuľku, otvorila ju a našla v nej koláčik a na ňom krásny nápis z hrozienok: ZJEDZ MA! „Aj ho zjem,“ povedala Alica, „a ak od neho narastiem, dočiahnem kľúčik; a ak sa ešte scvrknem, podleziem popod dvere; nedbám, čo sa so mnou stane, lebo do záhrady sa tak či onak nejako dostanem!“

Odhryzla si z koláča a s úzkosťou si vravela: „Tak ako? Hore, či dolu?“ Ruku si položila na vrch hlavy, aby zistila, či rastie, alebo sa zmenšuje, a veľmi ju prekvapilo, že sa vôbec nemení. Isteže, tak to býva, keď človek zje koláč; no Alica si už tak zvykla na samé nevídané a neslýchané veci, že obyčajnýýživotsajej videl nudnýa hlúpy;

Pustila sa teda do koláča a čoskoro bolo po ňom.

% second chapter
\section{Mláka sĺz}

„Div najdivovejší! Div najdivovejší!“ zhíkla Alica. (Od prekvapenia zabudla aj správne hovoriť.) „Veď ja sa naťahujem ako ten najdlhší ďalekohľad na svete. Zbohom, nožičky moje!“ (Lebo keď si pozrela na nohy, zdalo sa jej, že ich už- už stratí z dohľadu, tak veľmi sa vzďaľovali.) „Ach, moje úbohé nožičky, kto vás teraz obuje a kto na vás natiahne pančuchy? Ja to už asi nedokážem! Budem trocha priďaleko na to, aby som sa mohla o vás starať. Musíte si poradiť, ako samy najlepšie viete.“ „Ale musím byť k nim dobrá,“ pomyslela si Alica, „ešte si zmyslia chodiť inou cestou, ako ja budem chcieť! Uvidíme. Na každé Vianoce im darujem nový pár topánok.“

A už vymýšľala, ako to zariadi. „Budem ich musieť poslať poštou,“ vravela si, „to bude psina, posielať dary vlastným nohám! A ako čudne bude vyzerať adresa!?)

\quad Vážená
\quad Pravá noha Alicina
\quad ROHOŽKA pred Kozubom
\quad PRI OCHRANNEJ MRIEŽKE
\quad (S pozdravom Alica)

Prepánajána, čo to táram!“

V tej chvíli narazila hlavou do povaly: mala vám teraz vyše dva aj trištvrte metra; hneď vzala zlatý kľúčik a hybaj do záhrady.

Chúďa Alica! Mohla si nanajvýš ľahnúť na bok a jedným okom do nej hľadieť, no dostať sa dnu bolo ešte beznádejnejšie ako predtým. Sadla si a znova sa rozplakala.

„Hanbi sa,“ povedala si Alica, „také veľké dievča,“ (to teda bola) „a takto revať ! Vravím ti, prestaň!“ Ale plakala ďalej a vyronila za celé vedrá sĺz, takže naplakala okolo seba velikánsku, zo desať centimetrov hlbokú mláku, čo sa rozlievala do polovice siene.

Po chvíli začula v diaľke drobné cupkavé krôčky; rýchlo si poutierala oči a pozrela, kto to ide. Vracal sa Biely králik, parádne vyobliekaný, v jednej ruke biele rukavice z pravej kozinky, v druhej veľký vejár. Náhlivo drobčil popri nej a šomral si popod nos: „Ach, tá Vojvodkyňa! Ach, tá Vojvodkyňa! Tá bude zúriť, že idem tak neskoro!“ Alica bola už taká zúfalá, že by bola požiadala o pomoc hocikoho; a tak keď sa k nej Králik priblížil, ozvala sa tichým, bojazlivým hláskom: „Boli by ste taký láskavý, pane…“ Králik sa strhol, biele rukavice i vejár mu vypadli z rúk a odtrielil do tmy tak rýchlo, ako len vládal.

Alica zodvihla vejár a rukavice, a pretože v sieni bolo veľmi teplo, začala sa ovievať. „Bože, bože!“ vzdychla si. „Aké je dnes všetko čudné! A ešte včera to bolo ako inokedy. Vari som sa v noci nejako zmenila? Porozmýšľajme: Keď som dnes ráno vstávala, bola som tá istá ako včera? Veru sa mi zdá, že som sa cítila akosi inakšie. Ale ak už nie som tá čo včera, otázka je: Kto vlastne som? To je tá záhada!“ A prebrala rad-radom všetky rovesníčky, či sa nebodaj nepremenila na niektorú z nich.

„Určite nie som Ada,“ povedala, „lebo tá má dlhé kučeravé vlasy, a moje sú rovné; nemôžem byť ani Mabel, lebo ja viem toľko vecí, a ona, ach, ona nevie skoro nič! A konečne, ona je ona a j a som ja, ach, môj ty svete, aké je to všetko pomotané! Zistím, či ešte viem všetko, čo som vedela. Povedzme: štyrikrát päť je dvanásť a štyrikrát šesť je trinásť a štyrikrát sedem je — ach, jaj, takto sa po dvadsiatku nikdy nedostanem! Ale násobilka nie je taká dôležitá, skúsim to so zemepisom. Londýn je hlavné mesto Paríža, Paríž je hlavné mesto Ríma a Rím — nie, tak to nie je, určite nie. Asi som sa predsa len zmenila na Mabel. Ešte skúsim nejakú básničku, povedzme \textit{Hľaďte na tie čudné zvyky}… Zložila si ruky, ako keď ju vyvolali v škole, a začala
recitovať. Ale hlas mala chrapľavý a cudzí, ba ani slová
nezneli tak ako inokedy:\\

\par
{\setlength{\parskip}{0em}
\par\textit{\quad Hľaďte na tie čudné zvyky:}
\par\textit{\quad karty mastia i paniky!}
\par\textit{\quad Sedia v klietke do polnoci,}
\par\textit{\quad hrajú, hrajú, niet pomoci.}
}\\
\par

{\setlength{\parskip}{0em}
\par\textit{\quad Sem sa, ľudia, pomáhajte,}
\par\textit{\quad karty z zoo odnášajte!}
\par\textit{\quad Na čo mladé, len sa smialo,}
\par\textit{\quad a čo staré, zutekala.}
}\\
\par

„Nie, to nie je dobre,“ povedala nešťastná Alica a oči sa
jej znova zaliali slzami: „Už je to tak, určite som Mabel
a teraz sa musím odsťahovať do toho ich tesného domca
a rozlúčiť sa so všetkými hračkami, a toho učenia, no hrôza!
Nie, už to mám: Ak som Mabel, tak radšej ostanem tu!
Môžu sem pchať hlavy a vyvolávať: ,Poď von, dušička!“ Iba
zodvihnem hlavu a poviem: ,Tak kto som? Najprv mi to
povedzte, a ak sa mi bude chcieť byť tým človekom, vyjdem;
ak nie, zostanem tu, kým sa nestanem niekým iným.“ Ale
- jój!“ rozplakala sa Alica a slzy jej vyhŕkli prúdom. „Len
keby sem dakto vopchal hlavu! Táto samota ma už celkom
umorila!“

Ako to povedala, pozrela si na ruky a prekvapene zistila,
že si medzi rečou natiahla na ruku Králikovu malú rukavičku
z kozinky. „Ako je to možné?“ rozmýšľala. „Iste sa zasa
zmenšujem.“ Vstala a podišla k stolíku, „aby sa k nemu
primerala, a len tak od oka odhadla, že už má znova len asi
sedemdesiat centimetrov a očividne sa zmenšuje ď alej.
Prišla na to, že je to od vejára, ktorý ešte stále držala v ruke.
Rýchlo ho odhodila, a bol veru najvyšší čas, inak by sa bola
v tej chvíli scvrkla načisto.

„Ale veľa mi už nechýbalo,“ povedala si Alica, poriadne

20

Alica v krajine zázrakov

vystrašená tou náhlou zmenou a zároveň šťastná, že ešte
stále jestvuje. „A teraz do záhrady!“ A rozbehla sa
k dvierkam, no beda-prebeda! Dvierka boli znova zavreté
a zlatý kľúčik ležal na sklenom stole ako predtým.

„Čím dalej horšie,“ pomyslela si chuderka, „lebo ešte
nikdy som nebola taká malá ako teraz. Nikdy! A to je zle,
veľmi zle!“

Ako to povedala, nohy sa jej pošmykli a čľup! — bola až
po bradu v slanej vode. V prvej chvíli si pomyslela, že padla
hádam do mora. „Tak, sa teda vrátim domov vlakom,“
povedala si. Bola totiž pri mori iba jeden jediný raz a celkom
vážne si myslela, že nech sa človek ocitne kdekoľvek na
anglickom pobreží, nájde tam množstvo kabín, deti, ktoré sa
hrajú v piesku s drevenými lopatkami, potom rad penziónov
a za nimi železničnú stanicuĺ Ale čoskoro prišla na to, že je
v kaluží sĺz, ktoré vyplakala, keď mala dva a trištvrte
metra.

„Nemala som toľko plakať,“ vyčítala si Alica, ako plávala
a usilovala sa dostať z kaluže. „Za trest sa teraz utopím vo
vlastných slzách! To je teda čudné, len čo je pravda! Ale
dnes je všetko čudné!“

A tu nedaleko nej v mláke čosi začľapotalo. Plávala ta
pozrieť, čo to je. Sprvu to vyzeralo ako nosorožec alebo
hroch, ale potom si uvedomila, aká je teraz malá, a zistila,
že je to iba myš, ktorá sa pošmykla takisto ako ona.

„Čo keby som sa jej prihovorila?“ rozmýšľala Alica. „Tu
je všetko také nezvyčajné, ktovie, možno tá myš aj rozpráva.
Tak či onak, nič sa nestane, keď to skúsim.“ A tak spustila:
„ó, Myš, ako by som sa dostala z tejto mláky? Už sOm od
plávania celá zmorená. ó, Myš!“ (Alica si myslela, že
s myšou sa treba takto rozprávať. Nikdy sa ešte s nijakou
nezhovárala, ale spomenula si, že v bratovej latinskej
gramatike videla napísané: „myš — od myši —- k myši —- myš
- ó, myš!“) Myš si ju skúmavo premerala, ba Alici sa
zazdalo, že jedným okom aj zažmurkala, no nepovedala
mc.

22

Mláka sĺz

„Možno nerozumie po anglicky,“ pomyslela si Alica.
„Bude to podistým francúzska myš, ktorá k nám prišla
s Viliamom Dobyvateľom.“ (Aj keď z dejepisu všeličo
vedela, nemala ani potuchy o tom, že tento vojvodca prišiel
do Anglicka v jedenástom storočí.) A tak začala inakšie:
„Oů est ma chatte?“ prvou vetou zo svojej učebnice
francúzštiny. V tej chvíli vyskočila Myš z vody a ani čo by sa
celá triasla od strachu. „Ach, prepáč mi,“ vyhŕkla Alica
vyľakaná, že sa dotkla citov úbohého zvieratka. „Celkom
som zabudla, že ty nemáš rada mačky.“

„Že nemám rada mačky!“ prenikavým a zlostným hlasom
zapišťala Myš. „Ty by si ich na mojom mieste mala rada?“

„Asi nie,“ tíšila ju Alica. „A už sa nehnevaj. No aj tak by
som ti rada ukázala našu mačku Tinu. Myslím, že by si si
mačky obľúbila, keby si ju čo len raz videla. Je to také milé,
tiché zvieratko,“ pokračovala Alica viac-menej pre seba
a pomaličky plávala v mláke. „Sedáva pri kozube, pekne
pradie, olizuje si labky a umýva si tváričku… a má taký
mäkučký kožuštek, čo sa dobre hladká. .. a v chytaní myší
nemá páru. .. Ach, prepáč!“ znova vykríkla Alica, lebo Myš
sa už celá zježila a Alica si uvedomila, že ju iste hlboko

urazila. „Ak ti je to nepríjemné, nebudeme už o nej   i '

hovoriť.“

„Ako to: nebudeme!“ skríkla Myš a triasla sa od hlavy až
po konček chvosta. „Akoby som sa o tom chcela rozprávať
práve ja! Naša rodina mačky odjakživa nenávidela. Tie
odporné, prízemné, vulgárne živočíchy! Nechcem už o nich
ani počuť!“

„V poriadku!“ súhlasila Alica a chytro zvrtla reč na iné.
„Máš rada… máš rada psov?“ Myš neodpovedala, a tak
Alica náhlivo pokračovala: „Blízko nás majú pekného
psíka, toho by som ti chcela ukázať. Vieš, je to foxteriér
s takými bystrýrni očkami a s hnedou dlhou a brčkavou
srsťou! Keď mu hodíš nejakú vec, on ti ju prinesie nazad, vie
si vyprosiť jedlo a vie veľa všelijakých kúskov - ani na
polovicu si teraz nespomeniem — patrí jednému gazdovi

23

 

 

Alica v krajine zázrakov

a ten vraví, že je taký užitočný, že by ho nedal ani za sto
libier! Doteraz mu vraj neušiel ani jeden potkan a — ach,
prepánajána!“ skríkla Alica ľútostivo. „Tuším som ju zase
urazila!“ Myš od nej plávala preč tak rýchlo, ako len vládala,
takže rozvírila celú mláku.

Alica za ňou vľúdne zavolala: „Milá Myška! Vráť sa,
prosím ťa, a už sa nebudeme zhovárať ani o mačkách, ani
o psoch, keď ich nemáš rada.“

Keď to Myš počula, obrátila sa a pomaly plávala nazad.
Tvár mala celkom bielu („To od zlosti,“ pomyslela si Alica)
a tichým, trasúcim sa hlasom povedala: „Poďme na breh,
tam ti porozprávam svoj príbeh, aby si pochopila, prečo
nenávidím mačky a psov.“

Bol najvyšší čas, lebo mláka sa už hemžila vtákmi
a zvieratami, ktoré do nej popadali. Bola tam Kačica
a Ohnivák Dodo, papagáj Lory a Orlík a iné čudesné
stvorenia. Alica plávala na čele a celá perepúť mierila
k brehu.

% third chapter
\section{Ablegačné preteky a príbeh s dlhočizným koncom}

2 IŠLA sa tam na brehu naozaj čudná spoločnosť

— vtáci s mokrým, špinavým perím, zvieratá
so zablatenou srsťou, a všetci premoknutí, podráždení
a mrzutí.

Ich prvou starosťou bolo, prirodzene, ako sa usušiť: ra-
dili sa a o chvíľku sa už Alica s nimi zhovárala nenútene
a kamarátsky, akoby ich poznala od narodenia. S papagá—
jom Lorym sa dohadovala tak dlho, kým jej namrzene
neodsekol: „Som starší ako ty, tak to musím vedieť lepšie.“
Alica mu to nechcela uznať. Chcela vedieť, koľko má Lory
rokov. No papagáj tvrdohlavo odmietal prezradiť jej svoj
vek. A tak sa nedalo nič robiť.

Napokon Myš, ktorá sa podľa všetkého tešila všeobecnej
vážnosti, zvolala: „Sadnite si a počúvajte, čo vám poviem!
Ja vás raz—dva usuším!“ Všetci si hneď posadali do veľkého
kruhu. Myš ostala v strede. Alica hľadela na ňu s obavou,
lebo si uvedomovala, že ak chytro neuschne, poriadne
prechladne.

„Hm, hm!“ odkašlala si Myš významne. „Dávate všetci
pozor? Toto je najsuchšia vec, akú poznám. Ticho, prosím!
Angličanov, ktorým chýbali vodcovia a v poslednom čase
iba lúpežili a zbíjali. Edwin a Morcar, grófovia z Mercie
a Northumbrie —“

„Bŕŕŕ!“ otriasol sa Lory.

„Prosím?“ spýtala sa Myš zamračene, ale zdvorilo.
„Povedal si niečo?“

„Ja? Ale nie!“ chytro odpovedal Lory.

„Zdalo sa mi, že hej,“ na to Myš. „Tak pokračujem.

25

Alica v krajine zázrakov

Edwin a Morcar, grófovia z Mercie a Northumbrie, sa pridali
k nemu, a dokonca Stigand, vlastenecký arcibiskup canter-
burský, si v tom našiel dôvod —“

„Našiel si? Kde?“ spýtala sa Kačica.

„No predsa v tom,“ odpovedala Myš namrzene, „vari
nevieš, kde si možno nájsť?“

„Viem to veľmi dobre,“ odpovedala Kačica. „Keď ja
hľadám žaby alebo červíky, viem, kde si ich nájdem. Otázka
je, kde si ich našiel ten váš arcibiskup?“

Myš si otázku ani nevšimla a náhlivo pokračovala:
„— našiel si v tom dôvod, aby sa vybral s Edgarom Athe-
lingom Viliamovi v ústrety a ponúkol mu korunu. Viliam sa
spočiatku správal mierumilovne. No bezočivosť jeho Nor-
manov — ako sa cítiš, moja milá?“ obrátila sa zrazu k Alici.

„Taká mokrá ako predtým,“ smutno odpovedala Alica.
„Tie tvoje reči ma akosi nesušia.“

„V tom prípade,“ vyhlásil Dodo slávnostne a vstal,
„navrhujem odložiť schôdzku za účelom neodkladného
akceptovania účinnejších procedúr.“

„Hovor zrozumiteľne!“ povedal Orlík. „Ani polovici tých
dlhočizných slov nerozumiem a neverím, že im rozumieš ty
sám!“ A Orlík sklonil hlavu, aby skryl úsmev. Ale podaktorí
vtáci sa celkom nahlas zachichotali.

„Chcel som povedať iba toľko,“ bránil sa Dodo urazene,
„že najrýchlejšie sa usušíme, ak usporiadame ablegačné
preteky.“

„A to je zas čo, tie ablegačné preteky?“ spýtala sa Alica;
nie že by až tak veľmi túžila dozvedieť sa to, ale Dodo sa
odmlčal, akoby čakal, že niekto niečo povie. Nikomu sa
však akosi nechcelo.

„No,“ povedal Dodo, „najnázornejšie si to vysvetlíme,
ak tie preteky usporiadame.“ (Ak by ste si to niekedy aj
vy skúsili, porozprávam vám, ako to Dodo zorganizoval.)

Najsamprv vyznačil na zemi pretekársku dráhu, bol to
približne kruh („Na presnom tvare nezáleží,“ povedal)
a potom všetku tú perepúť rozostavil dookola na pretekár-

26

 

Alica v krajine zázrakov

sku dráhu. Nijaké „Pripraviť sa! Pozor! Už!“, každý si
vybehol, kedy chcel, i dobehol, kedy chcel, takže sa ťažko
dalo povedať; kedy sa vlastne preteky skončili. No keď už
behali asi pol druha hodiny a všetci boli suchí, DOdo zrazu
zmal: „Koniec pretekov!“ a všetci ho obStaĺi ä Zadychčane
sa vypytovali: „Kto vyhral?“

Kým im Dodo odpovedal, hlboko sa zamyslel a hodnú
chvíľu stál s prstom pritlačeným na čelo (asi tak ako na
obrázkoch Shakespeare) a ostatní ticho čakali. Konečne
Dodo vyhlásil: „Vyhrali všetci a každý dostane cenu.“

„Ale kto tie ceny odovzdá?“ ozvali sa zborove.

„Akože kto? No predsa ona,“ povedal Dodo a prstom
ukázal na Alicu. Všetci ju obstali a jeden cez druhého
vykrikovali: „Ceny! Ceny!“

Alica raz nevedela, čo robiť; v bezradnosti siahla do
vrecka, vytiahla škatuľku ovocných cukríkov (slaná voda ju
našťastie nepremočila) a rad-radom ich rozdávala ako ceny.
Každému sa ušiel jeden.

„Ale aj jej musíme dať cenu, nie?“ povedala Myš.

„Samozrejme,“ odpovedal Dodo smrteľne vážne. „Čo
máš ešte vo vrecku?“ obrátil sa na Alicu.

„Už iba náprstok,“ povedala Alica smutne.

„Sem s ním!“ na to Dodo.

Znova sa zhrčili okolo nej a Dodo jej sláVDostne odo—
vzdal náprstok so slovami: „Ráč prijať tento elegantný
náprstok.“ A keď ten krátky prejav dopovedal, všetci
skríklj sláva.

Alici sa celá záležitosť videla nezmyselná, ale všetci okolo
nej boli takí vážni, že si netrúfala zasmiať sa. Nič vhodné jej
neprišlo na um, tak sa iba uklonila a celkom vážne náprstok
prevzala.

Potom prišli na rad cukríky. Nezaobišlo sa to bez kriku
a zmätku. Veľkí vtáci sa sťažovali, že sú malé a že ich nestihli
ani ochutnať, malí sa zasa dusili a museli im búchať po
chrbte. Ked bolo po cukríkoch, znova si posadali do kruhu
a prosili Myš, aby im ešte niečo rozprávala.

28

Ablegačné preteky a príbeh s dlhočizným koncom

„Sľúbila si mi, že porozprávaš niečo zo svojho života,“
povedala Alica, „a prečo nemáš rada M a P,“ dodala
pošepky, lebo sa bála, že ju zasa urazí.

„Je to príbeh o myši s dlhočizným a smutným koncom,“
povedala Myš Alici a vzdychla si.

Alica nepočula dobre, čo to Myš vraví, a začudovane sa
zahľadela na jej chvost. „Naozaj je dlhočizný,“ pripustila,
„ale prečo vravíš, že je smutný?“ A rozmýšľala nad tým po
celý čas, čo Myš rozprávala, takže sa jej príbeh začal meniť
pred očami na takýto zvlnený chvost:

 

Zlostný pes
stretol myš.
Riekol jej:
„Len sa smej,
ešte dnes
dopištíš.
Na súd bež,
čosi zvieš.
H ybaj už!
Alenskús   'řl ,.
zdvihnúť hlas! ĺ . \
Dnes mám čas,

. / ;
nuž ti tam
osud sám
prichystám. “
Riekla myš

psisku zas:
„Nemám chuť
na ten špás.
O ten súd
málo dbám;
mrha'š čas!“
„Ja ti dám! “
n'ekol pes
Zúrivec.
„ Súdim sám!
Ešte dnes
celú vec
právo mám
rozhodnúť.
Žiadny klam!
Prisaha'm:
Čakaj smrť!“ =

   

 

 

 

Alica v krajine zázrakov

„Ved ty ma nepočúvaš!“ hrešila Myš Alicu. „Na čo
myslíš?“

„Prepáč,“ povedala Alica skrúšene. „Ak sa nemýlim, máš
už za sebou štyri zákruty.“

„Tak krutý!“ ešte väčšmi sa nazlostila Myš.

„Vravíš — krutý!?“ ustarostene sa obzerala okolo seba
Alica. „Ach, veď povedz, nedovolím, aby ti ubližoval!“

„O nič ťa neprosím!“ odsekla Myš, vstala a šla preč.
„Urážaš ma tými nezmyselnými rečami!“

„Ja som to tak nemyslela,“ bránila sa chudera Alica.
„Prečo si taká urážlivá?“

Myš namiesto odpovede len čosi zahundrala.

„Prosím ťa, vráť sa a dorozprávaj nám to!“ volala za ňou
Alica. A ostatní za ňou zborove: „Áno, áno, prosíme!“ Ale
Myš iba zlostne pokrútila hlavou a pridala do kroku.

„Škoda, že tu neostala!“ vzdychol si Lory, keď sa im
stratila z očí. A stará Krabová to využila a povedala dcére:
„Dieťa moje, nech ti je to poučením, že sa nikdy netreba dať
vyviesť z rovnováhy.“ „Mlč, mama!“ odsekla jej mladá. „Ty
by si dožrala ajýustricu.“

„Keby som tu mala Tinu, veď by som si ja poradila,“
povedala Alica nahlas, ale iba pre seba. „Tá by mi ju
raz-dva dovliekla nazad.“

„A kto je to Tina, ak sa smiem spýtať?“ zaujímal sa
Lory.

Alica odpovedala ochotne, lebo o svojom rniláčikovi vždy
rada rozprávala: „Tina je naša mačka. Tá vám chytá myši,
jedna radosť! A keby ste ju videli, ako striehne na vtákov!
Len vám na vtáčka pozrie, a už je po ňom!“

Tieto slová vyvolali medzi prítomnými hotový rozruch.
Niektorí vtáci sa odrazu zodvihli a odišli. Jedna stará Straka
sa dôkladne zababušila do vlniaka a utrúsila: „J a už musím
ísť. Večerný vzduch mi škodí na hrdlo.“ A Kanán'k ro-
zochveným hlasom zvolával deti: „Poďte, deťúrence moje!
Už aby ste boli v posteli!“ Tak sa pod rozličnými zámien—
kami jeden za druhým povytrácali a Alica ostala sama.

30

Ablegačné preteky a príbeh s dlhočizným koncom

„Nemala som tú Tinu spomínať!“ povedala si smutne.
„Tu dolu ju akosi nik nemá rád, hoci je to tá najlepšia mačka
na svete. Ach, moja drahá Tina! Či ťa ja ešte dakedy
uvidím!“ A v tej chvíli sa nešťastná Alica od veľkého žiaľu
a clivoty znova rozplakala. Ale o chvíľku začula z diaľky
cupkanie. Dychtivo zodvihla hlavu, či si to Myš predsa len
nerozmyslela a nejde jej dopovedať svoj príbeh.

% fourth chapter
\section{Ako Vilo vyletel z komína}

AE to k nej pomaličky cupkal Biely králik a vystraše-
ne sa obzeral, akoby niečo stratil. Počula, ako si
šomre popod fúzy: „Tá Vojvodkyňa! Tá Vojvodkyňa! Ach,
labôčky moje! Ach, kožuštek môj, fúziky moje! Tá ma dá
iste popraviť, ako že dvakrát dva je štyri! Keby som len
vedel, kde som ich pohodil!“ Alica hneď uhádla, že Králik
hľadá vejár a biele kozinkové rukavice, tak ich bez dlhých
rečí začala hľadať, ale nikde ich nevidela. Po tom kúpeli
v kaluží akoby sa všetko zmenilo a veľká sieň so skleným
stolíkom a dvierkami načisto zmizla.

Králik si všimol, ako tam Alica pobehuje, a hned ju
okríkol: „Prosím vás, Marianna, čo tu robíte?! Už aj bežte
domov a doneste mi rukavice a vejár! Ale chytro!“ Alica sa
tak zľakla, že v tej chvíli sa rozbehla, kde jej Králik ukázal,
ani si len netrúfala vysvetliť mu, že je to nejaký omyl.

„Očividne si ma zmýlil so svojou slúžkou,“ povedala si
v behu.

„Ten bude otvárať oči, keď zistí, kto som! Ale ten jeho
vejár i s rukavicami mu už len donesiem — samozrejme, ak
ich nájdem.“ Len čo to povedala, ocitla sa pred chutným
domčekom. Na dverách sa blyšťala mosadzná tabuľka
s vyrytým menom B. KRÁLIK. Vošla dnu bez zaklopania
a vybehla hore schodmi, celá vystrašená, aby náhodou
nestretla skutočnú Mariannu. Tá by ju iste vyhnala z domu
skôr, ako nájde vejár a rukavice.

„Aké je to čudné,“ vravela si Alica, „že robím poslíčka
Králikovi! Najbližšie si urobí zo mňa poslíčka aj Tina!“
Predstavovala si, ako by to asi vyzeralo: „Slečna Alica!
Poďte sem chytro a Oblečte sa na prechádzku!“ „Hneď,

32

Ako Vilo vyletel z komína

hneď, ňanka! No kým sa Tina vráti, musím dozrieť na túto
myšaciu dieru, aby jej myš neušla.“ „Ibaže,“ uvažovala
ďalej, „Tinu' by sme v dome asi dlho netrpeli, keby nám
takto rozkazovala.

Medzitým Alica vošla do útulnej izbičky; pri okne bol
stolík a na ňom (ako dúfala) vejár a dva ci tri páry drob-
ných bielych rukavíc z kozinkyŠ Vzala vejár i jedny rukavice
a už chcela z izby odísť, keď tu zrazu pri zrkadle zbadala
fľaštičku. Tentoraz nebol na nej lístok s nápisom VYPI MA!,
no aj tak vytiahla zátku a priložila si fľaštičku k perám.
„Vždy keď niečo zjem alebo vypijem,“ povedala si, „prihodí
sa čosi zaujímavé. Uvidím, čo urobí táto fľaštička. Dúfam,
že od nej zasa narastiem, lebo byť takýmto škvŕňaťom,
toma už naozaj otravuje!“

Tak sa aj stalo, a oveľa rýchlejšie, ako sa nazdala.
Nevypila ešte ani pol fľaštičky, a už narazila hlavou do
povaly a musela sa zohnúť, aby si nedolámala väzy. Chytro
položila fľašku na zem a povedala si: „To stačí — dúfam, že
viac už neporastiem — veď ani takto už neprejdem cez
dvere — škoda, že som toho'tolko vypila!“

Beda-prebeda! Neskoro ľutovala. Len rástla a rástla,
čoskoro si musela klaknúť na zem, no O chvíľu už ani to
nestačilo, skúsila si teda ľahnúť, lakťom sa oprieť 0 dvere
a druhou rukou si podložiť hlavu. No ešte vždy rástla, a tak
jej napokon. neostávalo nič iné, len jednu ruku vystrčiť von
oknom a jednu nohu vopchať do komína. Potom si
povedala: „A teraz nech sa už robí, čo chce, ja už viac
nemôžem. Čo len so mnou bude?“

Naštastie zázračná fľaštička práve prestala účinkovať
a Alica už viacej nerástla. Bola tam však veľmi stiesnená.
Vyzeralo to, že sa z tej izby tak ľahko nedostane, nuž
nečudo, že bola nešťastná.

„Doma mi bolo oveľa lepšie,“ pomyslela si nešťastná
Alica, „tam sa človek každú chvíľu nezväčšoval a nezmen-
šoval a nerozkazovali mu tam všelijaké myši a králiky.
Hádam som nemala liezť do tej králičej nory — nuž ale —


 

Alica v krajine zázrakov

nuž ale — takýto život je predsa len zaujímavý! Keby
som vedela, čo sa to vlastne so mnou porobilo! Keď som si
čítala rozprávky, myslela som si, že sa také veci nikdy
nestávajú, a teraz som to zažila na vlastnej koži! Niekto by
mal o mne napísať knihu, teda to by mal! Ked narastiem,
napíšem ju sama - ale veď už som narástla,“ dodala
zarmútene. „Aspoň v tejto izbe už d alej rásť nemôžem.“

„Ale to znamená,“ rozmýšľala Alica, „že ani roky mi
nebudú pribúdať! Keby zo mňa nikdy nebola starena, to by
bolo ohromné. Ale zasa — večne sa len učiť! Och, to by
sa mi nechcelo!“

„Ty hlúpa Alica,“ odpovedala si. „Ako sa tu chceš učiť?
Veď tu niet miesta ani pre teba, ako by sa sem zmestili
učebnice?“

A tak to šlo ďalej; najprv to zvážila z jednej strany,
potom z druhej a dlho sa tak rozprávala sama so sebou. Ale
o chvíľu začula zvonka nejaký hlas, nuž stíchla a počú—
vala.

„Marianna! Marianna!“ ozývalo sa vonku. „Hneď mi
prineste rukavice!“ A potom niekto cupkal po schodoch.
Alica vedela, že to ju hľadá Králik, a zachvela sa, až sa
domček otriasol. Celkom zabudla, že teraz je najmenej
tisíckrát väčšia ako Králik a nemá sa čoho báť.

Králik podišiel k dverám a chcel ich otvoriť, ale keďže sa
otvárali dnu a Alica sa opierala o ne lakťom, nepodarilo sa
mu to. Alica počula, ako si Králik vraví: „Tak dom obídem
a vojdem dnu oknom.“

„A to teda nie!“ pomyslela si Alica, chvíľu počkala,
a keď sa jej zdalo, že Králik je už pod oknom, rýchlo
roztvorila dlaň a chmatla do prázdna. Nechytila nič, ale
niekto zvrieskol a zvalil sa a zarinčalo sklo; usúdila, že
Králik spadol do pareniska s uhorkami alebo do niečoho
takého.

Potom niekto zlostne skríkol — bol to Králik: „Pat! Pat!
Kde si?“ A nato hlas, ktorý dosiaľ nepočula: „Ná de bych
ból? Tot jablká kopem, Vaša Milost.“

34

Ako Vilo vyletel z komína

„No prosím, on si kope jablká!“ hundral Králik. „Pod
sem a pomôž mi stadeto!“

(A znovu zarinčalo sklo.)

„A teraz mi, Pat, povedz, čo je to tamto v okne?“

„Ná co by to malo byt! Ruka, Vaša Milost.“

' „Ruka, ty chumaj! Ktože kedy videl takú obrovskú
ruku?! Veď jej je plné okno!“

„Bár jéj je plné okno, aj tak je to ruka, Vaša Milost,“

„Ruka tam nemá čo robiť. Už aj ju odprac!“

Potom bolo ticho, iba občas počula Alica šepot ako:
„Mne sa to, Vaša Milost, nepozdáva, éj, nijako sa mi to
nepozdáva!“ „Rob, ako tí kážem, ty zbabelec!“ Znova
roztvorila dlaň a po druhý raz chmatla do prázdna. Tento-
raz zvrieskli dvaja a zasa zarinčalo sklo. „Koľko je tam tých
parenísk s uhorkami!“ pomyslela si Alica. „Som zvedavá,
čo urobia teraz! Keby ma tak chceli vytiahnuť z okna, nič
iné si neželám. Lebo jedno je isté, že už mám tohto tu
dosť!“

Chvíľu čakala, ale nič už nepočula. Napokon prihrkotal
nejaký vozrk a potom sa viacero hlasov navzájom prekriko—
valo: „Kde je druhý rebrík? - Mal som priniesť iba jeden.
Druhý má Vilo. Vilo! Podaj ho sem, chlapče! Tuto, o tento
roh ich opri! — Nie, najprv ich zviaž — inak nedosiahnu ani
do polovice. — Tak, teraz to už stačí. Len pokojne. — Hej,
Vilo! Chyť ten povraz! — Len či ho strecha udrží! — Pozor
na uvoľnenú škridlu — už letí! Pozor na hlavu!“ (niečo
tresklo) — „Kto to urobil? — To iste Vilo. — Kto vlezie do
komína? — Ja nie, chod si ty! — Ešte čo! — Nech ta ide
Vilo. — Počuješ, Vilo? Pán vraví, že máš vliezť do komína
ty!“

„Tak teda Vi lo polezie do komína,“ povedala si Alica.
„Zdá sa, že všetko váľajú na toho Vila. Nechcela by som
byť na jeho mieste, kozub je úzky, len čo je pravda. Ale
trocha si hádam len kopnem!“

Stiahla nohu z komína, ako sa len dalo, a čakala, kým
nepočula, ako sa nejaké zvieratko (neuhádla, aké) tesne nad

35

 

Alica v krajine zázrakov

ňou škriabe a driape v komíne. Povedala si: „To je asi ten
Vilo,“ kopla z celej sily a čakala, čo sa bude. robiť.

Najprv Zborove skríkli: „Vilo letí!“ Nato Králikov hlas:
„Chyťte ho, vy tam pri plote!“ Potom ticho, a znova zmätok
a krik: „Nadvihnite mu hlavu! — Trocha pálenky! — Nech
sa nezadusí! — Čo to bolo, braček? Čo sa ti stalo?
Rozprávaj !“

Konečne niekto slabučko zapišťal. „To bude Vilo,“
pomyslela si Alica. „Čo ja viem — už nie, ďakujem, už mi je
lepšie - ešte som z toho priveľmi rozrušený — viem len
toľko, že sa na mňa čosi vyrútilo ani čert zo škatule, a už
som letel ako raketa.“

„To si teda letel, braček,“ prikyvovali ostatní.

„Dom treba podpáliť!“ ozval sa Králik. Alica zvolala
z celej sily: „Ak to urobíte, pošlem na vás/Iinulĺ

Nastalo hrobové ticho a Alica si pomyslela: „Ktovie, čo
urobia teraz? Keby mali trocha rozumu, strhli by strechu.“
Po minúte-dvoch zasa začali pobehovať a Alica začula
Králika: „Na začiatok stačí za jedny táčky.“

„Za jedny táčky? Ale čoho?“ rozmýšľala Alica. No
nebola dlho na pochybách, lebo v nasledujúcej chvíli vleteli
do okna okruhliaky a niektoré ju zasiahli do tváre. „Toto im
zarazím!“ povedala si Alica a skríkla: „Prestaňte s tým!“
Opäť nastalo hrobové ticho.

Alica s úžasom zistila, že len čo kamienky dopadnú na
dlážku, premenia sa na koláče. Razom ju osvietila myšlien—
ka: „Ak jeden z tých koláčov zjem,“ pomyslela si, „iste to
zapôsobí na moju veľkosť.—A keďže väčšia už nemôžem byť,
asi budem od toho menšia.“

Jeden teda zjedla a s radosťou zistila, že sa skutočne
zmenšuje. Keď už bola taká malá, že prešla dverami,
vybehla z domu a vonku na ňu čakal celý kŕdeľ drobných
zvieratiek a vtáčkov. Uprostred stál chudák jašteričiak
Vilo, podopierali ho dve morčatá a napájali ho z fľaše.
Ked sa Alica zjavila, všetko sa vrhlo na ňu, ale ona uháňala,
ako len vládala, a čoskoro bola v bezpečí v hustom lese.

36

Ako Vilo vyletel z komina

— ako to mám urobiť? Myslím, že by som mala niečo vypiť
alebo zjesť, alebo niečo také, no otázka je — čo?“

Čo, to bola naozaj veľká otázka. Alica sa obzerala po
kvetoch a trávach, no nič také, čo by sa za daných okolností
dalo zjesť alebo vypiť, tam nevidela. Neďaleko rástla huba,
skoro taká veľká ako ona. Obzrela si ju odspodku iz oboch
strán a prišlo jej na um, aby sa pozrela, čo je navrchu.

Zodvihla sa na prsty a ponad okraj huby pozrela, čo je na
klobúku. Oči sa jej stretli s veľkými modrými očami

,Húseničiaka.» Sedel tam so založenými rukami, ticho si
bafkal z dlhej vodnej fajky a nevšímal si ani Alicu, ani nič
me.

% fifth chapter
\section{Húseničiakova rada}

HÚSENIČIAK a Alica chvíľu mlčky na seba hľade-
li. Napokon Húseničiak vybral fajku z úst a ozval sa
mdlým, ospanlivým hlasom:

„A ty si kto?“

Takýto začiatok veru nepovzbudzoval do rozhovoru.
Alica odpovedala zarazene: „Ani — ani neviem, pane, teraz
naozaj neviem — viem iba to, kto som bola dnes ráno, keď
som vstala, ale odvtedy som sa už iste niekoľkokrát
premenila.“

„Ako to myslíš?“ spýtal sa Húseničiak strmo. „Vyjadri
sa!“

„Ľutujem, pane, ale vyjadriť sa nemôžem,“ povedala
Alica, „lebo ja nie som ja —- rozurniete?“

„Nerozumiem,“ odpovedal Húseničiak.

„Ľutujem, ale zrozumiteľnejšie vám to už neviem pove-
dať,“ zdvorilo odpovedala Alica, „neviem, ako by som
začala, viete, ja tomu sama nerozumiem; mať za jeden
jediný deň také premenlivé rozmery, to človeka celkom
popletie.“

„Vôbec nie,“ povedal Húseničiak.

„No, možno ste to ešte neskúsili,“ povedala Alica, „ale
keď sa zakuklíte — a jedného dňa sa tomu nevyhnete
— a potom sa zmeníte na motýľa, iste nebudete celkom vo
svojej koži, nemyslite?“

„Ani trocha,“ Húseničiak na to.

„Zdá sa, že vy to prijímate inakšie,“ povedala Alica, „no
keby šlo o mňa, ja by som sa určite cítila nesvoja.“

„Ty, ty!“ opovržlivo povedal Húseničiak. „A kto Vlastne
si?“

40

Húseničiakova rada

Tak sa v rozhovore dostali znova tam, kde začali.
Húseničiakove skúpe odpovede Alicu už trocha hnevali, nuž
sa vystrela a odsekla mu: „Hádam by ste mi mali povedať
najsamprv vy, kto ste!“

„A prečo?“ spýtal sa Húseničiak.

Ďalšia otázka, na ktorú Alica nevedela odpovedať. Nič
vhodné jej neprichádzalo na um a Húseničiak mal podľa
všetkého zlú náladu, nuž sa rozhodla, že odíde.

„Vráť sa!“ volal za ňou Húseničiak. „Poviem ti niečo
dôležité.“

To už bola iná reč. Alica sa zvrtla a vrátila sa.

„Nebuď zlostná,“ povedal Húseničiak.

„A to je všetko?“ spýtala sa Alica a len-len že nevybuch-
la od zlosti.

„Nie,“ riekol Húseničiak.

Alica si povedala, že teda ešte počká, veď aj tak nemá čo
robiť, a možno jej napokon predsa len povie niečo, čo bude
stáť za to.“ Niekoľko minút bez slova pobafkával. Potom
rozhodil rukami, vybral z úst fajku a povedal: „Tak ty si
myslíš, že si sa premenila, čo?“

„Už je to tak, pane,“ povedala Alica. „Nepamätám si, čo
som kedysi vedela - a každých desať minút mením veľ-
kosť!“

„Čo si nepamätáš?“ spýtal sa Húseničiak.

„No, skúsila som zarecitovať ,Hľaďte na tie čudné zvyky“,
a vyšlo mi z toho niečo celkom iné,“ odpovedala Alica
zronene.

„Tak mi zarecituj niečo z klasiky,“ povedal Húseni-
čiak.

Alica si zložila ruky a začala:

„Starý ste, Otče, priam vo veku kmeťa,“
rečie syn, „dávno obelel va'm vlas.

Ako to, že vám nebránia tie letá

na hlave stáť — a denne toľký čas?“

41

Alica v krajine zázrakov

   

„Za mladi, priznám, mával som aj strac ,“

rečie ten, „mozog či tým neutrpí.

No dnes už viem, že v lebke mám len prach,
nuž báť sa - čoho? Nie som predsa hlúpy.“

„Starý ste, otče, hrubý ani žoc ,“
syn na to, „tučný ako prasa v žite.
Načo vám saltá, stojky, odzemok?
Aký v tom zmysel? Nože vysvetlite!“

„Za mladi, “ starec prázdnym pohne ďasnom,
„býval som, synu, šibký ako tr'stie

a touto iba masti! som sa masťou.

Za lacný groš ti predám za dve hrste. “

„Čeľuste, otče, slabé máte už,
krom kaše ine' ťažko pohryziete.
Kam zmizli kosti, zobák, celá hus?
Ako to všetko zjete? Môj ty svete!“

„Narobil som vždy všade veľa kriku,
zloby a hádok - tak sa škrane tužia.
Čas žičí tým, čo nevychodia z cviku;
čeľuste im až po hrob dobre slúžia. “

Húseničiakova rada

„Starý ste, otče, “ mladík neustáva,
„oči sa kalia so slabnúcim zrakom.

A dobre viete: sláva — poľná tráva;
načo na nose balansovať s rakom ?!“
„ Otázok troje —- dosť je, synku milý, “
odvetí starec, „prisám, neznáš miery.
Čuš! Lebo počúvať ťa ďalej — v chvíli
vykopnúť by ťa musel z týchto dverí!“

„Tak to nie je,“ povedal Húseničiak.

„Aj mne sa tak vidí,“ povedala Alica boj azlivo. „Niektoré
slová sú celkom inakšie.“

„Je to zle od začiatku do konca,“ vyhlásil Húseničiak
rozhodným hlasom a potom bolo niekoľko minút ticho.

Prvý sa ozval Húseničiak.

„Aká veľká by si chcela byť?“ spýtal sa.

„Na veľkosti mi ani tak nezáleží,“ vyhŕkla Alica, „no
nepáči sa mi, že sa tak často mením, viete?“

„Neviem,“ odpovedal Húseničiak.

Alica nepovedala nič. Za celý život jej nikto toľko
neodporoval. Cítila, že už stráca trpezlivosť.

„Takto si so sebou spokojná?“ spýtal sa Húseničiak.

„Nuž, ak dovolíte, radšej by som bola o trochu väčšia,“
povedala Alica. „Výška desat centimetrov, to je predsa len
úbohosť.“

„To je celkom dobrá výška,“ nasrdil sa Húseničiak a za
reči sa vystrel (meral presne desať centimetrov).

„Ale keď ja na to nie som zvyknutá,“ bránila sa nešťastná
Alica. A pomyslela si: „Keby sa len to čudo tak neurá—
žalo!“

„Časom si zvykneš,“ povedal Húseničiak, vopchal si fajku
do úst a zasa fajčil.

Tentoraz Alica trpezlivo čakala, kým sa mu uráči preho-
voriť. O chvíľku Húseničiak vybral z úst vodnú fajku, raz či

43

Alica v krajine zázrakov

dva razy zazíval a striasol sa. Potom zišiel z huby, odliezol do
trávy a len tak mimochodom utrúsil: „Keď vezmeš z jednej
strany, narastieš, keď z opačnej, zmenšíš sa.“

„Z jednej strany. .. z opačnej — ale čoho?“ rozmýšľala
Alica.

„Huby,“ ozval sa Húseničiak, akoby sa bola spýtala
nahlas; a v ďalšej chvíli ho už nebolo.

Zamyslená Alica si hubu asi minútu prezerala a rozmýšľa-
la, kde vlastne má tie dve strany; ťažko povedať, keď je
celkom okrúhla. A tak roztiahla ruky, objala hubu čo možno
najďalej a každou rukou z kraja kúsok odštipla.

„A teraz ktorá je ktorá?“ povedala si a odhryzla z kúska

v pravej ruke, aby si overila, čo sa stane. Vtom ju niečo
udrelo do brady; narazila ňou na vlastné nohy.
j Náhla zmena ju veľmi vyľakala, ale uvedomila si, že ak sa
tak rýchlo scvrkáva, nesmie márniť čas a treba sa pustiť do
druhého kúska. Bradu mala takú pritlačenú k nohám, že
ledva ústa otvorila; napokon sa jej horko—ťažko podarilo
prehltnúť kúštik huby z ľavej ruky.

„Konečne mám voľnú hlavu,“ povedala si Alica natešene,
ale jej radosť sa v najbližšom okamihu zmenila na hrôzu,
lebo zistila, že si nedovidí na plecia. Keď pozrela dolu, videla
iba dlhočizný krk, čo vyrastal ani tenké steblo z mora
zeleného lístia, ktoré ležalo hlboko pod ňou.

„Čo je to za zeleň?“ čudovala sa Alica. „A kde sa mi
podeli plecia? Ach, a moje úbohé ručičky, ako to, že vás ani
nevidím?!“ Pri tých slovách pohýbala rukami, ale nič
nedocielila, iba to zelené lístie sa v diaľke zachvelo.

Kedže nemala zrejme ani najmenšiu nádej dočiahnuť si
rukami hlavu, pokúsila sa skloniť hlavu k rukám a s poteše—
ním zistila, že krk sa jej ľahko ohýba na všetky strany celkom
ako had. Podarilo sa jej ohnúť ho do elegantnej krivky a už
ponárala hlavu do lístia — boli to koruny stromov, pod
ktorými sa predtým prechádzala —, keď tu na ňu niekto
prudko zasipel, takže sa zháčila; vyletela proti nej veľká
Holubica a zúrivo ju tĺkla krídlami. \

44

Húseničiakova rada

„Had!“ zvrieskla. .

„Nie som nijaký had!“ odpovedala Alica nahnevane.
„Dajte mi pokoj!“

„Had si, keď ti vravím,“ opakovala Holubica miernejšie
a zavzlykala: „Skúšala som už všeličo, ale tým, ako vidieť,
v ničom nevyhoviem.“

„Nemám poňatia, o čom hovoríte,“ povedala Alica.

„Skúšala som to v koreňoch stromov, skúšala som to na
brehoch rieky, skúšala som to v živých plotoch,“ vykladala
Holubica a ani ju nepočúvala, „ale tie hady! V ničom im
neulahodíš!“

Alicu jej reči čoraz väčšmi miatli, no usúdila, že nemá
zmysel miešať sa do toho, kým Holubica neskončí.

„Nie dosť, že znášam vajíčka,“ povedala Holubica, „ešte
aby som ich vo dne v noci chránila pred hadmi. Tri týždne
som ani oka nezažmúrila!“

„Mrzí ma, že ste sa tak vyplašili,“ povedala Alica, ktorej
pomaly začalo svitať.

„A keď som si napokon našla najvyšší strom v lese,“
pokračovala Holubica a celkom sa rozkričala, „a myslela si,
že konečne budem mať od nich pokoj, pn'plazia sa ku mne
rovno z neba! Fuj, hadisko akési!“

„Ale veď 'vám vravím, že nie som had!“ povedala Alica.
„Ja som — ja som —“ '

„Tak čo si teda?“ spýtala sa Holubica. „Čo si napochytro
vymyslíš?“

„Ja. . . ja som dievčatko,“ povedala Alica neistým hlasom,
lebo si spomenula, koľko ráz sa za ten deň už preme-
nila.

„Povedač !“ zasmiala sa Holubica opovržlivo. „Videla
som už veľa dievčatiek, ale ani jedno nemalo taký krk! Nie,
nie. Si had a nemá zmysel zapierať to. Ešte mi nakoniec
povieš, že si v živote nemala v ústach vajce!“

„Mala som, čoby nie,“ povedala Alica, navyknuté vravieť
vždy pravdu, „ved dievčatká majú rady vajíčka takisto ako
hady.“

45

Alica v krajine zázrakov

„To teda neverím,“ povedala Holubica, „ale ak áno, tak
aj ony patria medzi hady a hotovo!“

Tento záver Alicu tak prekvapil, že na chvíľu celkom
stratila reč. Holubica to využila a dodala: „Je mi jasné, že
sliediš za vajíčkami, a nezáleží mi na tom, či si dievča alebo
had.“

„Ale mne na tom záleží,“ vyhŕkla Alica, „okrem toho, ja
za vajíčkami nesliedim. A keby aj, o vaše nestojím; surové
nemám rada.“

„Tak zmizni!“ povedala Holubica nasrdene a znova sa
usalašila v hniezde.

Alica sa skrčila a predierala sa pomedzi stromy. No aj tak
musela podchvíľou zastať a vymotať si krk, čo jej uviazol
medzi konármi. Po chvíľke si spomenula, že v rukách ešte
stále drží kúsky huby; veľmi opatrne sa do nich pustila, po
kúštičku odštipla najprv z jedného, potom z druhého, raz
trocha podrástla, nato sa zasa scvrkla, kým sa jej napokon
nepodarilo dosiahnuť pôvodnú výšku.

Už dávno nebola ani približne taká veľká ako teraz, takže
spočiatku sa cítila nesvoja; no o chvíľu si zvykla a začala sa
zhovárať sama so sebou, ako to robievala aj inokedy: „Tak,
polovica môjho želania sa splnila. Aké záhadné sú tie
premenil Človek   nikdy nevie, čo sa s ním stane v najbližšej

chyíli! Tak ci onak, mám zasa svoju výšku, terazuž len dôstať '—
sa do tej krásnej záhrady - ale ako?“ Len čo to povedala,
ocitla sa zrazu na čistinke; stál tam domček vysoký len čosi
vyše metra. „Nech už tam býva ktokoľvek,“ pomyslela si
Alica, „takáto veľká k nim nemôžem prísť, veď by od ľaku
zošaleli!“ A tak začala znova odštipkávať z huby v pravej
ruke, a až keď sa zmenšila asi na štvrť metra, odvážila sa

priblížiť k domcu.

% sixth chapter
\section{Prasacie bábätko}

< VÍĽU stála, hľadela na domček a rozmýšľa-
la, čo urobiť, keď tu zrazu vybehol z lesa

lokaj  V liv'ýrejilĺza lokaja ho považovala pre tú livrej; inak
podľa tváre by hádala skôr na rybu) a členkom zabúchal na
dvere. Otvoril mu iný livrejovaný lokaj s okrúhlou tvárou
a vypúlenými žabími očami. Alica si všimla, že obaja lokaji
mali po celej hlave napudrované kučery. Bola veľmi
zvedavá, o čo vlastne ide, nuž vyšla kúsok z lesa, aby si ich
vypočula.

Ryllílokajyytiahol spod pazuchy list, veľký skoro ako on
sám, a škrobene ho podal druhému lokajovi: „Pre Vojvod-
 kyňu. Pozvanie odKráľovnej na partiu kroketu.“ Žabí lo-
kaj to po ňom opakoval rovnako škrobene, ibaže trochu
poprehadzoval slová: „Od Kráľovnej. Pozvanie pre Voj-
vodkyňu na partiu kroketu.“

Nato sa obaja hlboko poklonili, až sa im kučery za-
plietli.

Alicu to tak rozosmialo, že sa znova musela stiahnuť do
lesa — bála sa, že ju začujú. Ked odtiaľ o chvíľu vykukla,
rybieho lokaja už nebolo a ten druhý sedel pri dverách na
zemi a tupo zízal do neba.

Alica placho pristúpila k dverám a zaklopala.

. „Klapať je zbytočné,“ povedal lokaj, „a to pre dve prí—

činy. Po prvé preto, že som na tej istej strane dverí ako ty.  
A po druhé — tam dnu je taký hrmot, že ťa nikto nepočuje.“ l

A skutočne, vnútri sa rozliehal neslýchaný lomoz — ktosi
tam neprestajne vrieskal a kýchal, do toho z času na čas čosi
treslo, ako keď sa tanier alebo hrniec rozbije na kusy.

47

  

 

' , A
5 a
D

Alica v krajine zázrakov

„Ale, prosím vás,“ spýtala sa Alica, „ako sa teda mám
dostať dnu?“

„Klopať by malo zmysel iba vtedy,“ pokračoval lokaj
a Alicu si vôbec nevšímal, „keby dvere boli medzi nami.
Napríklad keby si bola dnu, mohla by si zaklopať a ja by
som ťa pustil von.“ Po celý čas ako rozprával, hľadel do neba
a to sa Alici videlo vyslovene neslušné. „Ale možno za to
nemôže,“ povedala si, „keď má oči takmer navrch hlavy. No
aspoň na otázku by mi mohol odpovedať. — Ako sa teda
mám dostať dnu?“ opakovala nahlas.

„Ostanem tu sedieť dozajtra,“ podotkol lokaj.

Vtom sa dvere na domci otvorili a vyletel z nich veľký
tanier, lokajovi rovno na hlavu; odrel mu nos a roztrieskal
sa na kúsky o strom za ním.

„— alebo možno do pozajtra,“ pokračoval lokaj rovna-
kým hlasom, akoby sa nič nestalo.

„Ako sa mám dostať dnu?“ spýtala sa Alica ešte
hlasnejšie.

„A máš sa ty vôbec dostať dnu?“ spýtal sa lokaj. „To je
totiž prvá otázka.“

To bola bezpochyby pravda, no Alici sa nepáčilo, že jej
niekto dal takú otázku. „To je hrozné,“ zahundrala si, „ako
sa tá čeliadka pre všetko škriepi. Človek by sa z toho
zbláznil.“

Lokajovi sa zdalo, že je vhodná chvíľa zopakovať s istými
obmenami predošlú poznámku: „Ostanem tu sedieť a se-
dieť, hoci aj celé dni,“ povedal.

„Ale čo mám robiť ja?“ spýtala sa Alica.

„Rob si, čo chceš,“ lokaj na to a začal si pískať.

„Och, nemá zmysel zhovárať sa s ním,“ povedala si zúfalá
Alica. „Veď je to akýsi chmuľo!“ Otvorila dvere a vošla
dnu.
rĎverami sa vchádzalo rovno do priestrannej, no neuveri-
teľne začmudenej kuchyne. Uprostred na trojnožke sedela

,Vojvodkyr'ia a pestovala dojča. Kuchárka sa skláňala nad
ohňom a miešala plný kotol, podľa všetkého s polievkou.

48

Prasacie bábätko

„V tej polievke bude priveľa korenia,“ povedala si Alica,
pokiaľ vôbec mohla, lebo kýchala a kýchala.

Vo vzduchu ho zaručene bolo priveľa. Dokonca aj
Vojvodkyňa občas kýchla a dojča, to kýchalo a rumádzgalo
bez prestania.

Nekýchala iba kuchárka a veľká mačka, ktorá ležala pri
kozube a škľabila sa od ucha k uchu.

„Povedali by ste mi, prosím vás,“ bojazlivo sa ozvala
Alica, lebo si nebola istá, či sa patrí, aby prehovorila prvá,
„prečo sa tá vaša mačka tak škľabí?“

„Lebo je to mačka Škľabka,“ povedala Vojvodkyňa.
„Preto je to tak. Ty prasa!“

Posledné slová vyštekla tak nečakane, že Alicu až myklo;
ale hneď zistila, že nepatrili jej, lež bábätku. Dodala si teda

odvahy a pokračovala:
„Nevedela som, že sú nejaké mačky Škľabky. Nevedela
som ani to, že sa mačky vôbec vedia škľabiť.“

sa aj škľabia.“

„Neviem ani o jednej, čo by sa škľabila,“ povedala Alica
veľmi zdvorilo, lebo bola rada, že sa má s kým poroz-
právať.

„Veľa toho teda nevieš,“ na to Vojvodkyňa, „len čo je
pravda.“

Tón tejto poznámky sa Alici nepozdával, nuž si pomysle-
la, že najlepšie bude rozprávať sa o niečom inom. Wm
premýšľala, o čom, kuchárka odtiahla kotol z ohňa a všetko,
čo jej prišlo pod ruky, z ačala odrazu hádzať do Vojvodkyne

49

  

(*
ľ
ľ
v

 

Alica v krajine zázrakov

a bábätka - najprv prišli na rad kliešte, kutáč a lopatka, za
nimi nasledovali panvice, taniere a misy. Vojvodkyňa si ich
nevšímala, dokonca ani vtedy, keď ju niektoré udreli;
a bábätko tak či tak neprestajne rumádzgalo, takže sa
nevedelo, či aj ono dačo neutřžilo.

„Prepánaj'ána, dávajte, prosím, pozor, čo robíte!“ vy-
kríkla Alica, od hrôzy celá bez seba. „Ach, noštek!“ zvolala,
keď mu popri nose preletela riadne veľká panvica a veľa
nechýbalo, bola by mu ho odrazila.

..»!Seby  každý ä.?afěl.-.lqn 9ĺiv 919   vggii povedala
Vojvodkyňa Vchr—apfavýiň hlasom, „svet by sa krútil oveľa
rýchlejšie-.“

„A čo by sme tým získali,“ povedala Alica a potešila sa,
že sa môže trochu pochválit svojimi vedomosťami. „Ved si
len predstavte, čo by sa stalo s dňom a nocou! Iste viete, že
Zem potrebuje dvadsaťštyri hodín, aby sa obrátila okolo
svojej osi, no keby sme z tohto času chceli odsek —“

„steknite jej  hlavu,“ skočila jej do reči Vojvodkyňa.
„Dobre, že si mi to pripomenula.“

Alica so strachom pozrela na kuchárku, či sa toho chytí,
ale tá už zase miešala polievku a podľa všetkého nepočúvala,
nuž Alica pokračovala: „Dvadsaťštyri hodín, myslím… Či
len dvanásť? Ja —“

„Daj mi s tým pokoj !“ povedala Vojvodkyňa. „Počty mi
nikdy nešli!“ A znova začala čičíkať dieťa, spievala mu
pritom akúsi uspávanku a po každom verši ním riadne
potriasla.

 ĺ/„Len ty zrúkni na chalana

ja zmla'ť ho, keď ti kýcha;
kýcha naschvál, líška planá,
trucuje, ako dýcha. “

ZBOR
(pridala sa kuchárka i bábätko):
„ Vau! Vau! Vau!“

50

Prasacie bábätko

Pri dmhej slohe Vojvodkyňa pohadzovala dieťa tak
prudko, že sa ten drobček náramne rozplakal a Alica

slovám piesne skoro nerozumela:

„ Chlapčisku sa výcvik zišiel:
hneď dostal, len čo kýchol;
tak koreniu na chuť prišiel

a doma máme ticho. “

ZBOR:
„ Vau! Vau! Vau!“

Na! Popestuj ho chvíľu, ak chceš,“ povedala Vojvodky-

ňa Alici a hodila jej decko. „Treba sa mi prichystať na ten
kroket s Kráľovnou, “ a už jej nebolo. Kuchárka šmarila za
ňou pekáč a len—len ze ju netrafilá.

Alica mala čo robiť, aby dieťa Zachytila; lebo to bolo akési <-
neforemné stvorenie, ruky a nohy mu trčali na všetky strany.
„Ako morská hviezdica,“ pomyslela si Alica. Keď to
úbožiatko konečne držala v náručí, fučalo ako rušeň,
neprestajne sa zmršťovalo a znova rozpínalo, takže v prvej
chvíli ho ledva udržala.

Keď prišla na to, ako ho má správne pestovať (zvinula ho
do akéhosi uzlíka a mocne držala za pravé ucho a ľavú nohu,
aby sa nemohlo vystrieť), vyniesla ho na čerstvý vzduch. „Ak
to dieťa neodnesiem,“ rozmýšľala, „za deň, za dva bude po
ňom. Nechať ho tu by bola hotová vražda.“ Posledné slová
povedala už nahlas a škvŕňa jej na odpoved zakrochkalo
(kýchať už prestalo). „Nekrochkaj,“ hrešila ho Alica, „to sa
nepatrí!“

Dieťa znova zakrochkalo a Alica mu znepokojene pozrela
do tváre, čo mu vlastne je. Noštek malo bezpochyby
priveľmi ohnutý, bol to skôr rypáčik ako poriadny nos; aj oči
boli na bábätko primalé: jedným slovom — to stvorenie sa
jej ani trocha nepáčilo. „Možno len zavzlykalo,“ pomyslela
si Alica a znova mu pozrela do očí, či v nich nemá slzy.

51

Alica v krajine zázrakov

Nie, slzy nemalo. „Ak sa zmeníš na prasiatko, moje milé,“
povedala Alica vážnym hlasom, „nič s tebou nechcem mať.
To si pamätaj!“ Úbohé škvŕňa zasa zavzlykalo (alebo
zakrochkalo —- ťažko povedať, čo to bolo) a chvíľu bolo
ticho.

Alica práve rozmýšľala: „Čo si ja s ním len doma
počnem?“ keď sa znova tak rozkrochkalo, že mu celá
vyľakaná pozrela do tváre. Nemýlila sa: bolo to naozaj
prasiatko a Alica si uvedomila, že niesť ho ďalej už nemá
zmysel.

Postavila teda zvieratko na zem a veľmi jej odľahlo, keď
pokojne odbehlo do lesa. „Keby podrástlo,“ povedala si,
„bolo by z neho veľmi škaredé dieťa. Ale prasiatko bude
z neho azda celkom pekná'j

Rozmýšľala o známych deťoch, z ktorých by boli celkom
obstojné prasiatka, a práve si vravela: „Len keby človek
vedel, ako ich premeniť —“ no vtom sa preľakla; niekoľko
metrov pred ňou sedela na strome mačka Škľabka.

Keď zbadala Alicu, iba sa zaškľabila. „Vyzerá celkom
dobrosrdečne,“ pomyslela si Alica. No pri pohľade na jej
zuby a dlhé pazúry si povedala, že bude istejšie správať sa
k nej úctivo.

„Milá Škľabka,“ začala nesmelo, lebo si nebola istá, či sa
jej také oslovenie bude pozdávať. Ale Mačka sa len väčšmi
zaškľabila. „No prosím, jej sa to celkom páči,“ pomyslela si
Alica a pokračovala. „Povedala by si mi, prosím, ako sa
stadeto dostanem?“

„Záleží na tom, kam sa chceš dostať,“ odpovedala jej
Mačka.

„To je mi jedno,“ Alica na to.

„Nuž tak je jedno, kade pôjdeš,“ povedala Mačka.

„ — len aby som sa dakde dostala,“ vysvetľovala Alica.

„Ale to sa iste dostaneš,“ povedala Mačka, „len musíš
dobre kráčať.“

Alica uznala, že to nemožno poprieť, a tak to skúsila
z iného konca. „Akí ľudia tu na okolí žijú?“

52

Prnsacie bábätko

„Týmto smerom,“ povedala Mačka a kývla pravou
labkogMKjgbučpíkg tamtým smerom,“ kývla druhou
sú šibnpth

„Ale ja nechcem ísť medzi šibnutých,“ bránila sa Alica.

 

sot; 1  Šillnutá.  Ty si šibnutá.“
ä„Ako vieš, že som šibnutá?“ spýtala sa Alica.

„Musíš byť,“ povedala Mačka, „inak by si sem ne-
prišla.“

Alica to nepovažovala za dôkaz, no pýtala sa ďalej: „A
ako vieš, že ty si šibnutá?“

„Predovšetkým,“ povedala Mačka, „pes nie je šibnutý.
V tom sa iste zhodneme, či nie?“

„Možno,“ súhlasila Alica.

„No prosím,“ pokračovala Mačka, „pes, keď je nazloste-
ný, vrčí, a keď má radosť, krúti chvostom. A ja vrčím, keď
mám radosť, a lcrútim chvostom, ked sa zlostím. Som teda
šibnutá.“

„Ja by som to nenazývala vrčanie, ale pradenie,“ namietla
Alica.

„Nazvi si to, ako chceš,“ povedala Mačka. „Hráš dnes
11 Kráľovnej kroket?“

„Rada by som,“ povedala Alica, „ale ešte som nedostala
pozvánku.“

„Uvidíš ma tam,“ povedala Mačka a zmizla.

Alicu to ani neprekvapilo, celkom si už zvykla, že sa tu
dejú čudné veci. Pozerala na miesto, kde predtým bola
Mačka, a tá sa zrazu opät zjavila.

„Mimochodom — čo je s tým dieťaťom?“ spýtala sa
Mačka. „Skoro som sa zabudla opýta .“

„Zmenilo sa na prasiatko,“ odpovedala Alica tak pokoj-
ne, akoby to bolo celkom normálne.

„Myslela som si to,“ povedala Mačka a opäť zmizla.

Alica chvíľu čakala, či sa ešte neukáže, ale keď sa
nezjavovala, o chvíľu pokračovala v ceste tým smerom, kde

53

 

 

 

 

Alica v krajine zázrakov

vraj býva Aprílový zajac. „Klobučníkov som už videla
dosť,“ povedala si, „Aprílový zajac bude oveľa zaujímavejší
— a keďže je máj, nebude hádam až tak veľmi šibnutý
— aspoň nie natoľko ako v apríli.“ Ako to povedala,
zodvihla hlavu a ľaľa, na strome znova sedela Mačka.

„Povedala si prasiatko, či mačiatko?“ spýtala sa.

„Povedala som prasiatko,“ odpovedala Alica, „ale nema-
la by si sa tak z ničoho nič zjavovať a zase miznúť. Človeku
sa,;štaoho krúti hlava!“

i';',Prosím,“ povedala Mačka a tentoraz sa strácala veľmi
pomaly, začala končekom chvosta a skončila úškľabkom
a ten tam ostal ešte chvíľu po tom, keď jej už nebolo.

„Nuž, mačku bez úškľabku som-už neraz videla,“ pomys-
lela si Alica, „aleýškľabok bez mačky?.To je najpodivuhod-
n ejšia vec, s .akou som sa kedy stretlagjf

Netrvalo dlho a stála pred domom Aprílového zajaca.
Dovtípila sa, že to bude on, lebo komíny pripomínali zajačie
uši a strecha nebola pokrytá slamou, ale kožušinou. Bol to
veľký dom, a tak si netrúfla priblížiť sa k nemu, kým si
neodštipla kúsok huby z ľavej ruky a nepovyrástla na
šesťdesiat centimetrov. Ešte aj potom kráčala k nemu
bojazlivo, lebo si myslela: „Čo ak je priveľmi šibnutý?
Tuším som mala ísť radšej ku Klobučníkovi.“

% seventh chapter
\section{Bláznivý olovrant}

  vý  zajacýaKlobučník pili čaj, medzi nimi sedel plch

P OD stromom na priedomí bol prestretý stôl; Aprílo-

Sedmospáč a spal ako zabitý. Tí dvaja sa opierali o neho
lakťami ako o operadlo a zhovárali sa mu nad hlavou.
„Bohvieaké pohodlné to nemá,“ pomyslela si Alica, „ale
keď spí, hádam je mu to jedno.“

Stôl bol veľký, ale tí traja sa tlačili na jednom rohu. „Niet
tu miesta! Niet tu miesta!“ vykrikovali, keď zbadali Alicu.

„Miesta je tu až-až,“ odsekla im Alica a sadla si do
veľkého kresla na druhom konci stola.

„Napi sa vína,“ ponúkal ju Aprílový zajac.

Alica pozrela na stôl, ale bol tam len čaj. „Nijaké víno
nevidím,“ povedala.

„Veď tam ani nie je,“ na to Aprílový zajac.

„Tak nebolo od vás veľmi slušné ponúkať mi ho,“ hnevala
sa Alica.

„Ani od teba nebolo veľmi slušné sadnúť si k nám len tak
bez pozvania,“ povedal jej Aprílový zajac.

„Nevedela som, že je to váš stôl,“ bránila sa Alica. „Veď
je tu prestreté pre viac ľudí, nie iba pre troch.“

„Mala by si sa dať ostrihaf,“ povedal Klobučník. Už dlhší
čas si Alicu zvedavo prezeral a iba teraz sa ozval.

„Nechajte si také osobné poznámky a nemiešajte sa do
mojich súkromných vecí,“ odsekla mu Alica. „To je
bezočivosť!“

Klobučník vypleštil oči, ale zmohol sa iba na otázku:
„Prečo je havran ako písací stôl?“

„Teraz bude veselo. Som rada, že začali s hádankami,“

55

Alica v krajine zázrakov

pomyslela si Alica. „Myslím, že to uhádnem,“ povedala
nahlas.

„Chceš povedať, že na to vieš odpovedať?“ spýtal sa
Aprílový zajac.

„Presne to,“ povedala Alica.

„Tak povedz, čo si myslíš,“ pokračoval Aprílový zajac.

„A či nevravím?“ vyhŕkla Alica. „Totiž — totiž myslím si
to, čo vravím — veď to je, dúfam, to isté.“

„Vôbec nie,“ odporoval jej Klobučník. „Kdeže. Tak by si
mohla povedať, že ,Vidím, čo jem“ je to isté ako ,Jem, čo
vidím“!“

„A takisto by si mohla povedať,“ ozval sa Sedmospáč
akoby zo sna, „že ,Dýcham, keď spírn' je to isté ako ,Spím,
ked dýcham'!“

„No, u teba je to, pravdaže, to isté,“ povedal Klobučník.
Rozhovor ustal, na chvíľu sa odmlčali. Alica sa v duchu
rozparnätávala, čo vie o havranoch a písacích stoloch.

Nebolo toho veľa.

Prvý sa ozval Klobučník. (ako je dnes?“ obrátil sa na
Alicu— Výytiahol—zĺrrĺecka hodinky, rozčulene ich potriasal
a prikladal si ich k uchu.

Alica chvíľu rozmýšľala: „Štvrtého.“

„Dva dni meškajú!“ vzdychol si Klobučník. „Vravel
som ti, že maslo im uškodí,“ oboril sa na Aprílového
zajaca.

„Veď to bolo výberové maslo,“ odpovedal Aprílový zajac
zronene.

„Dobre, ale s maslom sa ta iste dostali aj omrvinky,“
hundral Klobučník, „nemal si to robiť nožom na chlieb.“

Aprílový zajac vzal hodinky a zamračené na ne pozeral,
potom ich ponoril do šálky čaju a znova si ich prezeral, ale
nič múdrejšie mu neprišlo na um, len čo už raz povedal:-
„Veď to bolo výberové maslo.“

Alica mu zvedavo pozerala ponad plece. „Aké sú to čudné
hodinky,“ utrúsila. „Ukazujú, kOľkého- je, a nie koľko je
hodín.“

56

Bláznivý olovrant

„Čo je na tom čudné?“ zahundral Klobučník. „Ukazujú

vari tvoje hodinky, ktorý rok máme teraz?“
„Jasné, že nie,“ pohotovo odpovedala Alica. „Načo by aj,

ked rok trvá tak dlho.“

„No a tak je toi v mojom prípade,“ na to Klobučník.

To Alicu už celkom zmiatlo. Klobučníkova poznámka
akoby nemala zmysel, hoci z jazykovej stránky bola
správna.

„Akosi vám dosť dobre nerozumiem,“ povedala čo
najzdvorilejšie.

„Ten Sedmospáč zasa spí,“ povedal Klobučník a kvapol

mu na nos trocha horúceho čaju.
Sedmospáč nervózne potriasol hlavou, ani oči neotvoril

 

a povedal:
„Jasné, jasné, aj ja som to chcel povedať.“

„Rozlúštíla si už tú hádanka?“ obrátil sa Klobučník na
Alicu.

„Nie, poddávam sa,“ odpovedala Alica. „Čo je to?“

„Nemám ani poňatia,“ povedal Klobučník.

„Ani ja,“ na to Aprílový zajac.

Alica omrzene vzdychla. „Podľa mňa by ste mali tráviť
čas múdrejšie, a nie zabíjať ho hádankami, čo sa nedajú
uhádnuť.“

ĺKeby si poznala Čas tak ako ja,“ povedal Klobučník,
„nevravela by si, že ho zabíjam. Skôr on zabíja mňa.“

„Ja vám nerozumiem,“ povedala Alica.

„Ani nemôžeš,“ povedal Klobučník a pohŕdavo kývol
hlavou. „Iste si sa s Časom ešte nikdy ani nezhováralafř

„Asi nie,“ poznamenala Alica Opatrne, „ale viem, že keď
cvičím na klavíri, musím si čas vydupávať nohou.“

„Ach! Teraz je to už jasné!“ povedal Klobučník. „Čas
neznáša, keď ho niekto vydupáva, a ešte k tomu nohou. Ale
. kto je s ním zadobre, tomu napraví hodiny, ako si zažiada.
'V Povedzme, je osem hodín ráno, začína sa vyučovanie. Len
Času- naznačíš, čo chceš, a ručičky doslova  preleýltŕiapo'
i ciferníku. Pol jednej, ide sa na obed!“

57

 

Alica v krajine zázrakov

„Na to by som pristal aj ja!“ zašepkal Aprílový zajac.

„To by bolo úžasné,“ zamyslela sa Alica, „ibaže, viete, ja
by som ešte nebola hladná.“

„Spočiatku možno nie,“ povedal Klobučník, „ale na tej
pol jednej by ručičky ostali tak dlho, koľko by si chcela.“

„A vy to tak robíte?“ spýtala sa Alica.

Klobučník smutne zavrtel hlavou. „Kdeže! Pohádali sme
sa vlani v apríli — vieš, tesne pred tým, ako tomuto tu
preskočilo —“ (a čajovou lyžičkou ukázal na Aprílového
zajaca) „bolo to na gálakoncerte u Srdcovej kráľovnej a ja
som tam mal spievať:

Keby som bol malým netopierom,
hojdal by sa každým podvečerom. ..

Poznáš tú pieseň?“
„Už som ju kdesi počula,“ povedala Alica.
„Tak vieš,“ pokračoval Klobučník, „ako to ide ďalej:

s napchatým gágorom,
s čajkami nad morom,
v zelenej suknici

ako čaj v kanvici…“

Vtom sa Sedmospáč strhol a začal spievat:

„Čaj, čaj, čaj, zelenaj sa háj,
čaj, čaj, čaj, zelenaj sa háj, “

a spieval tak dlho, kým ho neuštipli, aby už prestal.
„Nuž a ledva som dospieva] prvú strofu, Kráľovná
zvrešťala:
  ,Zabíja čas! Odtnite mu hlavu!“ “
1] „To je strašne surové!“ skríkla Alica.
.' „A odvtedy,“ pokračoval Klobučm'k smutne, ,môžem
„ prosiť, koľko chcem, v ničom mi nevyhovie. Stále je päť
( hodín.“

58

Alica v krajine zázrakov

Alici svitlo: „Tak preto sa vám tu nakopilo toľko čajových
šálok?“ spýtala sa.  ,

„Hej, preto,“ prisvedčil Klobučník a vzdychol si: , ,lStále je
Šolovrant, nemáme čas ani riad poumýva; i

„Tak si len zakaždým presadnete o kúsok ďalej?“ opýtala
sa Alica.

„Ako vravíš,“ prisvedčil Klobučník, „keď sa jedny šálky
a pn'bory zašpinia, trocha sa posunieme.“

„A čo keď sa vrátite na samý začiatok?“ osmelila sa spýtať
Alica.

„Zhovárajme sa radšej o niečom inom,“ skočil jej do reči
Apn'lový zajac a zívol. „Už ma to zunovalo. Navrhujem, aby
nám slečinka porozprávala nejakú rozprávku.“

„Ale ja nijakú neviem,“ preľakla sa Alica.

„Tak nech rozpráva Sedmospáč!“ skríkli obaja. „Hore
sa, Sedmospáčik!“ a štuchali do neho z jednej i druhej
strany.

Sedmospáč pomaly otváral oči. „] a som nespal,“ ozval sa
mdlým, chrapľavým hlasom, „počul som každé vaše slovo,
bračekovci.“

„Rozprávaj nám rozprávku!“ povedal Aprílový zajac.

„Áno, rozprávaj!“ prosíkala Alica.

„A ponáhlaj sa s ňou,“ dodal Klobučník, „lebo zasa za-
spíš skôr, ako dôjdeš na koniec.“

„Kde bolo, tam bolo, boli raz tri sestričky,“ spustil
Sedmospáč tak rýchlo, ako len vedel, „Elza, Lolka a Tilka,
a žili na dne studničky —“

„A čím sa živili?“ spýtala sa Alica, ktorá vždy prejavovala
veľký záujem o jedenie a pitie.

„Živili sa melasou,“ odpovedal Sedmospáč po krátkom
rozmýšľaní.

„To nie je možné,“ poznamenala Alica mierne. „Z toho
by ochoreli.“

„Ved aj ochoreli, a ako!“ povedal Sedmospáč.

Alica sa pokúšala predstavit si, ako asi vyzerá taký

60

Bláznivý olovrant

nezvyčajný život, ale nevedela si
s tým rady, tak pokračovala: „Prečo
vlastne žili na dne studničky?“

„Dolej si ešte čaju,“ povedal Alici
Aprílový zajac celkom vážne.

„Nijaký čaj som ešte nemala,“
odpovedala Alica urazene, „tak si nemôžem doliať.“

„Chceš povedať odliať, nie?“ povedal Klobučník. „Tam,
kde nič nie je, ľahšie sa dolieva ako odlieva.“

„Vás sa nik nepýtal,“ odvrkla mu Alica.

„Aha! A kto teraz robí osobné poznámky?!“ spýtal sa
Klobučník víťazoslávne.

Alica nevedela, čo na to povedať. Naliala si teda trocha
čaju, vzala si chlieb s maslom a znova sa spýtala Sedmospá-
ča: „Prečo vlastne žili na dne studničky?“

Sedmospáč zasa chvíľu rozmýšľal a potom povedal: „Bola
to melasová studnička.“

„Také nebývajú!“ rozčúlila sa Alica, ale Klobučník
a Aprílový zajac ju tíšili: „Pst! Pst!“ — a Sedmospáč pove-
dal namrzene: „Keď sa nevieš slušne správať, dorozprávaj
si rozprávku sama!“

„Nie, rozprávajte ďalej !“ prosila Alica. „Už vás nebudem
vyrušovať. Myslím si, že jedna taká studňa by voľakde azda
aj mohla byť.“

„Vraj jedna a azda!“ rozhorčoval sa SedmosPáč. No dal si
povedať a pokračoval: „Tie tri sestričky sa veľmi vyčerpávali
tým, že —“

„Čo? Vyčerpávali?“ spýtala sa Alica; celkom zabudla, čo
sľúbila.

 

61

Alica v krajine zázrakov

„Čo vyčerpávali?! No melasu!“ odpovedal Sedmospáč
tentoraz bez rozmýšľania.

„Chcem čistú šálku,“ prerušil ich Klobučník, „presadni-
me si o jedno miesto.“

A hned si sadol na vedľajšiu stoličku. Sedmospáč urobil to
isté. Aprílový zajac si presadol na Sedmospáčovo miesto
a Alica si voľlcy—nevoľky sadla na miesto Aprílového zajaca.
Dobre pochodil iba Klobučník, Alica bola na tom oveľa
horšie ako predtým, lebo Aprílový zajac si pred chvíľou
prevrhol na tanier kanvicu na mlieko.

Alica nechcela Sedmospáča znova uraziť, preto začala
veľmi opatrne: „Ale ja tomu dobre nerozumiem. Skade tú
melasu vyčerpávali?“

„Ak sa zo studne na vodu čerpá voda,“ povedal Klobuč-
ník, „tak sa zo studne na melasu čerpá melasa, ty hlu-
paňa!“

„Ale ako z nej mohli čerpať, keď boli v nej?“ povedala
Alica Sedmospáčovi; Klobučníkovu poznámku si ani ne-
všimla.

„V nej čerpali z nej a z nej čerpali v nej,“ vysvetľoval jej
Sedmospáč.

ĺ Táto odpoveď chuderu Alicu tak zrniatla, že Sedmospáča
chvíľu vôbec neprerušovala.

„Ako som povedal, veľmi sa tým vyčerpávali,“ pokračo-
val Sedmospáč, pritom zíval a trel si oči, lebo bol už veľmi
ospanlivý, „a kým nevyčerpali tú melasu a všetko ostatné, čo
sa začína na M —“

„Prečo na M?“ spýtala sa Alica.

„Prečo nie?“ spýtal sa Aprílový zajac.

Alica zrnĺkla.

Sedmospáč už zavieral oči a podriemkaval, ale keď ho
Klobučník štuchol, slabo zaškrečal, trocha sa prebral a po-
kračoval: „— všetko, čo sa začína na M, napríklad myšacie
chvosty, mesiac, márnosť, mucha — viete, že sa vraví:
,Márnosť, mucha mu sadla na nos“ — no počuli ste už, aby
niekto čerpal zo studne muchy?“

62

Bláznivý olovrant

„Keď sa tak pýtate,“ povedala Alica v rozpakoch, „ja
naozaj nemyslím  “

„Tak by si nemala ani hovoriť,“ prerušil ju Klobučm'k.

To už bolo priveľa aj na Alicu. Znechutene vstala
a odchádzala. Sedmospáč ihneď zaspal a tí dvaja si ani
nevšimli, že odchádza, hoci sa raz či dva razy obzrela
v nádeji, že ju zavolajú nazad. Naposledy zazrela, ako
pchajú Sedmospáča do čajovej kanvice.

„Tam sa už nevrátim za nič na svete,“ povedala si Alica,
keď sa predierala lesom. „Bláznivejší olovrant som ešte
nezažila.“

Ako to povedala, všimla si, že na jednom strome sú
dvierka. ,;To je čudné!“ pomyslela si. „Ale dnes je všetko
čudné. Najlepšie bude, keď pôjdem rovno dnu.“ A vošla.

A znova sa ocitla v dlhej sieni, hneď vedľa skleného
stolíka. „Už si dám lepší pozor,“ povedala si, vzala zlatý
kľúčik a odomkla dvierka do záhrady. Potom si odštipkávala
z huby (kúštik ešte mala vo vrecku), kým sa nezmenšila asi
na tridsať centimetrov. Potom prešla chodbičkou a potom
— konečne bola v prekrásnej záhrade medzi žiarivými
záhonmi kvetov a chladnými vodometmi.

% eighth chapter
\section{Kráľovnino kroketové ihrisko}

RI vchode do záhrady bol veľký ružový ker. Rástli na

ňom biele ruže, ale traja záhradníci ich usilovne
natierali na červeno. Alici sa to videlo veľmi čudné, chcela sa
im prizrieť zblízka, a keď došla k nim, počula, ako jeden
vraví: „Dávaj pozorďä torka! Fŕkaš na mňa farbu!“

„Ja za to nemôžem,“ zahundral Pätorka. „Sedmička ma
štuchol do lakťa.“

Sedmička hneď zdvihol hlavu a povedal: „Tak, tak,
Pätorka! Len to vždy zvaľ na druhého.“

„Ty sa radšej neozývaj!“ povedal Pätorka. „Práve včera
som počul Kráľovnú, ako vravela, že by si si zaslúžil, aby ťa
popravili.“

„A za čo?“ spýtal sa ten prvý.

„Teba, Dvojka, do toho nič!“ povedal Sedmička.

„Ako to, že nič!“ povedal Pätorka. „Ja mu to poviem - za
to, že namiesto cibule doniesol kuchárovi tulipánové ci-
buľky.“

Sedmička šmaril štetec o zem a začal lamentovať: „Počuli
ste, taká nespravodlivpsť —“ no vtom zbadal Alicu, ako tam
postáva a pozoruje ich, a hneď sa zarazil. Aj ostatní sa
obzreli a všetci sa hlboko poklonili.

„Povedzte mi, prosím,“ začala Alica nesmelo, „prečo
natierate tie ruže?“

Pätorka a Sedmička nepovedali nič, iba hľadeli na
Dvojku. Dvojka ticho vysvetľoval: „Viete, slečinka, to je
tak, toto mali byť červené ruže, ale my sme omylom zasadili
biele. Keby to Kráľovná zistila, prišli by sme o hlavy. A tak,
slečinka, robíme, čo môžeme, kým príde —“ Vtom Pätorka,
ktorý vystrašene napínal zrak cez celú záhradu, zvolal:

64

Alica v krajine zázrakov

„Kráľovná! Kráľovná!“ Všetci traja záhradníci sa razom
hodili tvárou k zemi. Ozval sa dupot a Alica sa obzrela, lebo
na Kráľovnú bola veľmi zvedavá.

Prví šli krížoví vojaci. Vyzerali ako tí traja záhradníci, boli
'podlhovastí a plochi a ruky a nohy mali v rohochf Potom šlo
- desať károýých dvoranov, takisto ako 'Vojaci aj oni pochodo-

vali v dvojstupe. Za nimi šlo desať kráľovských detí; drobizg
sa po dvoch držal za ruky a veselo poskakoval. Deti boli
vyzdobené srdcami. Ďalej kráčali hostia, zväčša kráľovia
a kráľovné, a medzi nimi Alica zazrela aj Bieleho králika;
rozčúlene niečo mlel, všetkému, čo kto povedal, sa smial
a prešiel popri Alici, ani ju nezbadal. Srdcový dolník niesol
na tmavočervencj zamatovej poduške kráľovskú korunu.
Na samom konci tohto veľkolepého sprievodu kráčali
i s R  1 )  cový KRÁĽ a SRDCQVÁ KRÁĽOVNÁ.

Alica váhala, či sa nevrhnút tvárou k zemi ako tí traja
záhradníci, ale nepamätala sa, že by bol pri sprievodoch taký
predpis. „Okrem toho, aký zmysel by mal sprievod,“
pomyslela si, „keby sa všetci ľudia hádzali tvárou k zemi
a nič by z neho nevideli?“ A tak tam len stála a čakala.

Keď sa sprievod priblížil k Alici, všetci zastali, pozerali na
ňu a Kráľovná sa strmo spýtala: „A toto je kto?“ Obrátila sa
na Srdcového dolníka, no ten sa iba uklonil a namiesto
odpovede sa usmial.

„Truľo!“ Kráľovná nazlostene mykla hlavou a obrátila sa
k Alici: „Ako sa voláš, dievča?“

„Volám sa Alica, vaša výsosť,“ odpovedala jej veľmi
zdvorilo, ale v duchu si povedala: „Ved sú to vlastne len
karty! Prečo by som sa ich mala báť?!“

„A čo tamtí?“ Kráľovná ukázala na troch záhradníkov pri
ružovom kríku. Ležali, ako vieme, tvárou k zemi a na chrbte
mali takú vzorku ako ostatní, nuž nevedela, či sú to
záhradníci, vojaci, dvorania, alebo dokonca tri z jej vlast-
ných detí.

„Ako to ja mám vedieť?“ odpovedala Alica, prekvapená
vlastnou trúfalosťou. „To nie je moja starosť.“

66

Kráľovnina kroketové ihrisko

'Kráľovná od zlosti očervenela ani paprika, chvíľu si ju
premeriavala ako dravá šelma a napokon zj ačala: „Hlavu jej
zoťať! Hlavu jej j“

„Hlúposť!“ zvolala Alica a Kráľovná stíchla.

Kráľ položil Kráľovnej ruku na plece a nesmelo sa
ozval:

„Uváž, drahá, že je to len dieťa.“

Kráľovná sa mu zlostne vytrhla a rozkázala Dolníkovi:

„Obráť ich!“

Dolnik ich opatrne jednou nohou poobracal.

„Vstaňte!“ zvrieskla Kráľovná. Traja záhradníci v mihu
skočili na nohy a klaňali sa Kráľovi, Kráľovnej, kráľovským
deťom a všetkým ostatným.

„Prestaňte!“ skríkla Kráľovná. „Už sa mi z vás hlava
krúti!“ Nato pozrela na ružový krik a spýtala sa: „Čo ste to
tu stvárali?“

„Ak dovolíte, vaša výsosť,“ povedal Dvojka ponížene
a kľakol si na jedno koleno, „my sme chceli -“

„Ach' ták!“ skočila mu do reči Kráľovná, ktorá si už
medzitým poobzerala ruže. „Hlavy im zoťať!“ A sprievod sa
pohol ď alej, zostali tam iba traja vojaci, aby vykonali
rozsudok. Nešťastní záhradníci hľadali pomoc 11 Alice.,

„Popraviť vás nepopravia!“ povedala Alica a strčila ich do
veľkého kvetináča, ktorý stál obďaleč. Vojaci sa tam chvíľu
motali, hľadali ich a potom pokojne odpochodovali za
ostatnými.

LTak čo je s ich hlavami?“ skn'kla Kráľovná.

„Sú preč, vaša výsosť!“ skríkli aj vojaci.,

„Dobre!“ skríkla Kráľovná. „Vieš hrať kroket?“

Vojaci hľadeli na Alicu, lebo zrejme sa pýtala jej.

„Viem!“ skríkla Alica.

„Tak pod s nami!“ zrevala Kráľovná. Alica sa pridala
k sprievodu a tŕpla, čo bude ď alej.

„Dnes je — dnes je veľmi pekne,“ placho sa ozval niekto
pri nej. Kráčala vedľa Bieleho králika a ten na ňu úzkostlivo
pokukoval.

67

Alica v krajine zázrakov

„Pekne je,“ povedala Alica. „Kde je Vojvodkyňa?“
„Pst! Pst!“ na to Králik ticho. Vystrašene sa poobzeral,
potom sa postavil na prsty a s ústami pri Alicinom uchu

zašepkal: „Odsúdili ju na smrť.“

„A za čo si to vyslúžila?“ dozvedala sa Alica.

„Vravíš, že si to zaslúžila?“ spýtal sa Králik.

„Nie, nie,“ povedala Alica. „Tak som to nemyslela.
Pýtala som sa, čo vyviedla, že ju idú popraviť?“

„Dala Kráľovnej zaucho —“ začal Králik. Alica vyprskla
smiechom. „Pst! Pst!“ šepkal Králik vyľakane. „Aby ťa
nepočula Kráľovná! Vieš, trocha sa oneskorila a Kráľovná

povedala -“
„Na svoje miesta!“ zrevala Kráľovná. Sprievod sa rozu-

 

Kráľovnina kroketové ihrisko

tekal na všetky strany, všetci do seba vrážali, no po chvíli
sa predsa len zoradili a hra sa začala.

Alica si povedala, že také čudné ihrisko v živote nevidela.
Všade samý hrboľ, všetko rozryté, namiesto kroketových
lopti či ek živé ježe, namiesto palíc živé plameniaky a bránky
robili vojaci, predklonení tak, že sa rukami i nohami opie—
rali o zem.

*Spočiatku mala Alica najväčšie ťažkosti s plameniakom.
Nevedela si s ním poradiť. Podarilo sa jej síce vopchať si jeho
trup pod pazuchu tak, že nohy mu viseli, no zakaždým, keď
mu narovnala krk a hlavou chcela odpáliť ježa, plameniak sa
obrátil a pozeral jej do tváre s takým zmäteným výrazom
v očiach, že ju to rozosmialo. Keď mu napravila hlavu a opäť
sa chystala udrieť, rozčúlil ju zasa jež, ktorý sa medzitým
rozvinul a liezol preč. Navyše zakaždým, keď chcela ježa
odpáliť, prekážal jej v tom hrbolec alebo ryha. Aj predklo-
není vojaci sa jednostaj vztyčovali a odchádzali na inú časť
ihriska, takže Alica čoskoro usúdila, že takto hrať je naozaj
veľmi ťažko.

Wšetci hráči hrali naraz, nik sa nedržal poradia, ustavične
sa škriepili pre ježe. Kráľovná sa tak rozzúrila, že takmer bez
prestania dupotala a jačala: „Zotnite mu hlavu!“ alebo
„Zotnite jej hlavu!“

Alici z toho bo o už úzko. S Kráľovnou sa síce ešte
nepovadila, no vedela, že k tomu môže dôjsť každú chvíľu.
„A čo potom so mnou bude?“ pomyslela si. „Akosi strašne
radi tu ľuďom stínajú hlavy, div, že tu' ešte niekto ostal
nažive.“

A tak hľadala spôsob, ako zmiznúť, rozmýšľala, či sa jej
podarí nepozorovane ujsť, a tu práve zbadala v povetrí akési
čudné zjavenie. Spočiatku ju to zmiatlo, ale keď sa lepšie
prizrela, videla, že to sa niekto škľabí, a hneď vedela: „Veď
je to mačka Škľabka, aspoň sa mám s kým porozprávať.“

„Ako sa máš?“ spýtala sa Mačka, keď jej pribudlo z úst

toľko, aby mohla rozprávať.

69

Alica v krajine zázrakov

Alica počkala, kým sa zjavia oči, a potom len prikývla.
„Nemá zmysel zhovárať sa s ňou,“ vravela si, „kým sa jej
nevynoria uši, alebo aspoň jedno.“ O chvíľu sa zjavila celá
hlava, a tak Alica postavila plameniaka na zem a rozhovorila
sa o hre; bola rada, že ju konečne niekto počúva. Mačka si
zrejme myslela, že už z nej vidieť dosť veľký kus, nuž viac sa
ani neobjavilo.

„To nie je nijaká hra,“ posťažovala sa Alica, „a tak
príšerne sa hádajú, že človek vlastného slova nepočuje.
Pravidlá podľa všetkého nepoznajú, a ak aj poznajú, nik ich
nedodržiava. Nevieš si predstaviť, aký je z toho zmätok, keď
je tu kadečo živé — napríklad bránka, do ktorej som práve

% ninth chapter
\section{Príbeh Falošnej korytnačky}

NEVIEŠ si predstaviť, srdiečko moje, aká som rada,

„ že ťa zasa vidím,“ povedala Vojvodkyňa, nežne chy-

tila Alicu pod pazuchu a tak spolu odchádzali z ihriska.

Alica bola rada, že Vojvodkyňa má dobrú náladu;

v kuchyni bola pravdepodobne taká zlostná iba od toho
korenia.

„Ked ja budem Vojvodkyňou,“ povedala si (ale neznelo
to veľmi presvedčivo), „nebudem mať V kuchyni ani zrnko
korenia. Polievka je výborná aj bez neho. Možno práve od
korenia bývajú ľudia podráždení,“ rozmýšľala ďalej celá
natešená, že prišla na nový objav, „a od octu kyslí, od
harmančeka horkí — a — a od cukríkov a podobných
sladkostí sa deti sladko usmievajú. Keby si to dospelí
uvedomili, iste by s cukríkmi tak nežgrlošili.“

Na Vojvodkyňu medzitým úplne zabudla a skoro sa
naľakala, keď sa jej ozvala pri samom uchu. „Ty o niečom
rozmýšľaš, dušička, a zabúdaš hovoriť. V tejto chvíli ne-
viem, aké z toho vyplýva ponaučenie, ale veď ja si spome-

niem.“
„A čo keď z toho nijaké ponaučenie nevyplýva?“ odvážila
sa podotknúť Alica.

„Ale, ale, dieťa!“ povedala Vojvodkyňa. „Zo všetkého
vyplýva nejaké ponaučenie, len ho treba nájsť.“ A pritisla sa
k nej ešte tuhšie.

Alica nebola jej blízkosťou práve nadšená. Po prvé preto,
lebo Vojvodkyňa bola veľmi škaredá, po druhé preto, lebo
bradou, nepríjemne špicatou, dosahovala presne po Alicino
plece. Nechcela však byť nezdvorilá a znášala to, ako sa
dalo.

75

Alica v krajine zázrakov

„Zdá sa, že hra je teraz zaujímavejšia,“ povedala, aby reč
nestála.

„Tak je,“ povedala Vojvodkyňa, „a z toho plynie
ponaučenie, že svetom hýbe láska.“

„Ktosi povedal,“ zašepkala Alica, „že svet sa hýbe vtedy,
keď sa každý stará, o čo sa má stara .“

„Hej! Veď to je takmer to isté,“ povedala Vojvodkyňa,
zaryla sa špicatou bradou do Alicinho pleca a dodala:
„Z toho plynie ponaučenie: Dávaj pozor na zmysel, slová
ti vyplynú samy.“

„Ako rada vo všetkom hladá ponaučenie,“ pomyslela si
Alica.

„Iste sa čuduješ, prečo ťa nechytím okolo pása,“ ozvala sa
Vojvodkyňa po chvíli. „Len preto, lebo neviem, akú má
povahu tvoj plameniak. Mám to skúsiť?“

„Mohol by vás uštipnúť, “ varovala ju Alica, lebo ani
najmenej netúžila po tom, aby to Vojvodkyňa skúsila.

„To je pravda,“ povedala Vojvodkyňa, „horčica a pla-
meniaky štípu. Z čoho plynie ponaučenie: Vrana k vrane
sadá, rovný rovného si hľadá.“

„Ibaže horčica nie je vták,“ podotkla Alica.

„Správne ako zvyčajne,“ povedala Vojvodkyňa. „Ako
jasne sa vyjadruješ.“

„Je to nerast, ak sa nemýlim,“ pokračovala Alica.

„Isteže,“ povedala Vojvodkyňa, ktorá sa zrejme rozhod-
la súhlasiť so všetkým, čo Alica povie. „Tu neďaleko je veľká
horčičná baňa. A z toho plynie ponaučenie: Aj najväčšia
baňa sa raz preberie.“

„Už viem!“ poslednú vetu Alica ani nepočula. „Je to
rastlina. Nevyzerá ako rastlina, ale je to rastlina.“

„Celkom s tebou súhlasím,“ povedala Vojvodkyňa, „a
z toho plynie ponaučenie: Buď taká, aká by si sa chcela
zdať. Alebo ak chceš, aby som to povedala jednoduchšie:
Nikdy si nepredstavuj , že si iná, než aká by si sa mohla zdať
iným, a že to, aká si, bola si či mohla si byť, bolo iné než to,
aká by si sa bola mohla zdať iným.“

76

Príbeh Falošnej korytnačky

„Azda by som to lepšie pochopila,“ povedala Alica veľmi
zdvorilo, „keby som si to napísala. Keď mi to takto vravíte,
celkom tomu nerozumiem.“

„To je ešte nič v porovnaní s tým, ako by som ti to vedela
povedať, keby som chcela,“ odpovedala natešená Vojvod-
kyňa.

„Len sa, prosím vás, neunúvajte povedať mi to ešte
obšírnejšie,“ prosila Alica.

„Aké unúvanie!“ povedala Vojvodkyňa. „Všetko, čo
som doteraz povedala, ti dávam do daru.“

„To je veru lacný dar!“ pomyslela si Alica. „Ešte dobre,
že sa také dary nedávajú na narodeniny!“ Ale nahlas sa to
neodvážila povedať.

„Zasa rozmýšľaš?“ spýtala sa Vojvodkyňa a znova sa
zaryla špicatou bradou do Alicinho pleca.

„Mám právo rozmýšľať,“ povedala Alica ostrým hlasom,
lebo Vojvodkyňa ju už začínala dožierať.

„To je asi také právo,“ povedala Vojvodkyňa, „aké majú
prasce na lietanie; a z toho plynie ponau —“

Vtom Vojvodkyni, na Alicino veľké prekvapenie, zamre—
li slová na perách, obľúbené slovo ,ponaučenie' už nedopo-
vedala a ruka, ktorá ju držala pod pazuchou, sa chvela. Alica
zodvihla hlavu a pred nimi stála Kráľovná so založenými
rukami a mračila sa ani pred búrkou.

„Aký krásny deň, vaša Výsosť,“ začala Vojvodkyňa
tichým hlasom.

„Naposledy vás varujem,“ zrúkla Kráľovná a dupla
nohou, „alebo zmiznete vy, alebo vaša hlava - ato skôr ako
narátam do troch! Tak si vyberte!“

Vojvodkyňa si vybrala, a už jej nebolo.

„Pokračujme v hre,“ povedala Kráľovná Alici; tá sa od
strachu nezmohla ani na slovíčko, len pomaly kráčala za
Kráľovnou na kroketové ihrisko.

Ostatní hostia využili Kráľovninu neprítomnosť a odpočí—
vali v tieni. Len čo ju zazreli, ozlomkrky sa vracali k hre.

77

Alica v krajine zázrakov

Kráľovná idúcky utrúsila, že kto sa čo len o sekundu
oneskorí, príde o život.

Hrali teda ď ale j. Kráľovná sa neprestajne hádala s hráčmi
a vykrikovala: „Zotnite mu hlavu!“ a „Zotnite jej hlavu!“
Tých, nad ktorými vyniesla ortieľ, vojaci hneď zatýkali,
atak, pravdaže, nemohli robiť bránky - takže po polhodine
alebo tak nejako nebolo na ihrisku ani jednej bránky a všetci
hráči okrem Krála, Kráľovnej a Alice sedeli vo väzení,
odsúdení na smrť.

Kráľovná, celá zadychčaná, nechala hru tak a spýtala sa
Alice: „Videla si už Falošnú korytnačku?“

„Nie,“ povedala Alica, „ani neviem, čo to je - Falošná
korytnačka.“

„Robí sa z nej falošná korytnačia polievka,“ vysvetľovala
Kráľovná.

„Nikdy som ju nevidela, ani o nej nepočula,“ povedala
Alica.

„Poď, ideme za ňou,“ povedala Kráľovná, „nech ti
rozpovie svoj príbeh.“

 

Príbeh Falošnej korytnačky

Keď odchádzali, Alica ešte začula Kráľa, ako vraví tichým
hlasom: „Všetkým udeľujem milosť.“ „No, aspoň to,“
povedala si, lebo z toľkých popráv, čo nariadila Kráľovná,
bola už celá nešťastná.

[čoskoro natrafili na Grifóna; spal na slnku. (Ak neviete,
ako vyzerá Grifón, pozrite si ho na obrázku.) „Vstávaj,
leňoch,“ povedala Kráľovná, „a zaveď túto slečinku k F aloš—
nej korytnačke, nech jej rozpovie svoj príbeh. Ja sa musím
vrátiť, aby som dozrela na niekoľko popráv, ktoré som
nariadila.“ Odišla a nechala Alicu s Grifónom osamote.
Alici sa neveľmi páčilo to čudné stvorenie, no usúdila, že
zostať s ním nie je o nič nebezpečnejšie než ísť za tou
zlostnou Kráľovnou. A tak čakala, čo bude ďalej.

Grifón si sadol a pretieral si oči. Sledoval Kráľovnú, kým
nezmizla, a potom sa uškrnul: „Je na smiech,“ povedal
spola sebe, spola Alici.

„Kto je na smiech?“ spýtala sa Alica.

„Kto? Nuž ona!“ povedal Grifón. „To všetko si iba
namýšľa. Veď oni aj tak nikdy nikoho nepopravia. Tak
poď!“

„Tu je každý samé ,Poď!“ pomyslela si Alica a vliekla sa
za ním. „V živote mi nik toľko nerozkazoval!“

Prešli kúsok a hned neďaleko zazreli Falošnú korytnačku,
ako sedí smutná a opustená na skalisku. Keď podišli bližšie,
počula ju Alica vzdychať, akoby jej malo srdce puknúť.
Veľmi jej bolo Korytnačky ľúto. „Prečo sa tak trápi?“
spýtala sa Grifóna. A Grifón jej odpovedal skoro tými istými
slovami ako pred chvíľou: „To si iba namýšľa, veď nijaký
žiaľ nemá. Poď!“

Falošná korytnačka si ich prezerala veľkými zaslzenými
očami, ale nič nevravelaJ

„Táto slečinka, rozumieš,“ vraví jej Grifón, „by si rada
vypočula tvoj pn'beh.“

„Porozprávam jej ho,“ ozvala sa Falošná korytnačka
hlbokým, dutým hlasom. „Sadnite si obaja, no kým neskon-
čím, nechcem počuť ani slovo.“

79

Alica v krajine zázrakov

Tak si sadli a chvíľu bolo ticho. Alica si pomyslela:
„Neviem, ako chce skončiť, ked ani nezačala.“ Ale trpezlivo
čakala.

„Kedysi,“ povedala napokon Falošná korytnačka a hl—
boko si vzdychla, „som bola pravá Korytnačka.“

A znova nastalo dlhé ticho, tu a tam ho narušilo iba
Gn'fónovo hdžkŕŕŕ! a Korytnačkino sr'dcervúce vzlykanie.

Alica už-už chcela vstať a povedať: „Ďakujem vám za váš
zaujímavý príbeh,“ ale akosi sa nevedela striasť myšlienky,
že to ešte nie je všetko, tak len ticho sedela a mlčala'.

„Keď sme boli malí,“ pokračovala konečne Falošná
korytnačka, teraz už pokojnejšie, aj keď z času na čas ešte
zavzlykala, „chodili sme do morskej školy. Učila nás tam
stará korytnačka a my sme ju nazvali Nazučila.“

„Prečo ste ju volali Nazučila, keď to nebola Nazučila?“
spýtala sa Alica.

„No preto, lebo nás učila, ty hlupaňa!“ zlostila sa Falošná
korytnačka.

„Hanbil by som sa pýtať na také samozrejmosti,“ dodal
Grifón a potom obaja mlčky sedeli a hľadeli na chuderu
Alicu, ktorá by sa bola najradšej pod zem prepadla.
Napokon riekol Grifón Korytnačke: „Tak do toho, starká.
Nech to netrvá celý deň,“ a Falošná korytnačka pokračo—
vala:

„Chodili sme teda do školy v mori, aj keď mi to možno
neuveríte -“

„Nepovedala som, že vám neverím,“ prerušila ju Alica.

„Povedala,“ tvrdila Falošná korytnačka.

„Ty čuš!“ okríkol Grifón Alicu, skôr ako otvorila ústa.
Falošná korytnačka pokračovala.

fĎostali sme to najlepšie vzdelanie — veď si len predstav-
te, chodili sme do školy každý deň.“

„Aj ja som chodila do školy každý deň,“ povedala Alica.
„Tak sa tým toľko nechvascite!“

„A mali ste aj nepovinné predmety?“ neistým hlasom sa  
spýtala Falošná korytnačka.

80

Alica v krajine zázrakov

„Mali sme,“ povedala Alica, „francúzštinu a hudobnú
výchovu.“

„A bielizeň?“ spýtala sa Korytnačka.

„To nie,“ ohradila sa Alica.

„Och, tak tá vaša škola za veľa nestála!“ Falošnej
korytnačke sa očividne uľavilo. „V našej škole bolo pri
platení školských poplatkov jasne uvedené: francúzština,
hudobná výchova a bielizeň -— zvláštny príplatok.“

„Ale veď bielizne ste vela nepotrebovali,“ namietla Alica,
„keď ste žili na morskom dne.“

„Však som si to ani nemohla dovoliť,“ vzdychla si Falošná
korytnačka. „J a som chodila iba na povinné predmety.“

„To boli ktoré?“ vyzvedala sa Alica.

„No predovšetkým pichanie a číhanie,“ odpovedala
Korytnačka, „a potom aritmetika a rozličné počtové úkony,
ako je štípanie, odtínanie, nasolenie a deseniřý

„O nasolení som ešte nepočula,“ odvážila sa povedať
Alica. „Čo je to?“

Grifón od úžasu zdvihol laby. „Ty si ešte nepočula
o nasolení?!“ vykríkol. „Ale čo je nacukrenie, to dúfam
vieš?“

„Áno,“ povedala Alica váhavo, „to je — keď sa niečo
posype cukrom.“

„Správne. Ak ešte ani teraz nevieš, čo je to nasolenie,
tak si obyčajný truľo,“ povedal Grifón.

Alica už nemala odvahu klásť ďalšie otázky, tak sa znova
obrátila na Falošnú korytnačku: „Čo ste sa ešte učili?“

„No, ešte sme mali hystériu,“ rátala Korytnačka na
labách, „hystériu starovekú a novovekú, morepis, ďalej
kriesenie — to nás učil starý úhor, chodieval iba raz do
týždňa —- a mali sme aj lisovanie a spaľovanie.“

„Čo ste robili na jeho hodinách?“

„To ti ja už ťažko ukážem,“ povedala Falošná korytnačka.
„Na to som už trocha neohybná. A Grifón sa to neučil.“

„Nemal som čas,“ povedal Grifón. „Ja som sa viac
venoval klasickým jazykom; u toho starého kraba.“

82

Príbeh Falošnej korytnačky

„K tomu som ja vôbec nechodila,“ povedala Falošná
korytnačka a vzdychla si. „Vraj vyučovegsätinu a škreč-
tinu. “ “

w„Vyučoval, vyučoval, to hej,“ povedal Grifón, pre zmenu
si vzdychol zasa on a obaja si zakryli tváre labami.

„A koľko hodín denne ste mali?“ spýtala sa Alica, aby
obrátila reč na iné.

„Prvý deň desať hodín,“ povedala Falošná korytnačka,
„druhýdeň deväťa tak ďalej.“

„Aké čudné vyučovanie!“ zvolala Alica.

„Prečo čudné? Veď vyučovanie, to je vlastne vylučova—
nie,“ vysvetľoval Grifón, „každý deň vylúčite jednu hodinu,
a tak si pekne vylučujete, a čím viac vylučujete, tým
rýchlejšie vyučujete, chápeš už?“

To bola pre Alicuíelkom nová myšlienka. Chvíľu o nej
rozmýšľala a zrazu vyhŕkla: „Jedenásty deň ste mali teda
voľno?“

„Pravdaže,“ povedala Falošná korytnačka.

„A čo ste robili dvanásty deň?“ vyzvedala sa Alica.

„Dosť už o vyučovaní,“ rázne zasiahol Grifón. „Teraz jej
povedz niečo o tom, ako ste sa hrali.“

% tenth chapter
\section{Račia štvorylka}

,ALOŠNÁ korytnačka si sťažka vzdychla a opakom
laby si pretrela oči. Pozrela na Alicu a chcela niečo
povedať, ale plač ju celkom zadúšal. „Akoby jej kósť
zaskočila,“ povedal Grifón a začal ju triasť a búchať do
chrbta. Napokon predsa len našla reč a so zaslzenou tvárou
rozprávala:

„Ty si v mori asi dlho nežila,“ („To veru nie,“ povedala
Alica) „a s Rakom si sa pravdepodobne nikdy nezoznámi-
la,“ (Už-už chcela Alica podotknúť: „Raz som jedného
ochutnala,“ ale vzápätí sa zháčila: „Nie, nikdy.“) „a tak
nemáš potuchy, aká krásna je račia štvorylka.“

„Nie, nemám,“ prisvedčila Alica. „Aký je to tanec?“

„No, to sa tanečníci najprv postavia na pobreží do radu,“
ozval sa Grifón.

„Do dvoch radov!“ skríkla Falošná korytnačka. „Tulene,
korytnačky, lososy a tak dalej. Potom sa odpracú z cesty
všetky medúzy —“

„Zvyčajne to chvíľu trvá,“ skočil jej do reči Grifón.

„— urobia sa dva kroky vpred —“

„Za tanečníka má každý morského raka!“ skríkol
Grifón.

„To je jasné,“ povedala Falošná korytnačka. „Dva kroky
vpred so svojím tanečníkom —“

„— potom sa tanečníci vymenia a cúvne sa dva kroky
vzad,“ pokračoval Grifón.

„Potom, chápeš,“ pokračovala Falošná korytnačka, „sa
hodia —“

„Raky!“ skríkol Grifón a vyskočil.

„— čo najďalej do mora —“

84

 

Račín štvorylka

„A pláva sa za nimi!“ zvreskol Grifón.

„V mori sa urobí kotrmelec!“ skríkla Falošná korytnačka
a skákala ako divá.

„Raky sa znova vymenia!“ reval Grifón z plného hrdla.

„Pláva sa naspäť na breh a — to je prvá figúra,“ povedala
Falošná korytnačka a hlas sa jej náhle zlomil; obe stvorenia,
čo do tej chvíle vyvádzali ani pojašené, sedeli tu odrazu
zasa smutné a tiché a pozerali na Alicu.

„Je to iste veľmi pekný tanec,“ bojazlivo sa ozvala
Alica.

„Chcela by si aspoň kúsok z neho vidieť?“ spýtala sa
Falošná korytnačka.

„Veľmi rada,“ povedala Alica.

„Skúsme teda prvú figúru,“ povedala Falošná korytnačka
Grifónovi. „Bez rakov sa hádam aj zaobídeme. Kto bude
spievať?“

„Spievaj ty,“ povedal Grifón. „Ja som zabudol slová.“

A tancovali vážne okolo Alice, a keď prechádzali p0pri
nej, stúpali jej na prsty, prednými labami si udávali tempo
a Falošná korytnačka ťahavo, smutne spievala:

„Pridaj, slimák, do kroku!“ belica mu ruku zviera.
„Delfín za mnou sa tak náhli, že mi päty dooškiera.
Pozri, ako sa sem zbehli —- korytnačky, morské raky,
všetci si chcú zatancovať, a nie tanec hocijaký. “

Pôjdeš, alebo nepôjdeš do tanca s nimi?
Pôjdeš, alebo nepôjdeš tancovať?
Pôjdeš, alebo nepôjdeš. do tanca s nimi?
Pôjdeš, alebo nepôjdeš tancovať?

„Spolu s rokmi ťa tu zdvihnú, vyhodia na šíre more.
Aká rozkoš, aké blaho, vyletieť tak zrazu hore!“
„Tak ďaleko?“ váha slimák. „ Tak ďaleko sa ja bojim. “

A že nechce, že ďakuje, že on a tanec nestojí.

85

 

Alica v krajine zázrakov

Že nepôjde, že nemôže do tanca s nimi,
že nepôjde, že nemôže tancovať.
Že nepôjde, že nemôže do tanca s nimi,
že nepôjde, že nemôže tancovať.

„Aká dialka? Čo ťa mata?“ šupinata' družka na to.
„Breh je aj na tamtej strane, s pieskom ako čisté zlato;
ak si ďalej od jídnebo, druhý breh je o to bližší,  
nuž'nebledni, milý slimák, zatancovať sme si prišli.“

Pôjdeš, alebo nepôjdeš do tanca s nami,
pôjdeš, alebo nepôjdeš tancovať?
Pôjdeš, alebo nepôjdeš do tanca s nami,
pôjdeš, alebo nepôjdeš tancovať?

„Ďakujem, na pohľad je to veľmi zaujímavý tanec,“
vydýchla si Alica, že už konečne skončili. „A tá zvláštna
pieseň o belici sa mi veľmi páči.“

„Ach, belica,“ vzdychla si Falošná korytnačka, „ predpo—
kladám, že belicu poznáš?“

„Áno,“ povedala Alica, „často som ich vídala pri obe -“
no ešte včas sa zarazila.

„Neviem, kde leží Ob,“ povedala Falošná korytnačka,
„ale keď si ich tak často vídala, iste vieš, ako vyzerajú?“

„No — áno,“ odpovedala Alica zamyslene. „Chvostíky
majú v papuľke — a celé sú posypané strúhankou.“

„S tou strúhankou to nie je pravda,“ povedala Falošná
korytnačka, „more by im ju zmylo. Ale cthstíky — tie majú
naozaj v papuľke, a to preto, lebo —“ vtom Falošná
korytnačka zívla a zatvorila oči.

„Povedz jej prečo a všetko ostatné,“ obrátila sa na
Grifóna.

„Preto, lebo chceli tancovať s morskými rakmi,“ povedal
Grifón. „A tak ich hodili do mora. A tak padali hrozne,
hrozne dlho. A tak si chvostíky vopchali do papuľky. A tak
si ich už nemohli vytiahnuť. A tak to už vieš.“

86

Račín šlvorylka

„Ďakujem,“ povedala Alica, „je to veľmi zaujímavé.
Doteraz som o beliciach veľa nevedela.“

„Ak chceš, poviem ti o nich viac,“ povedal Gn'fón. „Vieš
napríklad, prečo sa volajú belice?“ .

„Nikdy som o tom nerozmýšľala,“ povedala Alica.
„Prečo?“

„Lebo sa nimi leštia topánky a čižmy,“ smrteľne vážne
povedal Grifón.

Alicu to zmiatlo. „Topánky a čižmy?“ opakovala začudo—
vane.

„No — a ty si čím leštíš topánky?“ spýtal sa Grifón. „No
— od čoho sa ti ligocú?“

Alica si pozrela na topánky a chvíľu uvažovala. „Myslím,
že od černidla.“

„Nuž a pod morom sa topánky leštia bielidlom — bielid-
lom z belíc,“ dodal Grifón hlbokým hlasom. „Tak už to

Vies.“

„A z čoho sa tam topánky vyrábajú?“ spýtala sa Alica
zvedavo.

„Z podustiev a z úhorov, z čoho iného?!“ podráždene
jej odpovedal Grifón. „To vie každý morský rak.“ '

„Keby som ja bola belica,“ odpovedala Alica, ktorej ešte
chodila po rozume tá pieseň, „povedala by som slimákovi:
Keď máš strach, uteč - a rýchlo! Nechceme ťa tu! Načo sa
toľko prosiť?“

„Vylúčené,“ povedala Falošná korytnačka. „Aj tak by
neušiel.“

„Ako to môžete vedieť?“ čudovala sa Alica.

„Čo sa vlečie, neutečie,“ riekla Falošná korytnačka.
„Nepoznáš to porekadlo? No a slimák sa predsa vlečie,
nie?“

Kým Alica nad tým uvažovala, ozval sa Grifón:

„Teraz nám porozprávaj ty niečo zo svojich dobrodruž-
stiev!“

„Mohla by som vám rozprávať, aké dobrodružstvá som
zažila — ale iba od dnešného rána,“ nesmelo povedala

87

Alica v krajine zázrakov

Alica, „rozprávať o včerajšku nemá zmysel, včera som bola
celkom iná osoba.“

„To nám musíš vysvetliť,“ povedala Falošná koryt-
načka.

„Nie, nie! Najprv dobrodružstvá,“ netrpezlivo nástojil
Grifón, „vysvetľovanie vždy trvá veľmi dlho.“

A tak im Alica začala rozprávať, čo zažila od chvíle, keď
po prvý raz zbadala Bieleho králika. Spočiatku bola trochu
nervózna, lebo obe stvorenia sa na ňu tlačili, každé z jednej
strany, s vypleštenými očami a ústami dokorán, ale potom
sa spamätala. Poslucháči ani nemukli, kým nedošla k tomu,
ako recitovala Húseničiakovi báseň Starý ste, otče…, ale
po jazyku sa jej plietli akési iné slová. Tu saFalošná korytnač—
ka zhlboka nadýchla a povedala: „To je akési čudné!“

„Čudnejšie to už ani nemôže byť,“ povedal Grifón.

„Všetky slová poplietla,“ zamyslene povedala Falošná
korytnačka. „Mala by skúsiť aj nám niečo zarecitovať.
Povedz jej, nech začne!“ A pozrela na Grifóna, akoby ten
mal nad Alicou nejakú moc.

„Vstaň a zarecituj nám Do trnava ma uvarih',“ povedal
Grifón.

„Tieto stvorenia by ustavične len rozkazovali,“ pomyslela
si Alica. „Nútia človeka odriekať úlohy — celkom ako
v škole.“ Voľky-nevoľky vstala a začala recitovať, no hlavu
mala ešte vždy plnú račej štvorylky, takže ani nevedela, čo
vraví, a aj tieto verše zneli akosi čudne:

Do trnava ma uvarili, ej, veru tak,

spieva si na morskom brehu udatný Rak.
Teraz nech mi vlasy cukria čistým prosom,
gombičky a remeň si viem zrovnať nosom,
ako kačky parádnice, ej, veru tak,

tak budem vykrúcať špice, nie naopak.

Veselí sa v suchom piesku udatný Rak,
od žraloka pohŕdavo odvracia zrak.

88

Alica v krajine zázrakov

Až keď príliv znova začne máčať skaly,
trasie sa is klepetami celkom malý.
Žraloka si opäť cení náš milý Rak,
najmä keď je nevarený, ej, veru tak.

„Neznie to tak, ako som sa to učil za mladi,“ povedal
Grifón.

„Ja som tú báseň v živote nepočula,“ povedala Falošná
korytnačka, „ale je to číry nezmysel.“

Alica nepovedala nič, sedela s tvárou v dlaniach a roz-
mýšľala, či okolo nej bude ešte niekedy niečo normálne
a prirodzené.

„Keby mi to niekto vysvetlil. . .“ nadhodila Falošná
korytnačka.

„Ona to nevie vysvetli ,“ prerušil ju Grifón. „Zarecituj
inú báseň!“

„Ale ako je to s tými gombičkami?“ nástojila Falošná
korytnačka. „Ako sa zrovnávajú nosom?“

„To je základný postoj pn' tanci,“ povedala Alica, lebo
bola z toho už celkom popletená a chcela to nejako
zahovoriť.

„Zarecituj inú báseň!“ opakoval Grifón. „Tú, ktorá sa
začína Čo všeličo človek vidí.“

Alica sa neodvážila odporovať, hoci vedela, že sa to zasa
nevydarí, a tak recitovala traslavým hlasom:

Čo všeličo človek vidí, keď po sade chodí:

tuhľa Pardál so Sovou si prichystali hody.

Mäso, kôrka, šťava z mäsa - všetko podľa manier
zožral Pardál; Sove zostal iba prázdny tanier.
Dovolil jej spratať príbor — keď nebolo ině —

a do vrecka mu ho vopchať — ako po hostine.
Keď zazrel nôž a vidlička, dostal vám hlad znovu,
zavrčal a s chuťou zožral i chuderu — - -

„Recitovať také táraniny nemá zmysel,“ prerušila ju

90

Račín štvorylkn

Falošná korytnačka, „keď nám to poriadne nevysvetlíš!
V živote som nepočula bláznivejšie verše.“

„Vari naozaj s tým radšej prestaň,“ povedal Grifón
a Alica mu veľmi rada vyhovela.

„Čo keby sme skúsili druhú figúru račej štvorylky?“
navrhoval Grifón. „Alebo ti má Falošná korytnačka niečo
zaspievať?“

„Áno, zaspievať, zaspievať, ak bude Falošná korytnačka
taká láskavá,“ prosila Alica tak žiadostivo, že Grifón
povedal urazene: „Hm. Proti gustu žiaden dišputát. Zaspie-
vaj jej teda ódu na polievku, starká!“

Falošná korytnačka si zhlboka vzdychla, a hoci sa
zadúšala od plaču, začala spievať:

„ó, moku nebeský! Hustá a zelená
čakáš ma horúca v náručí taniera.
Komu by pred tebou neklesli kolená?
Ktože ti odolá, kráľovská večera?
Hustá a zelená voniaš mi z taniera.

 

ó, polievka! 0- 6, polievka!
Nič lepšie nepoznám.
O, polievka! 0- -ó, polievka!
Kto postaví ti chrám?

ó, moku nebeský! Kto raz ťa okúsi,
zabudne na ryby, na misy s divinou,
posledný halier dá, veď v jeho oku si
odteraz jedinou vábivou vidinou.
Kto raz ťa okúsi, pohrdne divinou.

ó, polievka! Ď-ó, polievka!
Nič lepšie nepoznám.

ó, polievka! ó-ó, polievka!
Kto postaví ti chrám?“

91

„Refrén ešte raz!“ skríkol
Grifón a Falošná korytnačka práve
začala, keď sa v diaľke ozvalo:

„Pojednávanie sa začína!“

„Poď!“ skríkol Grifón, zdrapil
Alicu za ruku a ťahal ju preč, ani
nevyčkal koniec piesne.

„A čo je to za súd?“ spýtala sa
Alica zadychčaným hlasom, ale
Grifón povedal len: „Pod!“ a bežal
ešte rýchlejšie. S vánkom k nim
z diaľky doletovala čoraz slabšia
ozvena melancholického refrénu:

„ Č, polievka! ó-ó, polievka!
Kto postaví ti chrám?“

% eleventh chapter
\section{Kto ukradol koláče?}

B EĎ ta prišli, Srdcový kráľ so Srdcovou kráľovnou
už sedeli na tróne a okolo nich sa hrčil veľký dav

- všakovaké drobné vtáctvo, zvieratká a všetky karty.
V popredí stál Dolník, ruky v reťaziach, po boku vojaci,
ktorí ho strážili. Pri Kráľovi stál Biely králik, v jednej ruke
držal trúbku, v druhej pergamenový zvitok. Uprostred
súdnej siene bol stôl a na ňom veľká misa koláčov. Vyzerali
také chutné, že sa Alici začali Sliny zbiehať. „Keby už bolo
po pojednávaní,“ pomyslela si, „a podávalo sa občerstve-
nie!“ Zatiaľ to nebolo na programe, nuž sa obzerala okolo
seba, aby jej čas rýchlejšie ubehol.

Na súde Alica ešte nebola, ale o pojednávaniach čítala
v knihách a veľmi ju tešilo, keď zistila, že vie, ako sa čo volá.
„Toto je sudca,“ povedala si, „lebo má veľkú parochňu!“

Sudcom, mimochodom, bol Kráľ; a keďže korunu si
založil na parochňu, necítil sa veľmi pohodlne, ba ani mu to
veľmi nepristalo.

„Toto je lavica pre porotcov,“ pomyslela si Alica, „a tých
dvanásť stvorení (musela ich nazvať stvoreniami, lebo boli
medzi nimi zvieratá aj niekoľko vtákov), to sú zrejme
porotcovia.“ Posledné slovo si zopakovala dva i tri razy
a bola hrdá, lebo si myslela — a oprávnene -, že iba veľmi
málo dievčat v jej veku pozná zmysel tohto slova. Ale takisto
by sa mohlo povedať členovia poroty.

Dvanásť porotcov usilovne písalo na bridlicové tabuľky.

„Čo to píšu?“ šepkala Alica Grifónovi. „Pojednávanie
sa ešte nezačalo, tak nemajú čo zapisovať.“

„Zapisujú si svoje mená,“ pošepkal jej Grifón, „le-

93

Alica v krajine zázrakov

bo sa boja, že ich zabudnú skôr, ako sa pojednávanie
skončí.“

„Chumaji!“ rozhorčene vybuchla Alica, ale hneď stíchla,
lebo Biely králik zavolal: „Ticho v súdnej sieni!“ a Kráľ si
založil okuliare a prezeral si miestnosť, aby zistil, kto sa to
ozval.

Alica videla, ani čo by im hľadela ponad plecia, že všetci
porotcovia si zapisujú na bridlicové tabuľky slovo ,chumaji“,
jeden z nich dokonca nevedel, či sa ,chumaji' píšu na konci
s mäkkým ,i“, alebo s tvrdým ,y', a pýtal sa na to suseda.
„Kým sa pojednávanie skončí, budú na tých tabuľkách
pekné nezmysly,“ pomyslela si Alica.

Jeden porotca mal grifeľ , ktorý škrípal. Alica to nezniesla,
obišla teda sieň, stala si mu za chrbát a pri najbližšej
príležitosti mu grifeľ vzala. Urobila to tak šikovne, že
chudák porotca (bol to jašteričiak Vilo) vôbec nechápal, čo
sa stalo. Chvíľu grifeľ márne hľadal a zvyšok pojednávania
si potom zapisoval prstom. Nebolo to, pravdaže, na nič,
lebo na tabuľke to nebolo poznať.

„Hlásateľ, prečítajte obžalobu!“ povedal Kráľ.

Biely králik tri razy zatrúbi], rozvinul pergamenový zvitok
a čítal:

„Srdcová kráľovná napiekla koláčov

s makom a tvarohom,

Srdcový dolník ich uchmatol z pekáčov
a zmizol za rohom. “

„Poraďte sa o rozsudku,“ povedal Kráľ porote.

„Ešte nie, ešte nie!“ náhlivo mu skočil do reči Králik.
„Predtým treba ešte prerokovať veľa otázok.“

„Tak predvolajte prvého svedka,“ povedal Kráľ; Biely
králik zatrúbi] a zvolal: „Prvý svedok!“

Prvým svedkom bol Klobučník. Vošiel do súdnej siene
v jednej ruke so šálkou čaju, v druhej s krajcom maslového
chleba. „Prosím vašu výsosť o prepáčenie,“ začal, „že

94

Kto ukradol koláče?

prichádzam takto, ale prišli po mňa a ja som sa nestihol
naolovrantovať.“

„Olovrant si ešte mal stihnúť,“ povedal Kráľ. „Kedy si
začal olovrantovať?“

Klobučník pozrel na Aprílového zajaca, ktorý šiel s ním až
na súd a popod pazuchu viedol Sedmospáča. „Ak sa
nemýlim, bolo to štrnásteho apríla,“ povedal.

„Pätnásteho,“ povedal Aprílový zajac.

„Šestnásteho,“ povedal Sedmospáč.

„Zapíšte to!“ povedal Kráľ porote; porotcovia si blesku-
rýchle poznačili na tabuľky všetky tri dátumy, sčítali ich
a výsledok prerátali na koruny a haliere.

„Klobúk dolu!“ povedal Kráľ Klobučníkovi.

„Nie je môj!“ povedal Klobučník.

„Ukradnutý!“ zvolal Kráľ; porotcovia na jeho pokyn
ihned spísali zápisnicu.

„Ja klobúky predávam,“ dodal Klobučník na vysvetlenie.
„Sám nijaký nemám. Som Klobučník.“

Tu si Kráľovná založila okuliare a premerala si Klobuční-
ka tak prísne, že pobledol a znervóznel.

„Vypovedaj!“ povedal Kráľ. „A netras sa, lebo ťa dám
na mieste popraviť.“

Kráľov výrok podľa všetkého nedodal svedkovi odvahy;
jednostaj prešľapoval z nohy na nohu, celý vyľakaný hľadel
na Kráľovnú a taký bol popletený, že namiesto z maslového
chleba si odhryzol zo šálky. V tej chvíli premkol Alicu čudný
pocit; hodnú chvíľu nevedela, čo je na príčine, napokon na
to prišla: znova sa zväčšovala. Najprv chcela vstať a odísť zo
sály, no potom sa rozhodla, že zostane, pokým tam bude mať
dosť miesta.

„Netlač sa tak!“ povedal Sedmospáč, ktorý sedel vedľa
nej.

„Nemôžem si pomôcť,“ povedala Alica skrúšene, „ras-
tiem.“

„Tu nemáš čo rásť,“ povedal Sedmospáč.

95

Alica v krajine zázrakov

„Netáraj nezmysly,“ povedala Alica smelšie, „dobre vieš,
že aj ty rastieš.“

„Áno, ale ja rastiem rozumne,“ povedal Sedmospáč, „nie
takou šialenou rýchlosťou.“ Namrzený vstal a prešiel na
druhú stranu súdnej siene.

Kráľovná po celý čas uprene hľadela na Klobučníka,
a práve keď sa Sedmospáč predieral sieňou, povedala
jednému súdnemu zriadencovi: „Prines mi zoznam spevá-
kov z posledného koncertu!“ Nato sa nešťastný Klobučník
tak roztriasol, že si striasol topánky z nôh.

„Vypovedaj!“ opakoval Kráľ nahnevane. „Lebo ťa dám
popraviť, či už máš strach, alebo nie.“

„Ja som taký úbohý človek, vaša výsosť,“ začal Klobučník
roztraseným hlasom, „ani ten čaj som nestihol dopiť - je to
vari týždeň alebo tak nejako — ani dojesť maslový chlieb, je
ho čoraz menej — a tie o—o—opletačky s olovrantom.“

„Aké o—o-opletačky?“ spýtal sa Kráľ.

„Nuž tie, ktorými sa začal náš olovrant,“ odpovedal
Klobučník.

„Samozrejme, že ,olovranť sa začína na ,o',“ osopil sa na
neho Kráľ. „Myslíš, že sa zhováraš s hlupákom? Hovor!“

„Ja som taký úbohý človek,“ pokračoval Klobučník, „a
všetky tie opletačky s olovrantom, ibaže Aprílový zajac
povedal —“

„Nepovedal,“ skočil mu do reči Aprílový zajac.

„Povedal si!“ tvrdil Klobučník.

„Popieram to!“ povedal Aprílový zajac.

„On to popiera,“ povedal Kráľ. „Túto časť vyne—
chajte!“

„V každom prípade Sedmospáč povedal -“ pokračoval
Klobučník a bojazlivo sa obzrel, či to aj SedmOSpáč
nepoprie, ale ten nič nepoprel, spal ako zabitý.

„Tak som si,“ pokračoval Klobučm'k, „odkrojil ešte
chleba a natrel si ho maslom —“

96

Kto ukradol koláče?

„Ale čo ten Sedmospáč povedal?“ spýtal sa jeden
z porotcov.

„Na to sa už nepamätám,“ povedal Klobučník.

„Rozpamätaj sa,“ prikázal mu Kráľ, „inak ťa dám
popraviť.“

Nešťastný Klobučník pustil na zem šálku čaju i maslový
chlieb a klesol na kolená: „Ja som taký úbohý človek, vaša
výsosť —“ začal.

„Najmä úbohý rečník,“ povedal Král.

Akési morča skn'klo Sláva mu!, ale také prejavy súdni
zriadenci ihned eliminovali. (Keďže je to dosť nezvyčajné
slovo, vysvetlím vám, čo urobili.

Mali plátenné vrece, ktoré sa
zaväzovalo na motúzik.
Morča doň strčili dolu
hlavou a potom si

naň sadli.)

 

Alica v krajine zázrakov

„Konečne som teda videla, ako sa to robí,“ pomyslela si
Alica. „Neraz čítam v novinách, že na konci pojednávania
,sa vyskytli ojedinelé prejavy súhlasu alebo nesúhlasu, ale
súdni zriadenci ich ihneď eliminovali“, a doteraz som
nevedela, čo sa za tým skrýva.“

„Ak o tom už viac nevieš, môžeš zísť dolu,“ pokračoval
Kráľ.

„Nižšie to už nejde,“ povedal Klobučník, „stojím na
dlážke.“

„Tak si sadni,“ povedal Kráľ.

A ďalšie morča skríklo Sláva mu!, no jeho prejav súhlasu
ihneď eliminovali.

„Morčatá sú teda odpísané,“ pomyslela si Alica. „Teraz
to už iste pôjde rýchlejšie.“

„Rád by som dokončil olovrant,“ povedal Klobučník
a uprel bojazlivý pohľad na Kráľovnú, ktorá si prezerala
zoznam spevákov.

„Môžeš íst,“ povedal Kráľ a Klobučník vyletel zo súdnej
siene, ani topánky si neobul.

„ - a vonku mu hneď zotnite hlavu,“ obrátila sa Kráľovná
na jedného dôstojníka, ale Klobučník sa stratil skôr, ako
dôstojník došiel k dverám.

„Predvolajte ďalšieho svedka,“ povedal Kráľ.

Ďalším svedkom bola Vojvodkynina kuchárka. V ruke
mala koreničku a Alica uhádla, kto je to, skôr než vkročila
do siene, lebo všetci ľudia pri dverách sa rozkýchali.

„Vypovedaj!“ vyzval ju Kráľ.

„Ešteže čo,“ povedala kuchárka.

Kráľ ustarostene pozrel na Bieleho králika a ten mu ticho
povedal: „Vaša výsosť, túto svedkyňu musíte podrobiť

krížovému výsluchu. “
„Nuž, keď musím, tak musím,“ zatváril sa Kráľ melancho—

licky, potom si založil ruky, zamračil sa na kuchárku, takže
mu oči celkom zapadli, a zahučal: „Z čoho sa robia
koláče?“

„Zväčša z korenia,“ povedala kuchárka.

98

Kto ukradol koláče?

„Z melasy,“ ozval sa za ňou ospanlivý hlas.

„Zdrapte tam toho Sedmospáča za golier!“ skríkla
Kráľovná. „Zotnite mu hlavu! Vyhodte toho plcha zo
súdnej siene! Eliminujte ho! Doštípte ho! Vyšklbte mu
fúzy!“

Kým Sedmospáča vyhadzovali, bol v sieni obrovský
zmätok, a ked konečne obnovili poriadok, kuchárky už
nebolo.

„Nič to zato,“ povedal král, akoby mu spadol kameň zo
srdca. „Predveďte ďalšieho svedka!“ A polohlasne povedal
Kráľovnej: „Drahá, tohto svedka musíte podrobiť krížové-
mu výsluchu vy. Mňa z toho už rozbolela hlava.“

Alica pozorovala Bieleho králika, ako sa nešikovne
prehrabuje v spisoch, a bola veľmi zvedavá na dalšieho
svedka, „— lebo veľa dôkazov ešte teda nemajú,“ povedala
si. Viete si predstaviť, ako ju prekvapilo, keď Biely králik
svojím piskľavým hlasom prečítal: „Ah'ca!“ '

% twelfth chapter
\section{Alicina výpoveď}

| U,“ skn'kla Alica taká rozčúlená, že celkom zabudla,

„ aká veľká narástla v poslednej chvíli, a vyskočila tak

prudko, že lemom sukne prevrátila lavicu pre porotu

a všetkých porotcov vysypala divákom na hlavy; ako tam

ležali a hrabali okolo seba, veľmi jej pripomínali guľaté

akvárium so zlatými rybkami, ktoré pred týždňom nešťast-
nou náhodou prevrhla.

„Ach, prepáčte!“ zvolala celá vydesená a zbierala ich tak
rýchlo, ako sa len dalo, lebo nehodu so zlatými rybkami mala
ešte čerstvo zapísanú v pamäti a marilo sa jej, že ak ich čo
najskôr nepozbiera a neposadí na lavicu pre porotu,
zahynú.

„Pojednávanie sa odkladá,“ povedal Kráľ veľmi vážne,
„kým všetci porotcovia nebudú na svojich miestach - ale
všetci,“ opakoval veľmi dôrazne a pri poslednom slove sa
na Alicu zamračil.

Alica pozrela na lavicu pre porotcov a zbadala, že v tej
rýchlosti posadila Vila dolu hlavou a to chúďa, neschopné
iného pohybu, teraz iba smutne kývalo chvostíkom. Ihneď
ho vzala a posadila Správne; „Nie že by na tom bohvieako
záležalo,“ povedala si, „veď či sedí tak, alebo onak,
pojednávanie to nijako neovplyvní.“

Len čo sa porotcovia trocha spamätali z otrasu a zriadenci
im pozbierali postrácané tabuľky a grifle, usilovne sa dali do
práce a celú nehodu dopodrobna zapisovali. Všetci okrem
Vila, ktorý bol ešte taký otrasený, že iba sedel s otvorenými
ústami a tupo zízal na povalu súdnej siene.

„Čo vieš o celej záležitosti?“ spýtal sa Kráľ Alice.

„Nič,“ odpovedala.

100

Alicina výpoveď

„Vôbec nič?“ nástojil Kráľ.
„Vôbec nič,“ opakovala Alica.
„To je veľmi dôležité,“ povedal Kráľ a obrátil. sa

k porotcom.
Už-už si to zapisovali na tabuľky, keď vtom sa ozval Biely

králik: „Vaša výsosť chcela azda povedať — nedôležité.“
Povedal to úctivo, no mračil sa pritom a strúhal na Kráľa
grimasy.

„Pravdaže som chcel povedať nedôležité,“ rýchlo sa
opravil Kráľ a ticho si mrmlal: „Dôležité — nedôležité
- nedôležité — dôležité —“ akoby skúšal, čo znie lepšie.

Niektorí z porotcov si zapísali ,dôležité“, iní ,nedôležité'.
Alica to videla, lebo stála celkom blízko. „Ale veď na tom

nezáleží,“ povedala si.

Vtom Kráľ, ktorý si už chvíľu usilovne čosi značil do
zápism'ka, zvolal: „Ticho!“ a prečítal odtiaľ: „Paragraf
štyridsiaty dmhý. Všetky osoby, ktoré merajú vyše kilomet-
ra, nech opustia súdnu sieň.“

Všetky oči sa upreli na Alicu.

„Ja nemeriam kilometer,“ povedala Alica.

„A vem meriaš,“ povedal Kráľ.

„Meriaš takmer dva kilometre,“ povedala Kráľovná.

„A keby aj, nepôjdem,“ povedala Alica, „nijaký taký
paragraf nejestvuje, vy ste si ho práve vymysleli.“

„Je to najstarší paragraf z celého zákonníka,“ povedal
Kráľ.

„Tak by to mal byť paragraf prvý,“ povedala Alica.

Kráľ zbledol a chytro zavrel zápisník. „Poraďte sa o roz-
sudku,“ povedal porote tichým, roztraseným hlasom.

„Máme tu ďalší dôkaz,“ Biely králik vyskočil zo stoličky
ani strela. „Ak dovolíte, vaša výsosť, na zemi sa práve našiel
tento papier.“

„Čo je na ňom?“ spýtala sa Kráľovná.

„Ešte som si ho neprezrel,“ povedal Biely králik, „ale
podľa všetkého je to list nejakého väzňa - no, niekomu.“

101

Alica v krajine zázrakov

„Ako ináč!“ povedal Kráľ. „Iba že by bol písaný — ni-
komu, a to sa, ako viete, nerobieva.“

„Aká je adresa?“ spýtal sa ktorýsi porotca.

„Nijaká,“ povedal Biely králik, „navrchu nie je nič
napísané.“ Medzitým papier rozložil a dodal: „Ani to
vlastne nie je list, je to báseň.“

„Je napísaná väzňovým rukopisom?“ spýtal sa iný po-
rotca.

„Nie, nie je,“ povedal Biely králik, „a to je na tom
najčudnejšie.“

(Porota bola z toho už celkom zmätená.)

„Tak teda napodobnil cudzí rukopis,“ povedal Kráľ.

(Tváre porotcov sa opäť vyjasnili.)

„Prosím, vaša výsosť,“ ozval sa Dolník, „ja som to nepísal
a nikto mi to nedokáže. Na konci nie je nijaký podpis.“

„Ak si to nepodpísal,“ povedal Kráľ, „tým horšie. Iste si
mal za lubom niečo nečestné, inak by si sa bol podpísal ako
každý poriadny človek.“

Ozval sa všeobecný potlesk; bola to prvá múdra myšlien-
ka, ktorú Kráľ v ten deň vyslovil.

„To je dôkaz, že je vinný,“ povedala Kráľovná, „a tak
mu hneď zot —“

„To nie je nijaký dôkaz,“ povedala Alica, „veď ani
neviete, čo v tých veršoch je.“

„Prečítaj ich!“ povedal Kráľ.

Biely králik si založil okuliare: „Kde začať, vaša vý-
sosť?“

„Začni od začiatku,“ povedal Kráľ vážne, „čítaj, a keď
prídeš na koniec, tak prestaň.“

V sieni ostalo ticho a Biely králik čítal:

„Rieklí mi, že si u nej bol

a spomenul ma sám;

vraj z dobrého som plota kôl,
len žiaľ, že neplávam.

102

Alica v krajine zázrakov

Dostal som odkaz: Zúrila.
Pravdivo to tam stálo.

Ak by to príliš súrila,

čo by sa s Tebou stalo?

Dal som jej jeden, dostal päť,
Ty si vzal tri (lesť umnál);
tak vrátili sa Tebe späť
všetky, čo boli u mňa.

 

Ak sme sa ja či ona (píš!)
zaplietli do tej siete,
verí,- že všetkých zachrániš
ako nás vtedy v lete.

Dodnes si myslím, že si bol
(než dostala ten záchvat)
prekážkou medzi ním a mnou
a tým — nuž pánbohzaplať!

Len nezrad, ak máš v tele česť,
ako ich ona ľúbi,

tá tajnosť nech už ododnes

len mňa a Teba snúbi.

„To je najzávažnejší dôkaz, aký sme si kedy vypočuli,“
povedal Kráľ a mädlil si ruky. „Nuž a teraz nech poro-
ta -“

„Ak niekto z poroty objasní zmysel týchto veršov,“
povedala Alica (v poslednej chvíli tak vyrástla, že sa už ani
trocha nebála skočiť Kráľovi do reči), „dám mu dvojkom—
náčku. Ja vravím, že nemajú ani byľku zmyslu.“

Celá porota si zapisovala na tabuľky: „Ona vraví, že
nemajú ani byľku zmyslu, ale nik sa neprihlásil, že verše
vysvet '.“

„Ak nemajú nijaký zmysel,“ povedal Kráľ, „ušetrí nám to

104

Alicina výpoveď

veľa práce, lebo ho nemusíme hľadať. A jednako, neviem,
neviem,“ rozložil si verše na kolene a jedným okom do nich
pozeral, „istý zmysel v nich vari predsa len vidím — ,len žiaľ,
že neplávam“ — a ty nevieš plávať, či vieš?“ obrátil sa
k Dolníkovi.

Dolm'k smutne potriasol hlavou: „Vari na to vyzerám?“
spýtal sa. (Naozaj na to nevyzeral, lebo bol z lepenky.)

„No prosím, zatiaľ to súhlasí,“ povedal Kráľ a dalej si
mrmlal verše: „ ,Dostal som odkaz“ — to nech si porota
poznačí — ,zúrila' - toto bude istotne o Kráľovnej — ,Ak by
to príliš súrila, čo by sa s Tebou stalo?“ - to by som aj ja rád
vedel — ,Dal som jej jeden, dostal päť, Ty si vzal tri“ - no
prosím, tu hovorí, čo robil s tými koláčmi, to je úplne
jasné.“

„Ale dalej tam stojí: ,tak vrátili sa Tebe späť všetky, čo
boli u mňa',“ povedala Alica.

„No a tu sú,“ povedal Král víťazoslávne a ukázal prstom
na misu. „J asné ako facka. Poďme dalej — ,než dostala ten
záchvat“ - čo sa pamätám, drahá, vy ste nikdy nemali
záchvat,“ povedal Kráľovnej.

„Nikdy!“ napajedila sa Kráľovná a za reči šmarila do Vila
kalamár. (Chudáčik Vilo už prestal písať na tabuľku prstom,
lebo zistil, že po takom písaní na tabuľke nič nevidieť; teraz
sa však znova chytro dal do písania; písal atramentom, čo mu

 

Alica v krajine zázrakov

stekal po tvári, a chcel využiť každú chvíľku, pokým úplne
nestečie.)

„Nuž v tomto prípade slová na vás neplatia,“ povedal
Kráľ a s úsmevom sa rozhliadol po sieni. Bolo v nej ticho ani
v hrobe.

„To je iba taká slovná hra!“ vysvetlil Kráľ nahnevane
a všetko sa rozosmialo. „Nech sa porota poradí 0 rozsudku,“
povedal v ten deň hádam už dvadsiaty raz.

„Kdeže! Kdeže!“ povedala Kráľovná. „Najprv trest
— potom rozsudok.“

„Čo je to za sprostosť!“ povedala Alica nahlas. „Kto to
kedy počul — najprv trest.“

„Ty čuš!“ povedala Kráľovná.

„A nebudem!“ povedala Alica.

„Zotnite jej hlavu!“ zjačala Kráľovná.

Nik sa ani len nepohol.

„Kto by sa vás bál?!“ povedala Alica (vtedy už mala zasa

normálnu výšku). „Veď ste len obyčajné karty!“
[Vi/tom všetky karty vzlietli do vzduchu a vzápätí sa zniesli
na ňu; Alica chabo vykríkla, sčasti od strachu, sčasti od
hnevu, a pokúsila sa striasť ich zo seba; tu zistila, že leží na
brehu rieky, s hlavou na sestrinom lone a sestra jej z tváre
zhadzuje suché lístie, čo jej tam napadalo zo stromu.

„Prebuď sa, Alica moja!“ vravela jej sestra. „Ale si si

„Ach, taký čudný sen sa mi prisnil,“ povedala Alica
a rozpovedala sestre, nakoľko si pamätala, všetky tie čudes-
né príhody, o ktorých ste práve čítali; keď dorozprávala,
sestra ju pobozkala a povedala: „Bol to, milá moja, čudný
sen, ale teraz sa bež naolovrantovať, lebo sa už stmieva.“

Alica vstala, rozbehla sa a celou cestou rozmýšľala, aký to
bol nádherný sen.

Sestra tam zostala sedieť, jednou rukou si podopierala
hlavu, pozorovala zapadajúce slnko a rozmýšľala o prího-
dách malej Alice. Napokon začala aj ona snívať a mala
takýto sen:

106

Alicina výpoveď

Najprv sa jej snívalo s Alicou, ako jej rúčkami objíma
kolená a jasnými zvedavými očkami dychtivo hľadí do jej
očí, počula zvuk jej hlasu a videla ju pred sebou, ako
známym pohybom pohadzuje hlavou a striasa dozadu
neposedné vlásky, ktoré jej jednostaj padali do tváre.
Načúvala teda, alebo aspoň sa jej zdalo, že načúva, a všetko
okolo nej ožilo stvoreniami zo sna jej sestričky.

Pri nohách jej zašušťala tráva, keď tade prebehol Biely
králik — v mláke obďaleč zašpliechala vyľakaná Myš
— počula cvengot čajových šálok pri nekonečnom olovrante
Aprílového zajaca a jeho druhov — a prenikavý Kráľovnin
hlas, keď vydávala rozkazy na popravy chudákov hostí
— kýchanie prasacieho bábätka vo Vojvodkyninom náručí
— Grifónove škreky, škrípanie Vilovho grifľ a na bridlicovej
tabuľke i pridusený piskot „eliminovaných“ morčiat — všet—
ko to napĺňalo vzduch a miešalo sa so vzdialenými vzlykmi
nešťastnej Falošnej korytnačky.

A tak tam sedela so zavretými očami a chcela veriť, že aj
ona sa ocitla v krajine zázrakov, hoci vedela, že len čo otvorí
oči, všetko sa znova zmení na všednú skutočnosť — tráva
bude šušťať len od vetra a mláku rozčerí len rozhojdané
tŕstie - cvengot čajových šálok sa zmení na zvonce pasúcich
sa oviec a Kráľovnine prenikaVé výkriky na pastierikov hlas
— kýchanie prasiatka, Grifónove škreky a všetko ostatné sa
ukáže (dobre to vedela) len ako každodenná mätež zvukov
na neďalekom gazdovstve a namiesto srdcervúcich vzlykov
Falošnej korytnačky doletí k nej z diaľky iba ťahavé bučanie
dobytka.

Napokon si predstavila, ako jej sestrička bude po rokoch
dospelou ženou a ako si aj v zrelom veku uchová čisté
a vrúcne srdce; ako bude priťahovať malé deti a rozsvecovať
im oči čudesnými rozprávkami — možno aj svojím dávnym
snom o krajine zázrakov; ako sa bude vžívať do ich malých
zánnutkov a nachádzať potešenie vo všetkých ich drobných
radostiach, lebo stále bude mať pred očami šťastné dni
svojho vlastného detstva.
}  % closing bracket for \ParallelRText

\end{Parallel}
\end{document}